\chapter{Bayesian model average}\label{chap10}

\section{Solutions of Exercises}\label{sec101}
\begin{enumerate}[leftmargin=*]

	\item The Gaussian linear model specifies $\bf{y}=\alpha\bm{i}_N+\bm{X}_m\bm{\beta}_m+\bm{\mu}_m$ such that $\bm{\mu}_m\sim{N}(\bm{0},\sigma^2\bm{I}_n)$, and $\bm{X}_m$ does not have the column of ones. Assuming that $\pi(\sigma^2)\propto 1/{\sigma^2}$, $\pi(\alpha)\propto 1$, and $\bm{\beta}_m|\sigma^2 \sim {N}(\bm{0}_{k_m}, \sigma^2 (g_m\bm{X}_m^{\top}\bm{X}_m)^{-1})$.
\begin{itemize}
	\item Show that the posterior conditional distribution of $\bm{\beta}_m$ is $N(\bm{\beta}_{mn},\sigma^2\bm{B}_{mn})$, where $\bm{\beta}_{mn}=\bm{B}_{mn}\bm{X}_m^{\top}\bm{y}$ and $\bm{B}_{mn}=((1+g_m)\bm{X}_m^{\top}\bm{X}_m)^{-1}$.
	\item Show that the marginal the marginal likelihood associated with model $\mathcal{M}_m$ is proportional to
	\begin{align*}
		p(\bm{y}|\mathcal{M}_m)&\propto \left(\frac{g_m}{1+g_m}\right)^{k_m/2} \left[(\bm{y}-\bar{y}\bm{i}_N)^{\top}(\bm{y}-\bar{y}\bm{i}_N)-\frac{1}{1+g_m}(\bm{y}^{\top}\bm{P}_{X_m}\bm{y})\right]^{-(N-1)/2},
	\end{align*}
	where all parameter are indexed to model $\mathcal{M}_m$, $\bm{P}_{X_m}=\bm{X}_m(\bm{X}_m^{\top}\bm{X}_m)^{-1}\bm{X}_m$ is the projection matrix on the space generated by the columns of $\bm{X}_m$, and $\bar{y}$ is the sample mean of $\bm{y}$.
	
	Hint: Take into account that $\bm{i}_N^{\top}\bm{X}_m=\bm{0}_{k_m}$ due to all columns being centered with respect to their means.
\end{itemize}
\textbf{Answer}

The marginal likelihood of this model is

\begin{align*}
	p({\bf{y}})&=\int_0^{\infty}\int_{R^K}\int_R\pi (\bm{\beta} | \sigma^2)\pi(\sigma^2)\pi(\alpha)p({\bf{y}}|\bm{\beta}, \sigma^2, \alpha)d\alpha d\bm{\beta} d\sigma^2\\
	&\propto \int_0^{\infty}\int_{R^K}\int_R (\sigma^2)^{-k_m/2} |g_m\bm{X}_m^{\top}\bm{X}_m|^{1/2}\exp\left\{-\frac{1}{2\sigma^2}(\bm{\beta}^{\top}(g_m\bm{X}_m^{\top}\bm{X}_m)\bm{\beta})\right\}(\sigma^2)^{-1}\\
	&\times (\sigma^2)^{-N/2}\exp\left\{-\frac{1}{2\sigma^2}(\bm{y}-\alpha\bm{i}_N-\bm{X}_m\bm{\beta})^{\top}(\bm{y}-\alpha\bm{i}_N-\bm{X}_m\bm{\beta})\right\}d\alpha d\bm{\beta} d\sigma^2.
\end{align*}
Taking into account that $(\bm{y}-\alpha\bm{i}_N-\bm{X}_m\bm{\beta})^{\top}(\bm{y}-\alpha\bm{i}_N-\bm{X}_m\bm{\beta})=(\bm{y}-\bm{X}_m\bm{\beta})^{\top}(\bm{y}-\bm{X}_m\bm{\beta})+N(\alpha-\bar{y})^2-N\bar{y}^2$, 

\begin{align*}
	p({\bf{y}})	&\propto \int_0^{\infty}\int_{R^K} (\sigma^2)^{-k_m/2} |g_m\bm{X}_m^{\top}\bm{X}_m|^{1/2}\exp\left\{-\frac{1}{2\sigma^2}(\bm{\beta}^{\top}(g_m\bm{X}_m^{\top}\bm{X}_m)\bm{\beta})\right\}(\sigma^2)^{-1}\\
	&\times (\sigma^2)^{-N/2}\exp\left\{-\frac{1}{2\sigma^2}[(\bm{y}-\bm{X}_m\bm{\beta})^{\top}(\bm{y}-\bm{X}_m\bm{\beta})-N\bar{y}^2]\right\}\\
	&\times \int_R \exp\left\{-\frac{1}{2\sigma^2}(N(\alpha-\bar{y})^2)\right\} d\alpha d\bm{\beta} d\sigma^2.
\end{align*}

The last term is the kernel of a normal density function with mean $\bar{y}$ and variance $\sigma^2/N$, then

\begin{align*}
	p({\bf{y}})	&\propto \int_0^{\infty}\int_{R^K} (\sigma^2)^{-k_m/2} |g_m\bm{X}_m^{\top}\bm{X}_m|^{1/2}\exp\left\{-\frac{1}{2\sigma^2}(\bm{\beta}^{\top}(g_m\bm{X}_m^{\top}\bm{X}_m)\bm{\beta})\right\}(\sigma^2)^{-1}\\
	&\times (\sigma^2)^{-N/2}(\sigma^2)^{1/2}\exp\left\{-\frac{1}{2\sigma^2}[(\bm{y}-\bm{X}_m\bm{\beta})^{\top}(\bm{y}-\bm{X}_m\bm{\beta})-N\bar{y}^2]\right\}d\bm{\beta} d\sigma^2.
\end{align*}

Collecting terms for $\bm{\beta}$, we have $(\bm{y}-\bm{X}_m\bm{\beta})^{\top}(\bm{y}-\bm{X}_m\bm{\beta})-N\bar{y}^2+\bm{\beta}^{\top}(g_m\bm{X}_m^{\top}\bm{X}_m)\bm{\beta}=(\bm{\beta}-\bm{\beta}_{mn})^{\top}\bm{B}_{mn}^{-1}(\bm{\beta}-\bm{\beta}_{mn})+\bm{y}^{\top}\bm{y}-N\bar{y}^2-\bm{\beta}_{mn}^{\top}\bm{B}_{mn}\bm{\beta}_{mn}$ where $\bm{\beta}_{mn}=\bm{B}_{mn}\bm{X}^{\top}\bm{y}$ and $\bm{B}_{mn}=((1+g_m)\bm{X}^{\top}\bm{X})^{-1}$. Then,

\begin{align*}
	p({\bf{y}})	&\propto \int_0^{\infty} (\sigma^2)^{-k_m/2} |g_m\bm{X}_m^{\top}\bm{X}_m|^{1/2}\exp\left\{-\frac{1}{2\sigma^2}(\bm{y}^{\top}\bm{y}-N\bar{y}^2-\bm{\beta}_{mn}^{\top}\bm{B}_{mn}\bm{\beta}_{mn})\right\}(\sigma^2)^{-1}\\
	&\times (\sigma^2)^{-N/2}(\sigma^2)^{1/2}\int_{R^K}\exp\left\{-\frac{1}{2\sigma^2}(\bm{\beta}-\bm{\beta}_{mn})^{\top}\bm{B}_{mn}^{-1}(\bm{\beta}-\bm{\beta}_{mn})\right\}d\bm{\beta} d\sigma^2.
\end{align*}
The last term is the kernel of a multivariate normal density with mean $\bm{\beta}_{mn}$ and variance $\bm{B}_{mn}$ (proof of the first bullet). Then,

\begin{align*}
	p({\bf{y}})	&\propto \int_0^{\infty}\left(\frac{g_m}{1+g_m}\right)^{k_m/2}(\sigma^2)^{-(N-1)/2-1} \exp\left\{-\frac{1}{2\sigma^2}(\bm{y}^{\top}\bm{y}-N\bar{y}^2-\bm{\beta}_{mn}^{\top}\bm{B}_{mn}\bm{\beta}_{mn})\right\}d\sigma^2.
\end{align*}

This is the kernel of an inverse-gamma density with parameters $\alpha_n=(N-1)/2$ and $\delta_n=\bm{y}^{\top}\bm{y}-N\bar{y}^2-\bm{\beta}_{mn}^{\top}\bm{B}_{mn}\bm{\beta}_{mn}$, where $\bm{y}^{\top}\bm{y}-N\bar{y}^2-\bm{\beta}_{mn}^{\top}\bm{B}_{mn}\bm{\beta}_{mn}=(\bm{y}-\bm{i}_N\alpha)^{\top}(\bm{y}-\bm{i}_N\alpha)-(1+g_m)^{-1}\bm{y}^{\top}(\bm{X}(\bm{X}^{\top}\bm{X})^{-1}\bm{X}^{\top})\bm{y}$. Then,

\begin{align*}
	p(\bm{y}|\mathcal{M}_m)&\propto \left(\frac{g_m}{1+g_m}\right)^{k_m/2} \left[(\bm{y}-\bar{y}\bm{i}_N)^{\top}(\bm{y}-\bar{y}\bm{i}_N)-\frac{1}{1+g_m}(\bm{y}^{\top}\bm{P}_{X_m}\bm{y})\right]^{-(N-1)/2},
\end{align*}

\item \textbf{Determinants of export diversification I}

\cite{Jetter2015} use BMA to study the determinants of export diversification. Use the dataset \textit{10ExportDiversificationHHI.csv} to perform BMA using the BIC approximation and MC3 with 10000 iterations to check if these two approaches agree.

\textbf{Answer}

The first aspect to note is that the BIC approximation is faster by far than the MC3 algorithm. We see from the results that the two approaches show that the most relevant variables to determine export diversification are \textit{avgedu5} (primary education) and \textit{avgnatres} (natural resources). See \cite{Jetter2015} for details of all variables. The model with the highest PMP using MC3 includes these two variables (PMP = 0.16), while the model with the highest PMP using the BIC approximation in addition to these two variables also includes Portugal former colony and population (PMP = 0.03). Both methods agree that export diversification increases with primary education, and decreases with natural resources.     

	\begin{tcolorbox}[enhanced,width=4.67in,center upper,
	fontupper=\large\bfseries,drop shadow southwest,sharp corners]
	\textit{R code. Determinants of export diversification}
	\begin{VF}
		\begin{lstlisting}[language=R]
rm(list = ls()); set.seed(010101)
Data <- read.csv("https://raw.githubusercontent.com/besmarter/BSTApp/refs/heads/master/DataApp/10ExportDiversificationHHI.csv", sep = ",", header = TRUE, quote = "")
attach(Data)
y <- Data[,1]; X <- as.matrix(Data[,-1]); K <- dim(X)[2]
BMAglm <- BMA::bicreg(X, y, strict = FALSE, OR = 50)
summary(BMAglm)
BMAreg <- BMA::MC3.REG(y, X, num.its=10000)
Models <- unique(BMAreg[["variables"]])
nModels <- dim(Models)[1]
nVistModels <- dim(BMAreg[["variables"]])[1]
PMPmc3 <- NULL
for(m in 1:nModels){
	idModm <- NULL
	for(j in 1:nVistModels){
		if(sum(Models[m,] == BMAreg[["variables"]][j,]) == K){
			idModm <- c(idModm, j)
		}else{
			idModm <- idModm
		} 
	}
	PMPm <- sum(BMAreg[["post.prob"]][idModm])
	PMPmc3 <- c(PMPmc3, PMPm)
}
PMPmc3
PIPmc3 <- NULL
for(k in 1:K){
	PIPk <- sum(PMPmc3[which(Models[,k] == 1)])
	PIPmc3 <- c(PIPmc3, PIPk)
}
plot(PIPmc3)
Meansmc3 <- matrix(0, nModels, K)
Varsmc3 <- matrix(0, nModels, K)
for(m in 1:nModels){
	idXs <- which(Models[m,] == 1)
	if(length(idXs) == 0){
		Regm <- lm(y ~ 1)
	}else{
		Xm <- X[, idXs]
		Regm <- lm(y ~ Xm)
		SumRegm <- summary(Regm)
		Meansmc3[m, idXs] <- SumRegm[["coefficients"]][-1,1]
		Varsmc3[m, idXs] <- SumRegm[["coefficients"]][-1,2]^2 
	}
}
BMAmeansmc3 <- colSums(Meansmc3*PMPmc3)
BMAsdmc3 <- (colSums(PMPmc3*Varsmc3)  + colSums(PMPmc3*(Meansmc3-matrix(rep(BMAmeansmc3, each = nModels), nModels, K))^2))^0.5 
plot(BMAmeansmc3)
plot(BMAsdmc3)
Ratio <- BMAmeansmc3/BMAsdmc3
RessulMC3 <- as.data.frame(cbind(PIPmc3, Models[1,], BMAmeansmc3, BMAsdmc3, Ratio))
PMPmc3
\end{lstlisting}
	\end{VF}
\end{tcolorbox} 
 

\item \textbf{Simulation exercise of the Markov chain Monte Carlo model composition continues}

Program an algorithm to perform MC3 where the final $S$ models are unique. Use the simulation setting of Section 10.2 of the book increasing the number of regressors to 40, this implies approximately 1.1e+12 models. 	
	
	\textbf{Answer}
	
The following code shows how to perform BMA using MC3 with the consideration that all final $S$ models should be different.

After running the algorithm 50000 ($<<2^{40}$) times, we can see that the PIP is 1 for variables $x_1$, $x_5$ and $x_{10}$, which are the variables in the data generating process (population statistical model). However, we can see that variable $x_{23}$ has a high PIP (0.49), this makes that the PMP of the model including $x_1$, $x_5$, $x_{10}$ and $x_{23}$ is the highest, followed by the model including $x_1$, $x_5$ and $x_{10}$, which is the population statistical model. This highlights the relevance of performing BMA; selecting just one model based on the highest PMP would induce a mistake. Although selecting the median probability model would uncover the population statistical model. 

Estimating the BMA mean shows that we get values very close to the population values, a remarkable results is all other BMA posterior means are close to 0, including the mean coefficient of $x_{23}$ despite that its PIP is almost 0.5. We calculate the posterior ratio between the mean and standard deviation, and get values higher than 2 just for $x_1$, $x_5$ and $x_{10}$, again given evidence for the data generating process.     
	
	\begin{tcolorbox}[enhanced,width=4.67in,center upper,
		fontupper=\large\bfseries,drop shadow southwest,sharp corners]
		\textit{R code. Markov chain Monte Carlo model composition}
		\begin{VF}
			\begin{lstlisting}[language=R]
rm(list = ls()); set.seed(010101)
N <- 1000
K1 <- 6; K2 <- 4; K3 <- 30; K <- K1 + K2 + K3
X1 <- matrix(rnorm(N*K1,1 ,1), N, K1)
X2 <- matrix(rbinom(N*K2, 1, 0.5), N, K2)
X3 <- matrix(rnorm(N*K3,1 ,1), N, K3)
X <- cbind(X1, X2, X3); e <- rnorm(N, 0, 0.5)
B <- c(1,0,0,0,0.5,0,0,0,0,-0.7, rep(0, 30))
y <- 1 + X%*%B + e
LogMLfunt <- function(Model){
	indr <- Model == 1
	kr <- sum(indr)
	if(kr > 0){
		gr <- ifelse(N > kr^2, 1/N, kr^(-2))
		Xr <- matrix(Xnew[ , indr], ncol = kr)
		# PX <- diag(N) - Xr%*%solve(t(Xr)%*%Xr)%*%t(Xr)
		# s2pos <- c(t(y)%*%PX%*%y/(1 + gr) + gr*(t(y - mean(y))%*%(y - mean(y)))/(1 + gr))
		PX <- Xr%*%solve(t(Xr)%*%Xr)%*%t(Xr)
		s2pos <- c((t(y - mean(y))%*%(y - mean(y))) - t(y)%*%PX%*%y/(1 + gr))
		mllMod <- (kr/2)*log(gr/(1+gr))-(N-1)/2*log(s2pos)
	}else{
		gr <- ifelse(N > kr^2, 1/N, kr^(-2))
		# PX <- diag(N)
		# s2pos <- c(t(y)%*%PX%*%y/(1 + gr) + gr*(t(y - mean(y))%*%(y - mean(y)))/(1 + gr))
		s2pos <- c((t(y - mean(y))%*%(y - mean(y))))
		mllMod <- (kr/2)*log(gr/(1+gr))-(N-1)/2*log(s2pos)
	}
	return(mllMod)
}
Xnew <- apply(X, 2, scale); M <- 100
Models <- matrix(rbinom(K*M, 1, p = 0.5), ncol = K, nrow = M + 800)
Models <- unique(Models)[1:M,]
mllnew <- sapply(1:M, function(s){LogMLfunt(matrix(Models[s,], 1, K))})
oind <- order(mllnew, decreasing = TRUE)
mllnew <- mllnew[oind]
Models <- Models[oind, ]
iter <- 50000; s <- 1
pb <- winProgressBar(title = "progress bar", min = 0, max = iter, width = 300)
\end{lstlisting}
		\end{VF}
	\end{tcolorbox} 

\begin{tcolorbox}[enhanced,width=4.67in,center upper,
	fontupper=\large\bfseries,drop shadow southwest,sharp corners]
	\textit{R code. Markov chain Monte Carlo model composition}
	\begin{VF}
		\begin{lstlisting}[language=R]
while(s <= iter){
	ActModel <- Models[M,]
	idK <- which(ActModel == 1)
	Kact <- length(idK)
	Continue <- 0
	while(Continue == 0){
		if(Kact < K & Kact > 1){
			CardMol <- K
			opt <- sample(1:3, 1)
			if(opt == 1){ # Same
				CandModel <- ActModel
			}else{
				if(opt == 2){ # Add
					All <- 1:K
					NewX <- sample(All[-idK], 1)
					CandModel <- ActModel
					CandModel[NewX] <- 1
				}else{ # Subtract
					LessX <- sample(idK, 1)
					CandModel <- ActModel
					CandModel[LessX] <- 0
				}
			}
		}else{
			CardMol <- K + 1
			if(Kact == K){
				opt <- sample(1:2, 1)
				if(opt == 1){ # Same
					CandModel <- ActModel
				}else{ # Subtract
					LessX <- sample(1:K, 1)
					CandModel <- ActModel
					CandModel[LessX] <- 0
				}
			}else{
				if(K == 1){
					opt <- sample(1:3, 1)
					if(opt == 1){ # Same
						CandModel <- ActModel
					}else{
						if(opt == 2){ # Add
							All <- 1:K
							NewX <- sample(All[-idK], 1)
							CandModel <- ActModel
							CandModel[NewX] <- 1
						}else{ # Subtract
							LessX <- sample(idK, 1)
							CandModel <- ActModel
							CandModel[LessX] <- 0
						}
					}
				}else{ # Add
					NewX <- sample(1:K, 1)
					CandModel <- ActModel
					CandModel[NewX] <- 1
				}
			}
		}
\end{lstlisting}
	\end{VF}
\end{tcolorbox} 

\begin{tcolorbox}[enhanced,width=4.67in,center upper,
	fontupper=\large\bfseries,drop shadow southwest,sharp corners]
	\textit{R code. Markov chain Monte Carlo model composition}
	\begin{VF}
		\begin{lstlisting}[language=R]
		check <- NULL
	for(j in 1:M){
		if(sum(Models[j,] == CandModel) == K){
			checkj <- 0
			check <- c(check, checkj)
		}else{
			checkj <- 1
			check <- c(check, checkj)
		}
	}
	dimUniModels <- sum(check)
	if(dimUniModels == M){
		Continue <- 1
	}else{
		Continue <- 0
	}
}
LogMLact <- LogMLfunt(matrix(ActModel, 1, K))
LogMLcand <- LogMLfunt(matrix(CandModel, 1, K))
alpha <- min(1, exp(LogMLcand-LogMLact)); u <- runif(1)
if(u <= alpha){
mllnew[M] <- LogMLcand
Models[M, ] <- CandModel
oind <- order(mllnew, decreasing = TRUE)
mllnew <- mllnew[oind]
Models <- Models[oind, ]
}else{
mllnew <- mllnew
Models <- Models
}
s <- s + 1
setWinProgressBar(pb, s, title=paste( round(s/iter*100, 0),"% done"))
}
close(pb)
ModelsUni <- unique(Models)
mllnewUni <- sapply(1:dim(ModelsUni)[1], function(s){LogMLfunt(matrix(ModelsUni[s,], 1, K))})
StMarLik <- exp(mllnewUni-mllnewUni[1])
PMP <- StMarLik/sum(StMarLik)
PIP <- NULL
for(k in 1:K){
PIPk <- sum(PMP[which(ModelsUni[,k] == 1)])
PIP <- c(PIP, PIPk)
}
PIP
\end{lstlisting}
	\end{VF}
\end{tcolorbox} 

\begin{tcolorbox}[enhanced,width=4.67in,center upper,
	fontupper=\large\bfseries,drop shadow southwest,sharp corners]
	\textit{R code. Markov chain Monte Carlo model composition}
	\begin{VF}
		\begin{lstlisting}[language=R]
Means <- matrix(0, M, K)
Vars <- matrix(0, M, K)
for(m in 1:M){
	idXs <- which(ModelsUni[m,] == 1)
	if(length(idXs) == 0){
		Regm <- lm(y ~ 1)
	}else{
		Xm <- X[, idXs]
		Regm <- lm(y ~ Xm)
		SumRegm <- summary(Regm)
		Means[m, idXs] <- SumRegm[["coefficients"]][-1,1]
		Vars[m, idXs] <- SumRegm[["coefficients"]][-1,2]^2 
	}
}
BMAmeans <- colSums(Means*PMP)
BMAsd <- (colSums(PMP*Vars)  + colSums(PMP*(Means-matrix(rep(BMAmeans, each = M), M, K))^2))^0.5 
plot(BMAmeans)
plot(BMAsd)
plot(BMAmeans/BMAsd)
\end{lstlisting}
\end{VF}
\end{tcolorbox}

\item \textbf{Simulation exercise of IV BMA continues}

Use the simulation setting with endogeneity in Section 10.2 to perform BMA based on the BIC approximation and MC3.

\textbf{Answer}

The following code shows how to perform BMA using the BIC approximation and MC3 in this simulation setting with endogeneity. We see from the results that the BIC approximation and MC3 do a good job with the PMP and the PIP, as the data generating process gets the highest PMP using both approaches, and the PIPs of the variables in the data generating process are equal 1. The critical point is the BMA posterior means of the endogenous regressors, as these are far from the population values. The population values of $x_{i1}$ and $x_{i2}$ are 0.5 and -1, whereas the posterior means are 0.97 and -0.52. 

\begin{tcolorbox}[enhanced,width=4.67in,center upper,
	fontupper=\large\bfseries,drop shadow southwest,sharp corners]
	\textit{R code. BIC and MC3 in model with endogeneity}
	\begin{VF}
		\begin{lstlisting}[language=R]
rm(list = ls())
set.seed(010101)
simIV <- function(delta1,delta2,beta0,betas1,betas2,beta2,Sigma,n,z) {
	eps <- matrix(rnorm(3*n),ncol=3) %*% chol(Sigma)
	xs1 <- z%*%delta1 + eps[,1]
	xs2 <- z%*%delta2 + eps[,2]
	x2 <- rnorm(dim(z)[1])
	y <- beta0+betas1*xs1+betas2*xs2+beta2*x2 + eps[,3]
	X <- as.matrix(cbind(xs1,xs2,1,x2)) 
	colnames(X) <- c("x1en","x2en","cte","xex")
	y <- matrix(y,dim(z)[1],1)
	colnames(y) <- c("y")
	list(X=X,y=y)
}
n <- 1000 ; p <- 3 
z <- matrix(runif(n*p),ncol=p)
rho31 <- 0.8; rho32 <- 0.5;
Sigma <- matrix(c(1,0,rho31,0,1,rho32,rho31,rho32,1),ncol=3)
delta1 <- c(4,-1,2); delta2 <- c(-2,3,-1); betas1 <- .5; betas2 <- -1; beta2 <- 1; beta0 <- 2
simiv <- simIV(delta1,delta2,beta0,betas1,betas2,beta2,Sigma,n,z)
nW <- 18
W <- matrix(rnorm(nW*dim(z)[1]),dim(z)[1],nW)
YXW<-cbind(simiv$y, simiv$X, W)
y <- YXW[,1]; X <- YXW[,2:3]; W <- YXW[,-c(1:4)]
Xnew <- cbind(X, W)
BMAglm <- BMA::bicreg(Xnew, y, strict = FALSE, OR = 50) 
summary(BMAglm)
BMAreg <- BMA::MC3.REG(y, Xnew, num.its=10000)
Models <- unique(BMAreg[["variables"]])
nModels <- dim(Models)[1]
nVistModels <- dim(BMAreg[["variables"]])[1]
K <- dim(Xnew)[2]
PMP <- NULL
for(m in 1:nModels){
	idModm <- NULL
	for(j in 1:nVistModels){
		if(sum(Models[m,] == BMAreg[["variables"]][j,]) == K){
			idModm <- c(idModm, j)
		}else{
			idModm <- idModm
		} 
	}
	PMPm <- sum(BMAreg[["post.prob"]][idModm])
	PMP <- c(PMP, PMPm)
}
PMP
PIP <- NULL
for(k in 1:K){
	PIPk <- sum(PMP[which(Models[,k] == 1)])
	PIP <- c(PIP, PIPk)
}
plot(PIP)
\end{lstlisting}
	\end{VF}
\end{tcolorbox} 

\begin{tcolorbox}[enhanced,width=4.67in,center upper,
	fontupper=\large\bfseries,drop shadow southwest,sharp corners]
	\textit{R code. BIC and MC3 in model with endogeneity}
	\begin{VF}
		\begin{lstlisting}[language=R]
Means <- matrix(0, nModels, K)
Vars <- matrix(0, nModels, K)
for(m in 1:nModels){
	idXs <- which(Models[m,] == 1)
	if(length(idXs) == 0){
		Regm <- lm(y ~ 1)
	}else{
		Xm <- Xnew[, idXs]
		Regm <- lm(y ~ Xm)
		SumRegm <- summary(Regm)
		Means[m, idXs] <- SumRegm[["coefficients"]][-1,1]
		Vars[m, idXs] <- SumRegm[["coefficients"]][-1,2]^2 
	}
}
BMAmeans <- colSums(Means*PMP)
BMAsd <- (colSums(PMP*Vars)  + colSums(PMP*(Means-matrix(rep(BMAmeans, each = nModels), nModels, K))^2))^0.5 
plot(BMAmeans)
plot(BMAsd)
plot(BMAmeans/BMAsd)
\end{lstlisting}
	\end{VF}
\end{tcolorbox}

The previous results are intuitive, as the PMP are calculated based on fit (and penalty for complexity), for instance, $BIC=k_m\log(N)-2\log(p(\hat{\bm{\theta}_m}|\bm{y}))$, where $\hat{\bm{\theta}}_m$ is the maximum likelihood estimator. Observe that the fit is not affected by endogeneity. Thus, the PMP are well calculated. However, the coefficients are not well identified.  

This suggests a simple strategy to perform BMA taking into account endogeneity, calculate the PMPs using standard BMA approaches, for instance, BIC approximation, and then estimate the BMA means using these PMPs, but estimating the different models using instrumental variables. This approach is easily implemented using packages from \textbf{R}. The following code does this:

\begin{tcolorbox}[enhanced,width=4.67in,center upper,
	fontupper=\large\bfseries,drop shadow southwest,sharp corners]
	\textit{R code. Easy IV BMA}
	\begin{VF}
		\begin{lstlisting}[language=R]
BMAglm <- BMA::bicreg(Xnew, y, strict = FALSE, OR = 50) 
summary(BMAglm)
PMPBIC <- BMAglm[["postprob"]]
ModelsBIC <- BMAglm[["which"]]
nModels <- dim(ModelsBIC)[1]
K <- dim(Xnew)[2]
Means <- matrix(0, nModels, K)
Vars <- matrix(0, nModels, K)
for(m in 1:nModels){
	idXs <- which(ModelsBIC[m,] == 1)
	if(length(idXs) == 0){
		Regm <- lm(y ~ 1)
	}else{
		Xm <- Xnew[, idXs]
		Regm <- ivreg::ivreg(y ~ Xm | z + W)
		SumRegm <- summary(Regm)
		Means[m, idXs] <- SumRegm[["coefficients"]][-1,1]
		Vars[m, idXs] <- SumRegm[["coefficients"]][-1,2]^2 
	}
}
BMAmeans <- colSums(Means*PMPBIC)
BMAsd <- (colSums(PMPBIC*Vars)  + colSums(PMPBIC*(Means-matrix(rep(BMAmeans, each = nModels), nModels, K))^2))^0.5 
BMAmeans
5.589366e-01 -9.431664e-01  1.035593e+00 -1.967444e-05 -1.429801e-04 -3.310215e-04  4.036846e-04 4.651796e-03 -2.673730e-03  6.462503e-05  4.286529e-04 -1.829889e-04  5.073229e-03 -1.007356e-04 -5.377972e-04  1.644475e-03 -4.029205e-04 -1.457381e-04  6.421582e-04  2.211994e-04  2.216503e-04
plot(BMAsd)
plot(BMAmeans/BMAsd)
17.764095075 -23.233601133  34.734610916  -0.005600369  -0.039168667  -0.071895067   0.080909448 0.274994618  -0.210854131   0.018790592   0.061271650  -0.047117887   0.305639285  -0.028830411 -0.092463544   0.169455952  -0.058299940  -0.038099792   0.102373567   0.056074670   0.055500343
		\end{lstlisting}
	\end{VF}
\end{tcolorbox} 

We observe that the Easy IV BMA means of the endogenous regressors are 0.59 and -0.94, which are closer to the population values (0.5 and -1) than the exogenous BMA means (0.97 and -0.52) and align closely with the IV BMA means calculated using conditional Bayes factors (0.51 and -0.98; see Section 10.2 in the book). Additionally, the ratios between the Easy IV BMA means and their standard deviations exceed 2 in absolute value for these variables, and the t-intervals encompass the population values   

\item \textbf{Determinants of export diversification II}

Use the datasets \textit{11ExportDiversificationHHI.csv} and \textit{12ExportDiversificationHHIInstr.csv} to perform IV BMA assuming that the log of per capita gross domestic product is endogenous (\textit{avglgdpcap}). See \cite{Jetter2015} for details.  

\textbf{Answer}

The results show that primary education (\textit{avgedu5}) and natural resources (\textit{avgnatres}) have again the highest PIP, 0.77 and 0.92, respectively. The former variable increases export diversification, and the latter decreases export diversification. However, it seems that endogeneity is not a concern in this application, as the 95\% credible interval of $\sigma_{12}$ is (-0.014, 0.024).   


\begin{tcolorbox}[enhanced,width=4.67in,center upper,
	fontupper=\large\bfseries,drop shadow southwest,sharp corners]
	\textit{R code. IV BMA in export diversification}
	\begin{VF}
		\begin{lstlisting}[language=R]
########################## Determinants of export diversification: BMA normal model ########################## 
rm(list = ls())
set.seed(010101)
DataMain <- read.csv("https://raw.githubusercontent.com/besmarter/BSTApp/refs/heads/master/DataApp/11ExportDiversificationHHI.csv", sep = ",", header = TRUE, quote = "")
DataInst <- read.csv("https://raw.githubusercontent.com/besmarter/BSTApp/refs/heads/master/DataApp/12ExportDiversificationHHIInstr.csv", sep = ",", header = TRUE, quote = "")
attach(DataMain)
attach(DataInst)
y <- DataMain[,1]
X <- as.matrix(DataMain[,2])
W <- as.matrix(DataMain[,-c(1:2)])
Z <- as.matrix(DataInst)
S <- 10000; burnin <- 1000
regivBMA <- ivbma::ivbma(Y = y, X = X, Z = Z, W = W, s = S+burnin, b = burnin, odens = S, print.every = round(S/10), run.diagnostics = FALSE)
PIPmain <- regivBMA[["L.bar"]] # PIP outcome
PIPmain
EVmain <- regivBMA[["rho.bar"]] # Posterior mean outcome
EVmain
PIPaux <- regivBMA[["M.bar"]] # PIP auxiliary
PIPaux
EVaux <- regivBMA[["lambda.bar"]] # Posterior mean auxiliary
plot(EVaux[,1])
EVsigma <- regivBMA[["Sigma.bar"]] # Posterior mean variance matrix
EVsigma
summary(coda::mcmc(regivBMA[["Sigma"]][1,2,]))
Iterations = 1:10000
Thinning interval = 1 
Number of chains = 1 
Sample size per chain = 10000 
1. Empirical mean and standard deviation for each variable,
plus standard error of the mean:
Mean             SD       Naive SE Time-series SE 
3.944e-03      9.592e-03      9.592e-05      4.043e-04 
2. Quantiles for each variable:
2.5%       25%       50%       75%     97.5% 
-0.014155 -0.002415  0.003592  0.009988  0.024047 
\end{lstlisting}
	\end{VF}
\end{tcolorbox}

\item Show that the link function in the case of the Bernoulli distribution is $\log\left(\frac{\theta}{1-\theta}\right)$.

\textbf{Answer}
\begin{align}
	p(\mathbf{y}|\theta)&=\theta^{y}(1-\theta)^{1-y}\nonumber\\
	&=(1-\theta)\exp\left\{ y\log\left(\frac{\theta}{1-\theta}\right)\right\}\nonumber\\
	&=\exp\left\{ y\log\left(\frac{\theta}{1-\theta}\right)+\log(1-\theta)\right\}\nonumber,
\end{align}

then $\eta(\theta)=\log\left(\frac{\theta}{1-\theta}\right)$, and this density in the canonical form is $p(y|\eta)=\exp\left\{ y\eta-\log(1+\exp(\eta))\right\}$  consequently, $\mathbb{E}[Y|\bm{x}]=\nabla\left(\log(1+\exp(\eta))\right)=\frac{\exp(\eta)}{1+\exp(\eta)}=\theta=\frac{\exp(\bm{x}^{\top}\bm{\beta})}{1+\exp(\bm{x}^{\top}\bm{\beta})}$. Then, the link function in the Bernoulli case is the \textit{logit} function.

\item Use the file \textit{13InternetMed.csv} to perform BMA using the logit link function.

\textbf{Answer}

\item Use the file \textit{14ValueFootballPlayers.csv} to perform BMA using the gamma distribution. 

\textbf{Answer} 

\item Use the file \textit{15Fertil2.csv} to perform BMA using the exponential link function.

\textbf{Answer} 
	
\end{enumerate}