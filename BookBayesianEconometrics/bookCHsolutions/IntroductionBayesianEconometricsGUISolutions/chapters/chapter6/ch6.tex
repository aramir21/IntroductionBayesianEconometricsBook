\chapter{Solutions of chapter 6\\
Univariate models}\label{chap6}

\section{Solutions of Exercises}\label{sec61}
\begin{enumerate}[leftmargin=*]

	\item Get the posterior conditional distributions of the Gaussian linear model assuming independent priors $\pi(\beta,\sigma^2)=\pi(\beta)\times\pi(\sigma^2)$, where $\beta \sim N(\beta_0, {\bf{B}}_0)$ and $\sigma^2 \sim IG(\alpha_0/2, \delta_0/2)$.
	
\textbf{Answer}
	
The joint posterior distribution of the parameters is
\begin{align*}
	\pi(\beta, \sigma^2|\bf{y}, \bf{X}) & \propto p({\bf{y}}|\beta, \sigma^2, \bf{X}) \pi(\beta) \pi(\sigma^2) \\
	&\propto (\sigma^2)^{-\frac{N}{2}}\exp\left\{-\frac{1}{2\sigma^2}(\bf{y}-\bf{X}\beta)^{\top}(\bf{y}-\bf{X}\beta)\right\}\\
	&\times \exp\left\{-\frac{1}{2}(\beta - \beta _0)^{\top}\bf{B}_0^{-1}(\beta - \beta _0)\right\}\\
	&\times\frac{(\delta_0/2)^{(\alpha_0/2)}}{\Gamma(\alpha_0/2)}\frac{1}{(\sigma^2)^{(\alpha_0/2+1)}}\exp\left\{-\frac{\delta_0}{2\sigma^2}\right\}\\
	&\propto\exp\left\{-\frac{1}{2}(\beta - \beta _0)^{\top}\bf{B}_0^{-1}(\beta - \beta _0)\right\}\\
	&\times\frac{1}{(\sigma^2)^{(\alpha_0+N)/2+1}} \exp\left\{-\frac{\delta_0+(\bf{y}-\bf{X}\beta)^{\top}(\bf{y}-\bf{X}\beta)}{2\sigma^2}\right\}\\
	&=\exp\left\{-\frac{1}{2}(\beta - \beta _0)^{\top}\bf{B}_0^{-1}(\beta - \beta _0)\right\}\times\underbrace{\frac{1}{(\sigma^2)^{(\frac{\alpha_n}{2}+1)}} \exp\left\{-\frac{\delta_n}{2\sigma^2}\right\}}_1.
\end{align*}

Observe that (1) is the kernel of an inverse-gamma density function. Thus, $\sigma^2|\beta,{\bf{y}},{\bf{X}} \sim IG(\alpha_n/2,\delta_n/2)$, where $\alpha_n=\alpha_0+N$ and $\delta_n=\delta_0+(\bf{y}-\bf{X}\beta)^{\top}(\bf{y}-\bf{X}\beta)$.

Let's see the posterior distribution of $\beta$, 
\begin{align*}
	\pi(\beta, \sigma^2|{\bf{y}}, {\bf{X}}) & \propto p({\bf{y}}|\beta, \sigma^2, {\bf{X}}) \pi(\beta) \pi(\sigma^2) \\
	&\propto (\sigma^2)^{-\frac{N}{2}}\exp\left\{-\frac{1}{2\sigma^2}({\bf{y}}-{\bf{X}}\beta)^{\top}({\bf{y}}-{\bf{X}}\beta)\right\}\\
	&\times \exp\left\{-\frac{1}{2}(\beta - \beta _0)^{\top}{\bf{B}}_0^{-1}(\beta - \beta _0)\right\}\\
	&\times\frac{(\delta_0/2)^{(\alpha_0/2)}}{\Gamma(\alpha_0/2)}\frac{1}{(\sigma^2)^{(\alpha_0/2+1)}}\exp\left\{-\frac{\delta_0}{2\sigma^2}\right\}\\
	&=(\sigma^2)^{-\frac{N}{2}}\exp\left\{-\frac{\sigma^{-2}}{2}[{\bf{y}}^{\top}{\bf{y}}-{\bf{y}}^{\top}X\beta-\beta^{\top}{\bf{X}}^{\top}{\bf{y}}+\beta^{\top}{\bf{X}}^{\top}{\bf{X}}\beta]\right\}\\
	& \times \exp\left\{-\frac{1}{2}[\beta^{\top}{\bf{B}}_0^{-1}\beta-\beta^{\top}{\bf{B}}_0^{-1}\beta_0-\beta_0^{\top}{\bf{B}}_0^{-1}\beta+\beta_0^{\top}{\bf{B}}_0^{-1}\beta_0]\right\}\\
	&\times \frac{(\delta_0/2)^{(\alpha_0/2)}}{\Gamma(\alpha_0/2)}\frac{1}{(\sigma^2)^{(\alpha_0/2+1)}}\exp\left\{-\frac{\delta_0}{2\sigma^2}\right\}\\
	&\propto \exp\left\{-\frac{1}{2}[\beta^{\top}({\bf{B}}_0^{-1}+\sigma^{-2}{\bf{X}}^{\top}{\bf{X}})\beta-2\beta^{\top}({\bf{B}}_0^{-1}\beta_0+\sigma^{-2}{\bf{X}}^{\top}{\bf{X}}\hat{\beta})]\right\}\\
	&\times\frac{1}{(\sigma^2)^{(\alpha_0+N)/2+1}}\exp\left\{-\frac{\delta_0+{\bf{y}}^{\top}{\bf{y}}}{2\sigma^2}\right\},
\end{align*}

where $\hat{\beta}=({\bf{X}}^{\top}{\bf{X}})^{-1}{\bf{X}}^{\top}{\bf{y}}$.

Adding and subtracting $\beta_n^{\top}{\bf{B}}_n^{-1}\beta_n$ where
\begin{align*}
	{\bf{B}}_n & = ({\bf{B}}_0^{-1}+\sigma^{-2}{\bf{X}}^{\top}{\bf{X}})^{-1}\\
	\beta_n&={\bf{B}}_n({\bf{B}}_0^{-1}\beta_0+\sigma^{-2}{\bf{X}}^{\top}{\bf{X}}\hat{\beta})={\bf{B}}_n({\bf{B}}_0^{-1}\beta_0+\sigma^{-2}{\bf{X}}^{\top}{\bf{y}}),
\end{align*}

and completing the square
\begin{align*}
	\pi(\beta, \sigma^2|{\bf{y}}, {\bf{X}}) & \propto \exp\left\{-\frac{1}{2}[\beta^{\top}({\bf{B}}_0^{-1}+\sigma^{-2} {\bf{X}}^{\top}{\bf{X}})\beta-2\beta^{\top}{\bf{B}}_n^{-1}{\bf{B}}_n({\bf{B}}_0^{-1}\beta_0+\sigma^{-2}{\bf{X}}^{\top}{\bf{X}}\hat{\beta})\right.\\
	&\left.+\beta^{\top}_n{\bf{B}}_n^{-1}\beta_n-\beta_n^{\top}{\bf{B}}_n^{-1}\beta_n]\right\}\times\frac{1}{(\sigma^2)^{(\alpha_0+N)/2+1}}\exp\left\{-\frac{\delta_0+{\bf{y}}^{\top}{\bf{y}}}{2\sigma^2}\right\}\\
	&=\exp\left\{-\frac{1}{2}[\beta^{\top}{\bf{B}}_n^{-1}\beta-2\beta^{\top}{\bf{B}}_n^{-1}\beta_n+\beta_n^{\top}{\bf{B}}_n^{-1}\beta_n]\right\}\\
	& \times\frac{1}{(\sigma^2)^{(\alpha_0+N)/2+1}}\exp\left\{-\frac{\delta_0+{\bf{y}}^{\top}{\bf{y}}-\sigma^2\beta_n^{\top}{\bf{B}}_n^{-1}\beta_n}{2\sigma^2}\right\}\\
	&=\underbrace{\exp\left\{-\frac{1}{2}(\beta-\beta_n)^{\top}{\bf{B}}_n^{-1}(\beta-\beta_n)\right\}}_1\\
	&\times(\sigma^2)^{-(\frac{\alpha_n}{2}+1)}\exp\left\{-\frac{\delta^{\ast}}{2\sigma^2}\right\},
\end{align*}

where $\delta^{\ast}=\delta_0+{\bf{y}}^{\top}{\bf{y}}+\sigma^2\beta^{\top}_n{\bf{B}}_n^{-1}\beta_n$ does not dependent on $\beta$.

We can see that (1) is the kernel of a multivariate normal distribution with mean equal to $\beta_n$ and covariance matrix ${\bf{B}}_n$, that is, $\beta|\sigma^2,{\bf{y}},{\bf{X}}\sim N(\beta_n,{\bf{B}}_n)$. 

We see that the posterior distributions are from the same family as the prior distributions.

\item Show that the posterior conditional distributions of the Gaussian linear model with heteroskedasticity assuming independent priors $\pi(\beta,\sigma^2,\tau)=\pi(\beta)\times\pi(\sigma^2)\times\prod_{i=1}^N\pi(\tau_i)$, where $\beta \sim N(\beta_0, {\bf{B}}_0)$, $\sigma^2 \sim IG(\alpha_0/2, \delta_0/2)$ and $\tau_i\sim G(v/2,v/2)$ are $\beta|\sigma^2,\tau,{\bf{y}},{\bf{X}}\sim N(\beta_n,{\bf{B}}_n)$, $\sigma^2|\beta,\tau,{\bf{y}},{\bf{X}}\sim IG(\alpha_n,\delta_n)$ and $\tau_i|\beta,\sigma^2,{\bf{y}},{\bf{X}}\sim G(v_{1n},v_{2in})$, where $\tau=[\tau_1,\dots,\tau_n]^{\top}$, ${\bf{B}}_n=({\bf{B}}_0^{-1}+\sigma^{-2}{{\bf{X}}}^{\top}\Psi{{\bf{X}}})^{-1}$, $\beta_n={\bf{B}}_n({\bf{B}}_0^{-1}\beta_0+\sigma^{-2}{\bf{X}}^{\top}\Psi{\bf{y}})$, $\alpha_n=\alpha_0+N$, $\delta_n=\delta_0+({\bf{y}}-{\bf{X}}\beta)^{\top}\Psi({\bf{y}}-{\bf{X}}\beta)$, $v_{1n}=v+1$, $v_{2in}=v+\sigma^{-2}(y_i-{\bf{x}}_i^{\top}\beta)^2$, and $\Psi=\text{diagonal}\left\{\tau_i\right\}$.

\textbf{Answer}

The joint posterior distribution of the parameters is
\begin{align*}
	\pi(\beta, \sigma^2, \tau|\bf{y}, {\bf{X}}) & \propto p({\bf{y}}|\beta, \sigma^2, \tau, {\bf{X}}) \pi(\beta) \pi(\sigma^2) \prod_{i=1}^N \pi(\tau_i) \\
	&\propto \left(\prod_{i=1}^N \tau_i^{1/2}\right)(\sigma^2)^{-\frac{N}{2}}\exp\left\{-\frac{1}{2\sigma^2}({\bf{y}}-{\bf{X}}\beta)^{\top}\Psi({\bf{y}}-{\bf{X}}\beta)\right\}\\
	&\times \exp\left\{-\frac{1}{2}(\beta - \beta _0)^{\top}\bf{B}_0^{-1}(\beta - \beta _0)\right\}\\
	&\times\frac{1}{(\sigma^2)^{(\alpha_0/2+1)}}\exp\left\{-\frac{\delta_0}{2\sigma^2}\right\}\prod_{i=1}^N \tau_i^{v/2-1}\exp\left\{-v\tau_i/2\right\}.
\end{align*}

Thus, the conditional posterior distribution of $\sigma^2|\beta, \tau, \bf{y}, {\bf{X}}$ is given by

\begin{align*}
	\pi(\sigma^2|\beta, \tau, \bf{y}, {\bf{X}}) & \propto (\sigma^2)^{-\frac{N+\alpha_0}{2}}\exp\left\{-\frac{1}{2\sigma^2}({\bf{y}}-{\bf{X}}\beta)^{\top}\Psi({\bf{y}}-{\bf{X}}\beta)+\delta_0\right\}.
\end{align*}

This is the kernel of an inverse-gamma distribution with shape parameter $\alpha_n$ and rate parameter $\delta_n$.

The conditional posterior distribution of $\tau_i|\beta, \sigma^2, \bf{y}, {\bf{X}}$ is
\begin{align*}
	\pi(\tau_i|\beta, \sigma^2,\bf{y}, {\bf{X}}) & \propto
	\tau_i^{(v+1)/2-1}\exp\left\{-\frac{\tau_i}{2}[\sigma^{-2}({\bf{y}}_i-{\bf{x}}_i^{\top}\beta)^2+v]\right\}.
\end{align*}

The conditional posterior distribution of $\beta|\sigma^2, \tau, \bf{y}, {\bf{X}}$ is given by

\begin{align*}
	\pi(\beta| \sigma^2, \tau, \bf{y}, {\bf{X}}) & \propto \exp\left\{-\frac{1}{2}[\sigma^{-2}({\bf{y}}-{\bf{X}}\beta)^{\top}\Psi({\bf{y}}-{\bf{X}}\beta)+(\beta - \beta _0)^{\top}\bf{B}_0^{-1}(\beta - \beta _0)]\right\}.
\end{align*} 

Following same steps as in the previous exercise we get $\beta|\sigma^2,\tau,{\bf{y}},{\bf{X}}\sim N(\beta_n,{\bf{B}}_n)$.

\item \textbf{The market value of soccer players in Europe continues}

Use the setting of the previous exercise to perform inference using a Gibbs sampling algorithm of the the market value of soccer players in Europe setting $v=5$ and same other hyperparameters as the homoscedastic case. Is there any meaningful difference for the coefficient associated with the national team compared to the application in the homoscedastic case?

\textbf{Answer}


\begin{tcolorbox}[enhanced,width=4.67in,center upper,
	fontupper=\large\bfseries,drop shadow southwest,sharp corners]
	\textit{R. code. The value of soccer players, programming our Gibbs sampler (heteroskedastic case)}
	\begin{VF}
		\begin{lstlisting}[language=R]		
rm(list = ls())
set.seed(010101)
########################## Linear regression: Value of soccer players ##########################
Data <- read.csv("DataApplications/1ValueFootballPlayers.csv", sep = ",", header = TRUE, fileEncoding = "latin1")
attach(Data)
y <- log(Value) 
# Value: Market value in Euros (2017) of soccer players
# Regressors quantity including intercept
X <- cbind(1, Perf, Age, Age2, NatTeam, Goals, Exp, Exp2)
# Perf: Performance. Perf2: Performance squared. Age: Age; Age: Age squared. 
# NatTeam: Indicator of national team. Goals: Scored goals. Goals2: Scored goals squared
# Exp: Years of experience. Exp2: Years of experience squared. Assists: Number of assists
k <- dim(X)[2]
N <- dim(X)[1]
# Hyperparameters
d0 <- 0.001/2
a0 <- 0.001/2
b0 <- rep(0, k)
c0 <- 1000
B0 <- c0*diag(k)
B0i <- solve(B0)
v <- 5
# MCMC parameters
mcmc <- 5000
burnin <- 5000
tot <- mcmc + burnin
thin <- 1
# Posterior distributions programming the Gibbs sampling
# Auxiliary parameters
an <- a0 + N
v1n <- v + 1
# Gibbs sampling functions
PostSig2 <- function(Beta, tau){
	dn <- d0 + t(y - X%*%Beta)%*%diag(tau)%*%(y - X%*%Beta)
	sig2 <- invgamma::rinvgamma(1, shape = an/2, rate = dn/2)
	return(sig2)
}
PostBeta <- function(sig2, tau){
	Bn <- solve(B0i + sig2^(-1)*t(X)%*%diag(tau)%*%X)
	bn <- Bn%*%(B0i%*%b0 + sig2^(-1)*t(X)%*%diag(tau)%*%y)
	Beta <- MASS::mvrnorm(1, bn, Bn)
	return(Beta)
}
PostTau <- function(sig2, Beta, i){
	v2n <- v + sig2^(-1)*(y[i]-X[i,]%*%Beta)^2
	taui <- rgamma(1, v1n, v2n)
	return(taui)
}			
\end{lstlisting}
	\end{VF}
\end{tcolorbox} 

\begin{tcolorbox}[enhanced,width=4.67in,center upper,
	fontupper=\large\bfseries,drop shadow southwest,sharp corners]
	\textit{R. code. The value of soccer players, programming our Gibbs sampler (heteroskedastic case)}
	\begin{VF}
		\begin{lstlisting}[language=R]		
PostBetas <- matrix(0, mcmc+burnin, k)
PostSigma2 <- rep(0, mcmc+burnin)
Beta <- rep(0, k)
tau <- rep(1, N)
for(s in 1:tot){
	sig2 <- PostSig2(Beta = Beta, tau = tau)
	PostSigma2[s] <- sig2
	Beta <- PostBeta(sig2 = sig2, tau = tau)
	PostBetas[s,] <- Beta
	tau <- sapply(1:N, function(i){PostTau(sig2 = sig2, Beta = Beta, i)})
}
keep <- seq((burnin+1), tot, thin)
PosteriorBetas <- PostBetas[keep,]
colnames(PosteriorBetas) <- c("Intercept", "Perf", "Age", "Age2", "NatTeam", "Goals", "Exp", "Exp2")
summary(coda::mcmc(PosteriorBetas))
PosteriorSigma2 <- PostSigma2[keep]
summary(coda::mcmc(PosteriorSigma2))
summary(coda::mcmc(exp(PosteriorBetas[,5])-1))
Iterations = 1:5000
Thinning interval = 1 
Number of chains = 1 
Sample size per chain = 5000 
1. Empirical mean and standard deviation for each variable,
plus standard error of the mean:
Mean             SD       Naive SE Time-series SE 
1.246823       0.235881       0.003336       0.004010 
2. Quantiles for each variable:
2.5%   25%   50%   75% 97.5% 
0.815 1.086 1.238 1.395 1.744 					
\end{lstlisting}
	\end{VF}
\end{tcolorbox} 
We see in this application that the value of a top soccer player in Europe increases 124\% ($\exp(0.80)-1)$) on average when he has played in the national team, the credible interval at 95\% is (81\%, 174\%). These values are not very different from the application assuming homoscedasticity in the book.      
	
\end{enumerate}