\chapter{Simulation methods}\label{chap5}

\section{Solutions of Exercises}\label{sec51}
\begin{enumerate}[leftmargin=*]
\item The inverse transform method ...


\begin{tcolorbox}[enhanced,width=4.67in,center upper,
	fontupper=\large\bfseries,drop shadow southwest,sharp corners]
	\textit{R code. Example: Math test}
\begin{VF}
\begin{lstlisting}[language=R]
N <- 50 # Sample size
y_bar <- 102 # Sample mean 
s2 <- 10 # Sample variance
alpha <- N - 1
serror <- (s2/N)^0.5 
y.H0 <- c(100, 100.5, 101, 101.5, 102)
test <- (y.H0 - y_bar)/serror
pval <- 2*pt(test, alpha)
pval
0.0000459 0.0015431 0.0299338 0.2690040 1
# p-values
PO01 <- (gamma(N/2)*((N-1)*serror^2)^(-0.5)*(1+test^2/alpha)^(-N/2))/(gamma(1/2)*gamma((N-1)/2))
PO01/(1+PO01)
0.0001705 0.0050345 0.0725330 0.3210223 0.4702050
# Posterior model probability of the null hypothesis.
\end{lstlisting}
\end{VF}
\end{tcolorbox}
\end{enumerate}