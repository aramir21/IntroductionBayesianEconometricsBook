\chapter*{Preface}
Approximately 17 million people in the USA (6{\%} of the
population) and 140 million people worldwide (this number is
expected to rise to almost 300 million by the year 2025) suffer
from \textit{diabetes mellitus}. Currently, there a few dozens of
commercialised devices for detecting blood glucose levels [1].
However, most of them are invasive. The development of a
noninvasive method would considerably improve the quality of life
for diabetic patients, facilitate their compliance for glucose
monitoring, and reduce complications and mortality associated with
this disease. Noninvasive and continuous monitoring of glucose
concentration in blood and tissues is one of the most challenging
and exciting applications of optics in medicine. The major
difficulty in development and clinical application of optical
noninvasive blood glucose sensors is associated with very low
signal produced by glucose molecules. This results in low
sensitivity and specificity of glucose monitoring by optical
methods and needs a lot of efforts to overcome this difficulty.

A wide range of optical technologies have been designed in
attempts to develop robust noninvasive methods for glucose
sensing. The methods include infrared absorption, near-infrared
scattering, Raman, fluorescent, and thermal gradient
spectroscopies, as well as polarimetric, polarization
heterodyning, photonic crystal, optoacoustic, optothermal, and
optical coherence tomography (OCT) techniques [1-31].

For example, the polarimetric quantification of glucose is based
on the phenomenon of optical rotatory dispersion, whereby a chiral
molecule in an aqueous solution rotates the plane of linearly
polarized light passing through the solution. The angle of
rotation depends linearly on the concentration of the chiral
species, the pathlength through the sample, and the molecule
specific rotation. However, polarization sensitive optical
technique makes it difficult to measure \textit{in vivo} glucose
concentration in blood through the skin because of the strong
light scattering which causes light depolarization. For this
reason, the anterior chamber of the eye has been suggested as a
sight well suited for polarimetric measurements, since scattering
in the eye is generally very low compared to that in other
tissues, and a high correlation exists between the glucose in the
blood and in the aqueous humor. The high accuracy of anterior eye
chamber measurements is also due to the low concentration of
optically active aqueous proteins within the aqueous humor.

On the other hand, the concept of noninvasive blood glucose
sensing using the scattering properties of blood and tissues as an
alternative to spectral absorption and polarization methods for
monitoring of physiological glucose concentrations in diabetic
patients has been under intensive discussion for the last decade.
Many of the considered  effects, such as changing of the size,
refractive index, packing, and aggregation of RBC under glucose
variation, are important for glucose monitoring in diabetic
patients. Indeed, at physiological concentrations of glucose,
ranging from 40 to 400 mg/dl, the role of some of the effects may
be modified, and some other effects, such as glucose penetration
inside the RBC and the followed hemoglobin glycation, may be
important [30-32].

Noninvasive determination of glucose was attempted using light
scattering of skin tissue components measured by a
spatially-resolved diffuse reflectance or NIR fre\-quen\-cy-domain
reflectance techniques. Both approaches are based on change in
glucose concentration, which affects the refractive index mismatch
between the interstitial fluid and tissue fibers, and hence
reduces scattering coefficient. A glucose clamp experiment showed
that reduced scattering coefficient measured in the visible range
qualitatively tracked changes in blood glucose concentration for
the volunteer with diabetes studied.



