\chapter{Panel/Longitudinal data models}\label{chap9}

We describe how to perform inference in panel/longitudinal models using a Bayesian framework. In this context, multiple cross-sectional units are observed repeatedly over time, a structure referred to as panel data by econometricians and longitudinal data by statisticians. Specifically, we present models for continuous (normal), binary (logit), and count (Poisson) responses. Applications and exercises illustrate the potential of these models.

In panel/longitudinal data sets, we have $y_{it}$ where $i=1,2,\dots,N$ and $t=1,2,\dots,T_i$. If $T_i=T$ for all $i$, the dataset is \textit{balanced}; otherwise, it is \textit{unbalanced}. Longitudinal data typically involves by far more cross-sectional units than time periods. It assumes that cross-sectional units are independent, though serial correlation exists within each unit over time, and unobserved heterogeneity for each unit must be accounted for. We can treat this unobserved heterogeneity as random variables, assuming it is either independent or dependent on control variables. Econometricians refer to these cases as \textit{random effects} and \textit{fixed effects}, respectively. The Bayesian literature takes a different approach, modeling the panel structure hierarchically, where the unobserved heterogeneity may or may not depend on other controls.\footnote{See \cite{rendon2013fixed} for a nice comparison of Frequentist and Bayesian treatments of panel data models.}

Remember that we can run our GUI typing

\begin{tcolorbox}[enhanced,width=4.67in,center upper,
	fontupper=\large\bfseries,drop shadow southwest,sharp corners]
	\textit{R code. How to display our graphical user interface}
	\begin{VF}
		\begin{lstlisting}[language=R]
		shiny::runGitHub("besmarter/BSTApp", launch.browser = T)\end{lstlisting}
	\end{VF}
\end{tcolorbox} 

in the \textbf{R} package console or any \textbf{R} code editor, and once our GUI is deployed, select \textit{Hierarchical Longitudinal Models}.



\section{Normal model}\label{sec91}

The panel/longitudinal normal model establishes $\bm{y}_i=\bm{X}_i\bm{\beta}+\bm{W}_i\bm{b}_i+\bm{\mu}_i$ where $\bm{y}_i$ are $T_i$-dimensional vectors corresponding to units $i=1,2,\dots,N$, $\bm{X}_i$ and $\bm{W}_i$ are $T_i\times K_1$ and $T_i\times K_2$ matrices, respectively. In the statistical literature, $\bm{\beta}$ is a $K_1$-dimensional vector of \textit{fixed effects}, and $\bm{b}_i$ is a $K_2$-dimensional vector of unit-specific \textit{random effects} that allow unit-specific means, and enable to capture marginal dependence among the observations on the cross-sectional units. Observe that we have random intercepts if $\bm{W}_i=\bm{i}_{T_i}$, where $\bm{i}_{T_i}$ is a $T_i$-dimensional vector of ones. We assume normal stochastic errors, $\bm{\mu}_i\sim{N}(\bm{0},\sigma^2\bm{I}_{T_i})$.

Panel data modeling in the Bayesian approach assumes a hierarchical structure in the \textit{random effects}. Following \cite{Chib1999}, we assume $\bm{b}_i\sim{N}(\bm{0},\bm{D})$ where $\bm{D}$ allows serial correlation within each cross-sectional unit $i$ such that $\bm{D}\sim{I}{W}(d_0,d_0\bm{D}_0)$. Thus, there is an additional layer of priors as there is a prior on the hyperparameter $\bm{D}$, that is, we have a hierarchical structure. 

In addition, we have standard conjugate prior distributions for $\bm{\beta}$ and $\sigma^2$, $\bm{\beta} \sim {N}(\bm{\beta}_0,\bm{B}_0)$ and 
$\sigma^2 \sim {I}{G}(\alpha_0, 1/\delta_0)$. 

\cite{Chib1999} propose a blocking algorithm to perform inference in longitudinal hierarchical models by considering the distribution of $\bm{y}_i$ marginalized over the random effects. Given that $\bm{y}_i| \bm{\beta},\bm{b}_i,\sigma^2,\bm{X}_i,\bm{W}_i\sim N(\bm{X}_i\bm{\beta}+\bm{W}_i\bm{b}_i,\sigma^2\bm{I}_{T_i})$, we can see that    $\bm{y}_i|\bm{\beta},\bm{D},\sigma^2,\bm{X}_i,\bm{W}_i\sim{N}(\bm{X}_i\bm{\beta},\bm{V}_i)$ given that $\mathbb{E}[\bm{b}_i]=\bm{0}$ and $Var[\bm{B}_i]=D$, where $\bm{V_i}=\sigma^2\bm{I}_{T_i}+\bm{W}_i\bm{D}\bm{W}_i^{\top}$.

Thus, we can deduce the posterior distribution of $\bm{\beta}$ given $\sigma^2$ and $\bm{D}$,

\begin{align*}
	\pi(\bm{\beta}|\sigma^2, \bm{D}) & \propto 
\end{align*} 

\begin{equation*}
	\bm{\beta}|\sigma^2,\bm{D},\bm{y}, \bm{X}, \bm{W} \sim {N}(\bm{\beta}_n,\bm{B}_n), 
\end{equation*}
\begin{equation*}
	\bm{b}_i|\bm{\beta},\sigma^2,\bm{D},\bm{y}, \bm{X}, \bm{W} \sim {N}(\bm{b}_i,\bm{B}_i), 
\end{equation*}
\begin{equation*}
	\bm{D}| \bm{b} \sim {I}{W}(d_n, \bm{D}_n),
\end{equation*}
\begin{equation*}
	\sigma^2| \bm{\beta}, \bm{D}, \bm{b}, \bm{y}, \bm{X}, \bm{W} \sim {I}{G}(\alpha_n, \delta_n),
\end{equation*}
\noindent where $\bm{B}_n = (\bm{B}_0^{-1} +\sigma^{-2}\sum_{i=1}^N \bm{X}_i^{\top}\bm{V}_i^{-1}\bm{X}_i)^{-1}$, $\bm{\beta}_n= \bm{B}_n(\bm{B}_0^{-1}\bm{\beta}_0 +\sigma^{-2} \sum_{i=1}^N\bm{X}_i^{\top}\bm{V}_i^{-1}\bm{y}_i)$, $\bm{B}_i = (\bm{D}^{-1}_0 +\sigma^{-2} \bm{W}_i^{\top}\bm{W}_i)^{-1}$, $\bm{b}_i^*= \bm{B}_i(\sigma^{-2} \bm{W}_i^{\top}(\bm{y}_i-\bm{X}_i\bm{\beta}))$, $d_n = d_0 + N$ and $\bm{D}_n = d_0\bm{D}_0 + \sum_{i=1}^N \bm{b}_i\bm{b}_i^{\top}$, $\alpha_n=\alpha_0+\frac{1}{2}\sum_{i=1}^N N_i$ and $\delta_n=1/\delta_0+\frac{1}{2}\sum_{i=1}^N(\bm{y}_i-\bm{X}_i\bm{\beta}-\bm{W}_i\bm{b}_i)^{\top}(\bm{y}_i-\bm{X}_i\bm{\beta}-\bm{W}_i\bm{b}_i)$.\\ 

\textbf{Example: The relation between productivity and public investment}

We used the dataset named \textit{8PublicCap.csv} used by \cite{Ramirez2017} to analyze the relation  between public investment and gross state product in the setting of a spatial panel dataset consisting of 48 US states from 1970 to 1986.
In particular, the specification to type into the main equation box of fixed effects is
$$\log(\text{gsp})\sim\log(\text{pcap})+\log(\text{pc})+\log(\text{emp})+\text{unemp},$$

\noindent where gsp in the gross state product, pcap is public capital, and pc is private capital all in US\$, emp is employment (people), and unemp is the unemployment rate in percentage.\\

We left empty the main equation box of random effects as we assumed that the unobserved heterogeneity is not associated with any particular regressors.
This means that we control for the unobserved heterogeneity using just the constant terms.
The variable which identifies the units is id.\\

We ran this application using 10,000 MCMC iterations plus a burn-in equal to 5,000 iterations, and a thinning parameter equal to 1.
We also used the default values for the hyperparameters of the prior distributions.
It seems that all posterior draws come from stationary distributions, as suggested by the diagnostics and posterior plots.\\

The 95\% symmetric credible intervals for public capital, private capital, employment, and unemployment, are (-2.54e-02, -2.06e-02), (2.92e-01, 2.96e-01), (7.62e-01, 7.67e-01) and (-5.47e-03, -5.31e-03), respectively.
The posterior mean elasticity estimate of public capital to gsp is -0.023, that is, an increase by 1\% in public capital means a 0.023\% decrease in gross state product.
The posterior mean estimates of private capital and employment elasticities are 0.294 and 0.765, respectively.
In addition, a 1\% increase in the unemployment rate means a decrease of 0.54\% in gsp.
It seems that all these variables are statistically relevant.
In addition, the posterior mean estimates of the variance associated with the unobserved heterogeneity and stochastic errors are 1.06e-01 and 1.45e-03.
We obtained the posterior chain of the proportion of the variance associated with the unobserved heterogeneity.
The 95\% symmetric credible interval is (0.98, 0.99) for this proportion, that is, unobserved heterogeneity is very important to explain the total variability.\\



\section{Logit model}\label{sec92}

\section{Poisson model}\label{sec93}