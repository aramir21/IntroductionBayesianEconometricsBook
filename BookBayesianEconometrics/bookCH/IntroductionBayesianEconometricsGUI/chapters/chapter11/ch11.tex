\chapter{Semi-parametric and non-parametric models}\label{chap11}

Non-parametric models are characterized by making minimal assumptions about the data-generating process. Unlike parametric models, which have a finite-dimensional parameter space, non-parametric models often involve infinite-dimensional parameter spaces. A major challenge in non-parametric modeling is the \textit{curse of dimensionality}, as these models require dense data coverage, necessitating large datasets to achieve reliable estimates.

Semi-parametric methods, on the other hand, combine parametric assumptions for part of the model with non-parametric assumptions for the rest. This approach offers a balance between flexibility, tractability and applicability.

In this chapter, we introduce finite Gaussian mixture models (GM) and Dirichlet mixture processes (DMP), the latter representing an infinite mixture. Both can be used to specify an entire statistical model (non-parametric specification) or to model stochastic error distributions in a semi-parametric framework. Additionally, we present non-parametric generalized additive models (GAM), where the outcome depends linearly on smooth non-parametric functions. This method mitigates the curse of dimensionality while remaining interpretable and flexible for practical applications. 

We let other useful Bayesian non-parametric approaches like Bayesian additive random trees (BART) and Gaussian process (GP) for Chapter \ref{chap13}. 

\section{Mixture models}\label{sec11_1}
Mixture models naturally arise in situations where a sample consists of draws from different \textit{subpopulations} (\textit{clusters}) that cannot be easily distinguished based on observable characteristics. However, performing inference on specific identified subpopulations can be misleading if the assumed distribution for each cluster is misspecified.  

Even when distinct subpopulations do not exist, finite and infinite mixture models provide a useful framework for semi-parametric inference. They effectively approximate distributions with skewness, excess kurtosis, and multimodality, making them useful for modeling stochastic errors.

In addition, mixture models help capture unobserved heterogeneity. That is, as data modelers, we may observe individuals with identical sets of observable variables but entirely different response variables. These differences cannot be explained solely by sampling variability; rather, they suggest the presence of an unobserved underlying process, independent of the observable features, that accounts for this pattern.

\subsection{Finite Gaussian mixtures}\label{sec11_11}
A finite Gaussian mixture model for regression with \( H \) known components assumes that a sample 
\( \bm{y}=\left[y_1 \ y_2 \ \dots \ y_N\right]^{\top} \) consists of observations \( y_i \), 
for \( i=1,2,\dots,N \), where each \( y_i \) is generated from one of the \( H \) components, 
\( h=1,2,\dots,H \), conditional on the regressors \( \bm{x}_i \). Specifically, we assume  
\[
y_i \mid \bm{x}_i \sim N(\bm{x}_i^{\top}\bm{\beta}_h, \sigma_h^2).
\]

Thus, the sampling distribution of \( y_i \) is given by  
\[
p(y_i \mid \{\lambda_h, \bm{\beta}_h, \sigma_h^2\}_{h=1}^H, \bm{x}_i) = 
\sum_{h=1}^H \lambda_h \phi(y_i \mid \bm{x}_i^{\top}\bm{\beta}_h, \sigma_h^2),
\]

where \( \phi(y_i \mid \bm{x}_i^{\top}\bm{\beta}_h, \sigma_h^2) \) is the Gaussian density with mean 
\( \bm{x}_i^{\top}\bm{\beta}_h \) and variance \( \sigma_h^2 \), \( 0 < \lambda_h < 1 \) represents 
the proportion of the population belonging to subpopulation \( h \), and the weights satisfy 
\( \sum_{h=1}^H \lambda_h = 1 \).

Then, we allow cross-sectional units to differ according to unobserved clusters (subpopulations) that exhibit homogeneous behavior within each cluster.

To model a finite Gaussian mixture, we introduce an individual cluster indicator or latent class \( \psi_{ih} \) such that  
\[
\psi_{ih}=
\begin{cases}
	1, & \text{if the } i\text{-th unit is drawn from the } h\text{-th cluster}, \\
	0, & \text{otherwise}.
\end{cases}
\]

Thus, \( P(\psi_{ih}=1) = \lambda_h \) for all clusters \( h=1,2,\dots,H \) and units \( i=1,2,\dots,N \). Note that a high probability of individuals belonging to the same cluster suggests that these clusters capture similar sources of unobserved heterogeneity.

This setting implies that  
\[
\bm{\psi}_i = [\psi_{i1} \  \psi_{i2} \ \dots \ \psi_{iH}]^{\top} \sim \text{Categorical}(\bm{\lambda}),
\] 
  
where \( \bm{\lambda} = [\lambda_1 \  \lambda_2 \  \dots \ \lambda_H]^{\top} \) represents the event probabilities.

We know from Subsection \ref{sec421} that the Dirichlet prior distribution is conjugate to the multinomial distribution, where the categorical distribution is a special case in which the number of trials is one. Thus, we assume that  
\[
\pi(\bm{\lambda}) \sim \text{Dir}(\bm{\alpha}_0),
\]  

where \( \bm{\alpha}_0 = [\alpha_{10} \ \alpha_{20} \ \dots \ \alpha_{H0}]^{\top} \), $\alpha_{h0}=1/H$ is recommended by \cite[p.~535]{gelman2021bayesian}.  

Observe that we are using a hierarchical structure, as we specify a prior on \( \bm{\lambda} \), which serves as the hyperparameter for the cluster indicators. In addition, we can assume conjugate families for the location and scale parameters to facilitate computation, that is, $\bm \beta_h\sim N(\bm{\beta}_{h0},\bm{B}_{h0})$ and $\sigma_h^2\sim IG(\alpha_{h0}/2,\delta_{h0}/2)$.

This setting allows to obtain standard conditional posterior distributions: $$\bm{\beta}_{h}\sim N(\bm{\beta}_{hn},\bm{B}_{hn}),$$ where $\bm{B}_{hn}=(\bm{B}_{h0}^{-1}+\sigma_h^{-2}\sum_{\left\{i:  \psi_{ih}=1\right\}}\bm{x}_i\bm{x}_i^{\top})^{-1}$ and $\bm{\beta}_{hn}=\bm{B}_{hn}(\bm{B}_{h0}^{-1}\bm{\beta}_{h0}+\sigma_h^{-2}\sum_{\left\{i:  \psi_{ih}=1\right\}}\bm{x}_iy_i)$.
$$\sigma_h^2\sim IG(\alpha_{hn}/2,\delta_{hn}/2),$$

where $\alpha_{hn}=\alpha_{h0}+N_h$, $\delta_{hn}=\delta_{h0}+\sum_{\left\{i:  \psi_{ih}=1\right\}}(y_i-\bm{x}_i^{\top}\bm{\beta}_h)^2$, and $N_h$ is the number of units in cluster $h$.

$$\bm{\lambda}\sim \text{Dir}(\bm{\alpha}_n),$$   
where $\bm{\alpha}_n=[\alpha_{1n} \  \alpha_{2n} \ \dots \ \alpha_{Hn}]^{\top}$, and $\alpha_{hn}=\alpha_{h0}+N_h$.

$$\bm{\psi}_{in}\sim \text{Categorical}(\bm{\lambda}_n),$$
where $P(\psi_{ih}=1)=\frac{\lambda_{h}\phi(y_i \mid \bm{x}_i^{\top}\bm{\beta}_h,\sigma_h^2)}{\sum_{j=1}^H\lambda_{j}\phi(y_i \mid \bm{x}_i^{\top}\bm{\beta}_j,\sigma_j^2)}$.

Mixture models have the label-switching identification problem, meaning they are nonidentifiable because the distribution remains unchanged if the group labels are permuted \cite{van2011bayesian}. This poses challenges when performing inference on specific mixture components, such as in the regression analysis presented here. However, it is not a concern when using mixtures to model stochastic errors, as in semi-parametric settings where stochastic errors are represented by mixtures. In the former case, post-processing strategies can mitigate the issue, such as \textit{random permutation of latent classes} (see \cite[p.~534]{gelman2021bayesian} and Algorithm 3.5 in \cite[p.~82]{fruhwirth2006finite}).  

A semi-parametric regression imposes a specific structure in part of the model and uses flexible assumptions in another part, for instance

\begin{align*}
	y_i&=\bm{x}_i^{\top}\bm{\beta}+\mu_i\\
	p(\mu_i \mid \left\{\lambda_h,\mu_h,\sigma_h^2\right\}_{h=1}^H)&=\sum_{h=1}^H\lambda_h\phi(\mu_i\mid \mu_h,\sigma_h^2) 
\end{align*}

Thus, the distribution of the stochastic error is a finite Gaussian mixture. Note that the mean of the stochastic error is not equal to zero; consequently, the intercept in the regression should be removed, as these two parameters are not separately identifiable \cite{van2011bayesian}. Additionally, this approach allows for multiple modes and asymmetric distributions of the stochastic errors, providing greater flexibility.

We can use a Gibbs sampling algorithm in this semi-parametric specification if we assume conjugate families. The difference from the previous setting is that we have the same slope parameters; thus, $\bm{\beta} \sim N(\bm{\beta}_{0},\bm{B}_{0})$. Additionally, we must specify the prior distribution for the means of the stochastic errors, given by $\mu_h \sim N(\mu_{h0},\sigma^2_{\mu 0})$. Then, the posterior distributions are:

$$\bm{\beta}\sim N(\bm{\beta}_{n},\bm{B}_{n}),$$ where $\bm{B}_{n}=(\bm{B}_{0}^{-1}+\sum_{h=1}^H\sum_{\left\{i:  \psi_{ih}=1\right\}}\sigma_h^{-2}\bm{x}_i\bm{x}_i^{\top})^{-1}$ and $\bm{\beta}_{n}=\bm{B}_{n}(\bm{B}_{0}^{-1}\bm{\beta}_{0}+\sum_{h=1}^H\sum_{\left\{i:  \psi_{ih}=1\right\}}\sigma_h^{-2}\bm{x}_i(y_i-\mu_h))$.
$$\sigma_h^2\sim IG(\alpha_{hn}/2,\delta_{hn}/2),$$

where $\alpha_{hn}=\alpha_{h0}+N_h$, $\delta_{hn}=\delta_{h0}+\sum_{\left\{i:  \psi_{ih}=1\right\}}(y_i-\mu_h-\bm{x}_i^{\top}\bm{\beta}_h)^2$, and $N_h$ is the number of units in cluster $h$.

$$\mu_h\sim N(\mu_{hn},\sigma_{hn}^2),$$

where $\sigma_{hn}^2=\left(\frac{1}{\sigma_{h}^{2}}+\frac{N_h}{\sigma_{h}^2}\right)^{-1}$ and $\mu_{hn}=\sigma_{hn}^2\left(\frac{\mu_{h0}}{\sigma_{\mu0}^2}+\frac{\sum_{\left\{i:\psi_{ih}=1\right\}} (y_i-\bm{x}_i\bm{\beta})}{\sigma_h^2}\right)$.

$$\bm{\lambda}\sim \text{Dir}(\bm{\alpha}_n),$$   
where $\bm{\alpha}_n=[\alpha_{1n} \  \alpha_{2n} \ \dots \ \alpha_{Hn}]^{\top}$, and $\alpha_{hn}=\alpha_{h0}+N_h$.

$$\bm{\psi}_{in}\sim \text{Categorical}(\bm{\lambda}_n),$$
where $P(\psi_{ih}=1)=\frac{\lambda_{h}\phi(y_i-\bm{x}_i^{\top}\bm{\beta} \mid \mu_h,\sigma_h^2)}{\sum_{j=1}^H\lambda_{j}\phi(y_i-\bm{x}_i^{\top} \mid \mu_j,\sigma_j^2)}$.
 
A potential limitation of finite mixture models is the need to specify the number of components in advance. This issue can be mitigated by setting $H$ large enough (e.g., 20 components), assigning $\alpha_h = 1/H$, and performing an initial run of the algorithm. If we are not interested in specific components of the clusters, this procedure is sufficient. Otherwise, the posterior distribution of $H$ can be obtained by tracking the number of nonempty clusters in each iteration. In a second run, $H$ can be fixed at the mode of this posterior distribution. Another approach is to avoid fixing the number of components altogether and instead use a Dirichlet process mixture (DPM).\\

\textbf{Example: Simulation exercise}

We simulate a simple regression mixture with two components such that $\psi_{i1}\sim \text{Ber}(0.5)$, consequently, $\psi_{i2}=1-\psi_{i1}$, and assume one regressor, $x_i\sim N(0,1)$, $i=1,2,\dots,1,000$. Then, 
$$p(y_i \mid \bm{x}_i) = 
0.5 \phi(y_i \mid 2+1.5x_i,1^2)+0.5 \phi(y_i \mid -1+0.5x_i,0.8^2).$$

The following code shows how to perform inference in this model, assuming $N(0,5)$ and $N(0,2)$ priors for the intercepts and slopes, respectively. Additionally, we use a $Cauchy(0,2)$ prior truncated at 0 for the standard deviations, and a $Dirichlet(1,1)$ prior for the probabilities. We use the \textit{brms} package in \textbf{R}, which in turn uses \textit{Stan}, setting number of MCMC iterations 2,000, a burn-in (warm-up) equal to 1,000, and 4 chains. Remember that \textit{Stan} software uses Hamiltonian Monte Carlo. 

\begin{tcolorbox}[enhanced,width=4.67in,center upper,
	fontupper=\large\bfseries,drop shadow southwest,sharp corners]
	\textit{R code. Simulation exercise: Gaussian mixture with 2 components using Hamiltonian Monte Carlo}
	\begin{VF}
		\begin{lstlisting}[language=R]
####### Simulation exercise: Gaussian mixture: 2 components #############
rm(list = ls())
set.seed(010101)
library(brms)
library(ggplot2)

# Simulate data from a 2-component mixture model
n <- 1000
x <- rnorm(n)
z <- rbinom(n, 1, 0.5)  # Latent class indicator
y <- ifelse(z == 0, rnorm(n, 2 + 1.5*x, 1), rnorm(n, -1 + 0.5*x, 0.8))
data <- data.frame(y, x)

# Plot
ggplot(data, aes(x = y)) +
geom_density(fill = "blue", alpha = 0.3) +  # Density plot with fill color
labs(title = "Density Plot", x = "y", y = "Density") +
theme_minimal()

# Define priors
priors <- c(
set_prior("normal(0, 5)", class = "Intercept", dpar = "mu1"),  # First component intercept
set_prior("normal(0, 5)", class = "Intercept", dpar = "mu2"),  # Second component intercept
set_prior("normal(0, 2)", class = "b", dpar = "mu1"),  # First component slope
set_prior("normal(0, 2)", class = "b", dpar = "mu2"),  # Second component slope
set_prior("cauchy(0, 2)", class = "sigma1", lb = 0),  # First component sigma
set_prior("cauchy(0, 2)", class = "sigma2", lb = 0),  # Second component sigma
set_prior("dirichlet(1, 1)", class = "theta")  # Mixing proportions
)

# Fit a 2-component Gaussian mixture regression model
fit <- brm(
bf(y ~ 1 + x, family = mixture(gaussian, gaussian)),  # Two normal distributions
data = data,
prior = priors,
chains = 4, iter = 2000, warmup = 1000, cores = 4
)
prior_summary(fit) # Summary of priors
summary(fit) # Summary of posterior draws
plot(fit) # Plots of posterior draws
		\end{lstlisting}
	\end{VF}
\end{tcolorbox}

The following code performs inference in this simulation from scratch using Gibbs sampling. We use non-informative priors, setting $\alpha_{h0}=\delta_{h0}=0.01$, $\bm{\beta}_{h0}=\bm{0}_2$, $\bm{B}_{h0}=\bm{I}_2$, and $\bm{\alpha}_0=[1/2 \ 1/2]^{\top}$. The number of MCMC iterations is 5,000, the burn-in is 1,000, and the thinning parameter is 2. In general, the Gibbs sampler appears to yield good posterior results.

\begin{tcolorbox}[enhanced,width=4.67in,center upper,
	fontupper=\large\bfseries,drop shadow southwest,sharp corners]
	\textit{R code. Simulation exercise: Gaussian mixture with 2 components using Gibbs sampling}
	\begin{VF}
		\begin{lstlisting}[language=R]
########### Perform inference from scratch ###############
rm(list = ls()); set.seed(010101)
library(brms); library(ggplot2)
# Simulate data from a 2-component mixture model
n <- 1000
x <- rnorm(n)
z <- rbinom(n, 1, 0.5)  # Latent class indicator
y <- ifelse(z == 0, rnorm(n, 2 + 1.5*x, 1), rnorm(n, -1 + 0.5*x, 0.8))
# Hyperparameters
d0 <- 0.001; a0 <- 0.001
b0 <- rep(0, 2); B0 <- diag(2); B0i <- solve(B0)
a01 <- 1/2; a02 <- 1/2
# MCMC parameters
mcmc <- 5000; burnin <- 1000
tot <- mcmc + burnin; thin <- 2
# Gibbs sampling functions
PostSig2 <- function(Betah, Xh, yh){
	Nh <- length(yh); an <- a0 + Nh
	dn <- d0 + t(yh - Xh%*%Betah)%*%(yh - Xh%*%Betah)
	sig2 <- invgamma::rinvgamma(1, shape = an/2, rate = dn/2)
	return(sig2)
}
PostBeta <- function(sig2h, Xh, yh){
	Bn <- solve(B0i + sig2h^(-1)*t(Xh)%*%Xh)
	bn <- Bn%*%(B0i%*%b0 + sig2h^(-1)*t(Xh)%*%yh)
	Beta <- MASS::mvrnorm(1, bn, Bn)
	return(Beta)
}
PostBetas1 <- matrix(0, mcmc+burnin, 2)
PostBetas2 <- matrix(0, mcmc+burnin, 2)
PostSigma21 <- rep(0, mcmc+burnin)
PostSigma22 <- rep(0, mcmc+burnin)
PostPsi <- matrix(0, mcmc+burnin, n)
PostLambda <- rep(0, mcmc+burnin)
Id1 <- which(y<1) # 1 is from inspection of the density plot of y 
N1 <- length(Id1); Lambda1 <- N1/n
Id2 <- which(y>=1)
N2 <- length(Id2); Lambda2 <- N2/n
Reg1 <- lm(y ~ x, subset = Id1)
SumReg1 <- summary(Reg1); Beta1 <- Reg1$coefficients
sig21 <- SumReg1$sigma^2 
Reg2 <- lm(y ~ x, subset = Id2); SumReg2 <- summary(Reg2)
Beta2 <- Reg2$coefficients
sig22 <- SumReg2$sigma^2
X <- cbind(1, x); Psi <- rep(NA, n)
\end{lstlisting}
	\end{VF}
\end{tcolorbox}
 
\begin{tcolorbox}[enhanced,width=4.67in,center upper,
	fontupper=\large\bfseries,drop shadow southwest,sharp corners]
	\textit{R code. Simulation exercise: Gaussian mixture with 2 components using Gibbs sampling}
	\begin{VF}
		\begin{lstlisting}[language=R]
for(s in 1:tot){
	for(i in 1:n){
		lambdai1 <- Lambda1*dnorm(y[i], X[i,]%*%Beta1, sig21^0.5)
		lambdai2 <- Lambda2*dnorm(y[i], X[i,]%*%Beta2, sig22^0.5)
		Psi[i] <- sample(c(1,2), 1, prob = c(lambdai1, lambdai2))
	}
	PostPsi[s, ] <- Psi
	Id1 <- which(Psi == 1); Id2 <- which(Psi == 2)
	N1 <- length(Id1); N2 <- length(Id2)
	sig21 <- PostSig2(Betah = Beta1, Xh = X[Id1, ], yh = y[Id1])
	sig22 <- PostSig2(Betah = Beta2, Xh = X[Id2, ], yh = y[Id2])
	PostSigma21[s] <- sig21; PostSigma22[s] <- sig22
	Beta1 <- PostBeta(sig2h = sig21, Xh = X[Id1, ], yh = y[Id1])
	Beta2 <- PostBeta(sig2h = sig22, Xh = X[Id2, ], yh = y[Id2])
	PostBetas1[s,] <- Beta1; PostBetas2[s,] <- Beta2
	Lambda <- MCMCpack::rdirichlet(1, c(a01 + N1, a02 + N2))
	Lambda1 <- Lambda[1]; Lambda2 <- Lambda[2]
	PostLambda[s] <- Lambda1 
}
keep <- seq((burnin+1), tot, thin)
PosteriorBetas1 <- coda::mcmc(PostBetas1[keep,])
summary(PosteriorBetas1)
plot(PosteriorBetas1)
PosteriorBetas2 <- coda::mcmc(PostBetas2[keep,])
summary(PosteriorBetas2)
plot(PosteriorBetas2)
PosteriorSigma21 <- coda::mcmc(PostSigma21[keep])
summary(PosteriorSigma21)
plot(PosteriorSigma21)
PosteriorSigma22 <- coda::mcmc(PostSigma22[keep])
summary(PosteriorSigma22)
plot(PosteriorSigma22)
\end{lstlisting}
	\end{VF}
\end{tcolorbox}

Let's perform another simulation exercise in which we conduct a semi-parametric analysis where the stochastic error follows a Student's t-distribution with 3 degrees of freedom. Specifically,  
\begin{align*}
	y_i &= 1 + 0.5x_{i1} - 1.2x_{i2} + \mu_i.
\end{align*}
The variables $x_{i1}$ and $x_{i2}$ are standard normally distributed.



\subsection{Direchlet processes}\label{sec11_12}

\section{Non-parametric generalized additive models}\label{sec11_2}
%\section{Additive non-parametric structure}\label{sec11_1}
%\subsection{Partial linear model}\label{sec11_21}

\section{Summary}\label{sec11_3}

\section{Exercises}\label{sec11_4}

\begin{enumerate}
	\item Simulate a semi-parametric regression where
	\begin{align*}
		y_i &= 0.5x_{i1} - 1.2x_{i2} + \mu_i\\
		p(\mu_i) &= 
		0.3 \phi(\mu_i \mid -0.5,0.5^2)+0.7 \phi(\mu_i \mid 1,0.8^2).		
	\end{align*}
Assume that $x_{i1}$ and $x_{i2}$ distribute standard normal, and the sample size is 500. Perform inference in this model assuming that you do not the number of components, and begin with $H=5$, get the posterior distribution of the number of component, and use non-informative priors setting $\alpha_{h0}=\delta_{h0}=0.01$, $\bm{\beta}_0=\bm{0}_2$, $\bm{B}_0=\bm{I}_2$, $\mu_{h0}=0$, $\sigma^2_{\mu 0}=10$ and $\bm{\alpha}_0=[1/H \ \dots \ 1/H]^{\top}$. 
\end{enumerate}

