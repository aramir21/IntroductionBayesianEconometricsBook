\chapter{Time series models}\label{chap8}
In this chapter, we provide a brief introduction to performing inference in time series models using a Bayesian framework. There is a large literature in time series in statistics and econometrics, and it would be impossible to present a good treatment in a few pages of an introductory book. However, there excellent books in Bayesian inference in time series, see for instance, \cite{west2006bayesian,petris2009dynamic,pole2018applied}.

A time series is a sequence of observations collected in chronological order, allowing us to track how variables change over time. However, it also introduces technical challenges, as we must account for statistical features such as autocorrelation and stationarity. Since time series data is time-dependent, we adjust our notation. Specifically, we use $t$ and $T$ instead of $i$ and $N$ to explicitly indicate time.

Remember that we can run our GUI typing

\begin{tcolorbox}[enhanced,width=4.67in,center upper,
	fontupper=\large\bfseries,drop shadow southwest,sharp corners]
	\textit{R code. How to display our graphical user interface}
	\begin{VF}
		\begin{lstlisting}[language=R]
		shiny::runGitHub("besmarter/BSTApp", launch.browser = T)\end{lstlisting}
	\end{VF}
\end{tcolorbox} 

in the \textbf{R} package console or any \textbf{R} code editor, and once our GUI is deployed, select \textit{Time series Models}. However, users should see Chapter \ref{chapGUI} for details.

\section{Autoregressive models}\label{sec81}

Let's start with the linear Gaussian model with autorregresive errors,

\begin{align}
	y_t & = \bm{x}_t^{\top}\bm{\beta}+\mu_t\label{eq1}\\
	\phi(L)\mu_t & = e_t \label{eq2}, 
\end{align}
where $e_t \stackrel{iid}{\sim} N(0,\sigma^2)$, $\phi(L)=1-\phi_1L-\phi_2L^2-\dots-\phi_pL^p$ is a polynomial in the lag operator ($L$), where $Lz_t=z_{t-1}$, and in general, $L^rz_t=z_{t-r}$.

Thus, we see that stochastic error $\mu_t$ follows an \textit{autoregressive process of order $p$}, that is, $\mu_t\sim AR(p)$. It is standard practice to assume that $\mu_t$ is second-order stationary, this implies that the mean, variance and covariance of $\mu_t$ are constant. This implies that all roots of $\phi(L)$ lie outside the unit circle.

The likelihood function conditional on the first $p$ observations is
\begin{align*}
	p(y_{p+1},\dots,y_T|y_{p},\dots,y_1,\bm{\theta})&=\prod_{t=p+1}^{T}p(y_t|H_{t-1},\bm{\theta})\\
	&\propto \sigma^{-(T-p)}\exp\left\{-\frac{1}{2\sigma^2}\sum_{t=p+1}^T(y_t-\hat{y}_{t|t-1,\bm{\theta}})^2\right\},
\end{align*} 
where $H_{t-1}$ is the past history, $\bm{\theta}$ collects all parameters ($\bm{\theta}, \phi_1,\dots,\phi_p, \sigma^2$), and $\hat{y}_{t|t-1,\bm{\theta}}=(1-\phi(L))y_t+\phi(L)\bm{x}^{\top}\bm{\beta}$.

We can see that multiplying the first expression in Equation \ref{eq1} by $\phi(L)$, we can express the model as 
\begin{align}\label{eq3}
	y_t^*=\bm{x}_t^{*\top}\bm{\beta}+e_t
\end{align}
where $y_t^*=\phi(L)y_t$ and $\bm{x}_t^{*}=\phi(L)\bm{x}_t$.

Thus, collecting all observations $t=p+1,p+2,\dots,T$, we have $\bm{y}^*=\bm{X}^*\bm{\beta}+\bm{e}$, where $\bm{e}\sim N(\bm{0},\sigma^2\bm{I}_{T-p})$, $\bm{y}^*$ is a $T-p$ dimensional vector, and $\bm{X}^*$ is a $(T-p)\times K$ dimensional matrix.

Assuming that $\bm{\beta}|\sigma\sim N(\bm{\beta}_0,\sigma^2\bm{B}_0)$, $\sigma^2\sim IG(\alpha_0/2,\delta_0/2)$ and $\bm{\phi}\sim N(\phi_0,\bm{\Phi}_0)\mathbbm{1}[\bm{\phi}\in S_{\phi}]$, where $S_{\bm{\phi}}$ is the stationary region of $\bm{\phi}=[\phi_1 \ \dots \ \phi_p]^{\top}$. Then, Equation \ref{eq3} implies that $\bm{\beta}|\sigma^2,\bm{\phi},\bm{y},\bm{X}\sim N(\bm{\beta}_n, \sigma^2{\bf{B}}_n)$, where $\bm{B}_n = (\bm{B}_0^{-1} + \bm{X}^{*\top}\bm{X}^{*})^{-1}$ and $\bm{\beta}_n = \bm{B}_n(\bm{B}_0^{-1}\bm{\beta}_0 + \bm{X}^{*\top}\bm{y}^{*})$. In addition, $\sigma^2|\bm{\beta},\bm{\phi},\bm{y},\bm{X}\sim IG(\alpha_n/2,\delta_n/2)$ where $\alpha_n=\alpha_0+T-p$ and $\delta_n=\delta_0+(\bm{y}^*-\bm{X}^{*}\bm{\beta})^{\top}(\bm{y}^*-\bm{X}^{*}\bm{\beta})+(\bm{\beta}-\bm{\beta}_0)\bm{B}_0^{-1}(\bm{\beta}-\bm{\beta}_0)$. Thus, the previous conditional posterior distributions imply that we can use a Gibbs sampling algorithm to perform inference of these parameters \cite{chib1993bayes}.

We know from Equation \ref{eq1} that $\mu_t=y_t-\bm{x}_t^{\top}\bm{\beta}$, from Equation \ref{eq2} that $\mu_t=\phi_1\mu_{t-1}+\dots+\phi_p\mu_{t-p}+e_t$, $t=p+1,\dots,T$. In matrix notation $\bm{\mu}=\bm{U}\bm{\phi}+\bm{e}$, where $\bm{\mu}$ is a $T-p$ dimensional vector, $\bm{U}$ is a $(T-p)\times p$ matrix whose $t$-th row is $[\mu_{t-1} \ \dots \ \mu_{t-p}]$. Thus, the posterior distribution of $\bm{\phi}|\bm{\beta},\sigma^2,\bm{y},\bm{X}$ is $N(\bm{\phi}_n, \bm{\Phi}_n)\mathbbm{1}[\bm{\phi}\in S_{\bm{\phi}}]$, where $\bm{\Phi}_n=(\bm{\Phi}_0^{-1}+\sigma^{-2}\bm{U}^{\top}\bm{U})$ and $\bm{\phi}_n=\bm{\Phi}_n(\bm{\Phi}_0^{-1}\bm{\phi}_0+\sigma^{-2}\bm{U}^{\top}\bm{\mu})$ (see Exercise 1).

Drawing from the model restricted to stationarity is straightforward: we simply sample from the multivariate normal distribution and discard draws that do not meet the stationarity condition. The proportion of draws that satisfy this restriction represents the conditional probability that the process is stationary.

Observe that the previous setting encompasses the particular relevant case  $y_t\sim AR(p)$, it is just omitting the covariates such that $y_t=\mu_t$. \cite{chib1994bayes} extend the Bayesian inference of linear regression with $AR(p)$ errors to $ARMA(p,q)$ errors.\\

\textbf{Example: Residential water demand}

Let's use the data set provided by \cite{hassan2018effects} to estimate the demand of residential water in Medellín (Colombia). This a set ....

A nice feature of Bayesian inference is its modular nature. This means we can break down a complex inferential problem into smaller, more manageable parts, this is, a ``divide and conquer" approach. This is possible due to the structure of conditional posterior distributions.  
   
\section{State-space representation}\label{sec82}
A \textit{state-space model} is composed by of an \textit{unobservable state vector}  $\bm{\beta}_t \in \mathbb{R}^K$, and an \textit{observed} measure $\bm{Y}_t \in \mathbb{R}^M$, $t=1,2,\dots$ such that (i) $\bm{\beta}_t$ is a \textit{Markov process}, this is, $\pi(\bm{\beta}_t|\bm{\beta}_{1:t-1})=\pi(\bm{\beta}_t|\bm{\beta}_{t-1})$, all the information regarding $\bm{\beta}_t$ based on all its history up to $t-1$ is carried by $\bm{\beta}_{t-1}$, and (ii) $\bm{Y}_t$ is independent of $\bm{Y}_s$ conditional on $\bm{\beta}_t$, $t\neq s$ \cite[Chap.~2]{petris2009dynamic}.

These assumptions imply that $\pi(\bm{\beta}_{0:t},\bm{y}_{1:t})=\pi(\bm{\beta}_0)\prod_{s=1}^{t}\pi(\bm{\beta}_s|\bm{\beta}_{s-1})\pi(\bm{y}_s|\bm{\beta}_s)$.\footnote{A \textit{state-space model} where the states are random variables taking discrete values is called \textit{hidden Markov model}.}

There are three key aspects of \textit{state-space models}: \textit{filtering}, \textit{smoothing}, and \textit{forecasting}. In \textit{filtering}, we aim to estimate the current state given observations up to time $t$, specifically obtaining the density $\pi(\bm{\beta}_{s}|\bm{y}_{1:t})$ for $s = t$. In \textit{smoothing}, we conduct a retrospective analysis of the system, obtaining $\pi(\bm{\beta}_{s}|\bm{y}_{1:t})$ for $s < t$. In \textit{forecasting}, we forecast future observations by first obtaining $\pi(\bm{\beta}_{s}|\bm{y}_{1:t})$ as an intermediate step to compute $\pi(\bm{y}_{s}|\bm{y}_{1:t})$ for $s > t$. A valuable feature of these methods is that all these densities can be calculated recursively. \cite{petris2009dynamic} show the recursive equations in Propositions 2.1 (filtering), 2.3 (smoothing) and 2.5 (forecasting).

An important class of \textit{state-space models} is the \textit{Gaussian linear state-space model}, also know as, \textit{dynamic linear model}:
\begin{align*}
	\bm{Y}_t&=\bm{X}_t\bm{\beta}_t+\bm{\mu}_t& \text{(Observation equations)}\\
	\bm{\beta}_t&=\bm{G}_t\bm{\beta}_{t-1}+\bm{w}_t & \text{(States equations)},
\end{align*}
where $\bm{\beta}_0\sim N(\bm{b}_0,\bm{B}_0)$, $\bm{\mu}_t\sim N(\bm{0}, \bm{\Sigma}_t)$, $\bm{w}_t\sim N(\bm{0}, \bm{\Omega}_t)$, $\bm{\beta}_0$, $\bm{\mu}_t$ and $\bm{w}_t$ are independent, $\bm{X}_t$ and $\bm{G}_t$ are $M\times K$ and $K\times K$ known matrices. Observe that this assumptions implies that $\bm{Y}_t|\bm{\beta}_t\sim N(\bm{X}_t\bm{\beta}_t,\bm{\Sigma}_t)$, and $\bm{\beta}_t|\bm{\beta}_{t-1}\sim N(\bm{G}_t\bm{\beta}_{t-1},\bm{\Omega}_t)$.

\cite{petris2009dynamic} show in Propositions 2.2, 2.4 and 2.6 the recursive equations to perform filtering, smoothing and forecasting in \textit{dynamic linear models} (\textit{Gaussian linear state-space models}), respectively. Preposition 2.2 shows the well-known Kalman filter, which allows to compute the predictive and filtering distributions in a forward way starting from $\bm{\beta}_0\sim N(\bm{b}_0,\bm{B}_0)$, then $\bm{\beta}_1|\bm{y}_1$, and proceeding recursively as new data arrives. Proposition 2.4 in \cite{petris2009dynamic} shows the Kalman smoother, which allows calculating in a backward way $\pi(\bm{\beta}_t|\bm{y}_{1:T})$ starting from $t=T-1$.

Let's see the \textit{state-space} representation of the $ARMA(p,q)$ model $y_t=\sum_{s=1}^{p}\phi_jy_{t-s}+\sum_{s=1}^{q}\theta_s e_{t-s}+e_t$. Setting $r=\max \left\{p,q+1\right\}$, $\phi_s=0$ for $s>p$ and $\theta_s=0$ for $s>q$, and defining the matrices $\bm{X}=[1 \ 0 \ \dots \ 0]$, $\bm{R}=[1 \ \psi_1 \ \dots \ \psi_{r-1}]^{\top}$ 
\begin{align*}
	\bm{G}=\begin{bmatrix}
		\phi_1 & 1 & 0 & \dots & 0\\
		\phi_2 & 0 & 1 & \dots & 0\\
		\vdots & \vdots & \ddots &  &\\
		 \phi_{r-1} & 0 & 0 & \dots & 1\\
		 \phi_r & 0 & 0 & \dots & 0\\
	\end{bmatrix} = \begin{bmatrix}
	\phi_1 & \vdots &  &  & \\
	\phi_2 & \vdots &  & \bm{I}_{r-1}  & \\
	\vdots & \vdots &  &  &\\
	\dots & \dots & \dots & \dots & \dots\\
	\phi_r & 0 & 0 & \dots & 0\\
\end{bmatrix}.
\end{align*} 
and give the \textit{state} vector $\bm{\beta}_t=[\beta_{1,t} \ \beta_{2,t} \ \dots \ \beta_{r,t}]^{\top}$, the $ARMA$ model has the following representation:
\begin{align*}
	y_t&=\bm{F}\bm{\beta}_t\\
	\bm{\beta}_t &= \bm{G}\bm{\beta}_{t-1}+\bm{R}e_{t}.
\end{align*}

This is a model where $\bm{\Sigma}_t=0$, and $\bm{\Omega}_t=\sigma^2\bm{R}\bm{R}^{\top}$ (see \cite{petris2009dynamic,chib1994bayes}).

A nice advantage of the \textit{state-space} representation of the $ARMA$ model is that the evaluation of the likelihood can be performed efficiently using the recursive laws. Extensions to $ARIMA(p,d,q)$ models can be seen in \cite[Chap.~3]{petris2009dynamic}.


\section{Time varying parameters models}\label{sec83}        
     

\section{Stochastic volatility models}\label{sec84}  



\section{Vector Autoregressive models}\label{sec85}

\section{Summary}\label{sec86}

\section{Exercises}\label{sec87}

\begin{enumerate}
	\item Show that the posterior distribution of $\bm{\phi}|\bm{\beta},\sigma^2,\bm{y},\bm{X}$ in the model $y_t=\bm{x}_t^{\top}\bm{\beta}+\mu_t$ where $\phi(L)\mu_t=e_t$ and $e_t\stackrel{iid}{\sim}N(0,\sigma^2)$ is $N(\bm{\phi}_n, \bm{\Phi}_n)\mathbbm{1}[\bm{\phi}\in S_{\bm{\phi}}]$, where $\bm{\Phi}_n=(\bm{\Phi}_0^{-1}+\sigma^{-2}\bm{U}^{\top}\bm{U})$, $\bm{\phi}_n=\bm{\Phi}_n(\bm{\Phi}_0^{-1}\bm{\phi}_0+\sigma^{-2}\bm{U}^{\top}\bm{\mu})$, and $S_{\phi}$ is the stationary region of $\bm{\phi}$.
	
	
\end{enumerate}
