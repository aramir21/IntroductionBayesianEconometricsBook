\chapter{Causal inference}\label{chap12}

Two critical aspects in the identification of causal effects are: (i) the presence of a strong source of \textit{exogenous variation} that influences the \textit{endogenous regressors}, which are often the primary variables of interest to researchers, as they may directly affect the outcome or response variables and can be influenced by policy decisions, for example, identifying the causal effects of social programs on income or education, or evaluating strategic interventions in industry, such as estimating the price elasticity of demand for a specific product; and (ii) the effective \textit{control of other relevant exogenous covariates}, such as pre-treatment characteristics or external factors in a demand model.

In this context, the use of \textit{non-parametric models} (see Chapters~\ref{chap12} and~\ref{chap13}) is valuable due to their flexibility and weaker structural assumptions. Therefore, non-parametric and machine learning approaches serve as \textit{powerful tools} that can be combined with strong exogenous variation to robustly identify causal effects \cite{chernozhukov2018double,chernozhukov2024applied}.


Read this reference \cite{iacovone2023bayesian} before begin working in this chapter! Also read \cite{imbens1997bayesian}.
\section{Instrumental variables}\label{sec12_1}
\subsection{Semi-parametric IV model}\label{sec12_11}
Use function \textit{rivDP} from package \textit{bayesm} in \textbf{R}.

\section{Sample selection}\label{sec12_2}
\cite{greenberg2012introduction}

\section{Regression discontinuity design}\label{sec12_3}
\cite{chib2016bayesian,chib2023nonparametric,kowalska2024bayesian}

\section{Regression kink design}\label{sec12_4}
\cite{chan2025minimum}

\section{Synthetic control}\label{sec12_5}
\cite{amjad2018robust,kim2020bayesian}

\section{Difference in difference estimation}\label{sec12_6}
\cite{normington2019bayesian,normington2022bayesian,breunig2024semiparametric}

\section{Event Analysis}\label{sec12_7}

\section{Bayesian exponential tilted empirical likelihood}\label{sec12_8}
Bayesian parametric approaches are often criticized on the basis that they require arbitrary distribution assumptions which often are not examined. Partial information approaches are based only on certain moment assumptions without making specific distributional assumptions. However, there are no free lunch, as these methods imply efficiency losses.

The point of departure of Bayesian exponential tilted empirical likelihood (BETEL) are moment conditions that are used to build the likelihood function.

\section{A general framework for updating belief distributions}\label{sec12_9}
Introduce \cite{bissiri2016general} as a generalization of \cite{chernozhukov2003mcmc}.

\section{Bayesian model averaging}\label{sec12_10}
We can use BMA as a sensible way to perform robustness analysis regarding model specification (regressors uncertainty) in performing inference of treatment effects.

%\section{Double-Debiased machine learning causal effects}\label{sec12_11a}
