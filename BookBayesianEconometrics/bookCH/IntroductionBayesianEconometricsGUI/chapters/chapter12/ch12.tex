\chapter{Causal inference}\label{chap12}

\section{Instrumental variables}\label{sec12_1}
\subsection{Semi-parametric IV model}\label{sec12_11}
Use function \textit{rivDP} from package \textit{bayesm} in \textbf{R}.

\section{Regression discontinuity design}\label{sec12_2}

\section{Regression kink design}\label{sec12_3}

\section{Synthetic control}\label{sec12_4}

\section{Difference in difference estimation}\label{sec12_5}

\section{Event Analysis}\label{sec12_6}

\section{Bayesian exponential tilted empirical likelihood}\label{sec12_7}
Bayesian parametric approaches are often criticized on the basis that they require arbitrary distribution assumptions which often are not examined. Partial information approaches are based only on certain moment assumptions without making specific distributional assumptions. However, there are no free lunch, as these methods imply efficiency losses.

The point of departure of Bayesian exponential tilted empirical likelihood (BETEL) are moment conditions that are used to build the likelihood function.

\section{Bayesian model averaging}\label{sec12_8}
We can use BMA as a sensible way to perform robustness analysis regarding model specification (regressors uncertainty) in performing inference of treatment effects.

\section{Double-Debiased machine learning causal effects}\label{sec12_9}
