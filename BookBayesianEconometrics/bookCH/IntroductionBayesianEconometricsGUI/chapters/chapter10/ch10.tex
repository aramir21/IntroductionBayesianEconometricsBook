\chapter{Bayesian model average}\label{chap10}

We describe in this chapter how to introduce model uncertainty and average over different models in a probabilistic consistent way. We describe .... 

Remember that we can run our GUI typing

\begin{tcolorbox}[enhanced,width=4.67in,center upper,
	fontupper=\large\bfseries,drop shadow southwest,sharp corners]
	\textit{R code. How to display our graphical user interface}
	\begin{VF}
		\begin{lstlisting}[language=R]
	shiny::runGitHub("besmarter/BSTApp", launch.browser = T)\end{lstlisting}
	\end{VF}
\end{tcolorbox} 

in the \textbf{R} package console or any \textbf{R} code editor, and once our GUI is deployed, select \textit{Bayesian Model Averaging}. However, users should see Chapter \ref{chapGUI} for other options and details.

\section{Foundation}\label{sec10_1}
Remember from Chapter \ref{chap1} that Bayesian model averaging (BMA) is an approach which takes into account model uncertainty. In particular, we consider uncertainty in the regressors (variable selection) in a regression framework where there are $K$ possible explanatory variables.\footnote{Take into account that $K$ can increase when interaction terms and/or polynomial terms of the original control variables are included.} This implies $2^K$ potential models indexed by parameters $\bm{\theta}_m$, $m=1,2,\dots,2^K$.

Following \cite{Simmons2010}, the posterior model probability is
\begin{equation*}
	\pi(\mathcal{M}_j |\bm{y})=\frac{p(\bm{y} | \mathcal{M}_j)\pi(\mathcal{M}_j)}{\sum_{m=1}^{2^K}p(\bm{y} | \mathcal{M}_m)\pi(\mathcal{M}_m)},
\end{equation*}
where $\pi(\mathcal{M}_j)$ is the prior model probability,\footnote{We attach equal prior probabilities to each model in our GUI. However, this choice gives more prior probability to the set of models of medium size (think about the $k$-th row of Pascal's triangle). An interesting alternative is to use the Beta-Binomial prior proposed by \cite{ley2009effect}.} 
\begin{equation*}
	p(\bm{y} | \mathcal{M}_j)=\int_{\bm{\Theta}_j} p(\bm{y}| \bm{\theta}_j,\mathcal{M}_j)\pi(\bm{\theta}_j | \mathcal{M}_j) d\bm{\theta}_{j}
\end{equation*}
is the marginal likelihood, and $\pi(\bm{\theta}_j | \mathcal{M}_j)$ is the prior distribution of $\bm{\theta}_j$ conditional on model $\mathcal{M}_j$.

Following \cite{Raftery93}, the posterior distribution of $\bm{\theta}$ is 
\begin{equation*}
	\pi(\bm{\theta}|\bm{y})= \sum_{m=1}^{2^K}\pi(\bm{\theta}_m|\bm{y},\mathcal{M}_m) \pi(\mathcal{M}_m|\bm{y})
\end{equation*}
where $\pi(\bm{\theta}_m|\bm{y},\mathcal{M}_m)$ is the posterior distribution of $\bm{\theta}$ under model $\mathcal{M}_m$, $\mathbb{E}[\bm{\theta}|\bm{y}]=\sum_{m=1}^{2^K}\hat{\bm{\theta}}_m \pi(\mathcal{M}_m|\bm{y})$, $Var({\theta}_{km}|\bm{y})= \sum_{m=1}^{2^K}\pi(\mathcal{M}_m|\bm{y}) \widehat{Var} ({\theta}_{km}|\bm{y},\mathcal{M}_m)+\sum_{m=1}^{2^K} \pi(\mathcal{M}_m|\bm{y}) (\hat{{\theta}}_{km}-\mathbb{E}[{\theta}_{km}|\bm{y}])^2$, $\hat{\bm{\theta}}_m$ is the posterior mean and $\widehat{Var}({\theta}_{km}|\bm{y},\mathcal{M}_m)$ is the posterior variance of the element $k$-th of $\bm{\theta}$ under model $\mathcal{M}_m$.

The posterior variance highlights how the BMA method takes into account
model uncertainty. The first term is the weighted variance of each model, averaged over all potential models, and the second term indicates how stable the estimates are across models. The more the estimates differ between models, the greater is the posterior variance.

The posterior predictive distribution is
\begin{equation*}
	\pi(\bm{Y}_0|\bm{y})= \sum_{m=1}^{2^K}p_m(\bm{Y}_0|\bm{y},\mathcal{M}_m) \pi(M_m|\bm{y})
\end{equation*}

where $p_m(\bm{Y}_0|\bm{y},\mathcal{M}_m)=\int_{\bm{\Theta}_m} p(\bm{Y}_0|\bm{y},\bm{\theta}_m,\mathcal{M}_m)\pi(\bm{\theta}_m |\bm{y}, \mathcal{M}_m) d\bm{\theta}_{m}$ is the posterior predictive distribution under model $\mathcal{M}_m$. 

Another important statistic in BMA is the posterior inclusion probability associated with variable $\bm{x}_k$, $k=1,2,\dots,K$, which is

\begin{equation*}
	PIP(\bm{x}_k)=\sum_{m=1}^{2^K}\pi(\mathcal{M}_m|\bm{y})\times \mathbbm{1}_{k,m},
\end{equation*}
where
$\mathbbm{1}_{k,m}= \left\{ \begin{array}{lcc}
	1&   if  & \bm{x}_{k}\in \mathcal{M}_m \\
	\\ 0 &  if & \bm{x}_{k}\not \in \mathcal{M}_m
\end{array}
\right\}.$\\

\cite{Kass1995} suggest that posterior inclusion probabilities (PIP) less than 0.5 are evidence against the regressor, $0.5\leq PIP<0.75$ is weak evidence, $0.75\leq PIP<0.95$ is positive evidence, $0.95\leq PIP<0.99$ is strong evidence, and $PIP\geq 0.99$ is very strong evidence.

There are two main computational issues in implementing BMA based on variable selection. First, the number of models in the model space is $2^K$, which sometimes can be enormous. For instance, three regressors imply just eight models, see Table \ref{tab:chap10}, but 40 regressors implies approximately  1.1e+12 models. Take into account that models always include the intercept, and all regressors should be standardized to avoid scale issues.\footnote{Scaling variables is always an important step in variable selection.} The second computational issue is calculating the marginal likelihood $p(\bm{y} | \mathcal{M}_j)=\int_{\bm{\Theta}_j} p(\bm{y}| \bm{\theta}_j,\mathcal{M}_j)\pi(\bm{\theta}_j | \mathcal{M}_j) d\bm{\theta}_{j}$, which most of the time does not have an analytic solution. 

\begin{table}[!ht]
	\tabletitle{Space of models: Three regressors.}\label{tab:chap10}
%	\begin{threeparttable}
		\resizebox{1\textwidth}{!}{\begin{minipage}{\textwidth}
				\begin{tabular}{ccccccccc}
					 Regressor & \multicolumn{8}{c}{Inclusion}\\
					\hline
					$x_1$ & 1 & 1 & 1 & 1 & 0 & 0 & 0 & 0\\
					$x_2$ & 1 & 1 & 0 & 0 & 1 & 1 & 0 & 0\\
					$x_3$ & 1 & 0 & 1 & 0 & 1 & 0 & 1 & 0\\ 			
				\end{tabular}
				\begin{tablenotes}[para,flushleft]
					\footnotesize \textit{Notes}: ``1" indicates inclusion of the regressor, and ``0" indicates no inclusion. The space of models is composed by 8 models. The model always includes intercept.\\
				\end{tablenotes}
		\end{minipage}}
%	\end{threeparttable}
\end{table}
The first computational issue is basically a problem of ranking models. This can be tackled using different approaches, such as Occam's window criterion \cite{Madigan1994,Raftery1997}, reversible jump Markov chain Monte Carlo computation \cite{Green1995}, Markov chain Monte Carlo model composition \cite{madigan95}, and multiple testing using intrinsic priors \cite{Casella2006} or nonlocal prior densities \cite{Johnson2012}. We focus on Occam's window and Markov chain Monte Carlo model composition in our GUI.\footnote{Variable selection (model selection or regularization) is a topic related to model uncertainty. Approaches such as stochastic search variable selection (spike and slab) \cite{George1993,George1997} and Bayesian Lasso \cite{Park2008} are good examples of how to tackle this issue. See Chapter \ref{chap13}.}

In Occam's window, a model is discarded if its predictive performance is much worse than that of the best model  \cite{Madigan1994,Raftery1997}.
Thus, models not belonging to $\mathcal{M}'=\left\{\mathcal{M}_j:\frac{\max_m {\pi(\mathcal{M}_m|\bm{y})}}{\pi(\mathcal{M}_j|\bm{y})}\leq c\right\}$ should be discarded, where $c$ is chosen by the user (\cite{Madigan1994} propose $c=20$).
In addition, complicated models than are less supported by the data than simpler models are also discarded, that is, $\mathcal{M}''=\left\{\mathcal{M}_j:\exists \mathcal{M}_m\in\mathcal{M}',\mathcal{M}_m\subset \mathcal{M}_j,\frac{\pi(\mathcal{M}_m|\bm{y})}{\pi(\mathcal{M}_j|\bm{y})}>1\right\}$. Then, the set of models used in BMA is $\mathcal{M}^*=\mathcal{M}'\cap \mathcal{M}''^c\in\mathcal{M}$. \cite{Raftery1997} find that the number of models in $\mathcal{M}^*$ is normally less than 25.

However, the previous theoretical framework requires finding the model with the maximum a posteriori model probability ($\max_m {\pi(\mathcal{M}_m|\bm{y})}$), which implies calculating all possible models in $\mathcal{M}$. This is computationally burdensome. Hence, a heuristic approach is proposed by \cite{Raftery2012} based on ideas of \cite{Madigan1994}. The search strategy is based on a series of nested comparisons of ratios of posterior model probabilities. Let $\mathcal{M}_0$ be a model with one regressor less than model $\mathcal{M}_1$, then:
\begin{itemize}
	\item If $\log(\pi(\mathcal{M}_0|\bm{y})/\pi(\mathcal{M}_1|\bm{y}))>\log(O_R)$, then $\mathcal{M}_1$ is rejected and $\mathcal{M}_0$ is considered.
	\item If $\log(\pi(\mathcal{M}_0|\bm{y})/\pi(\mathcal{M}_1|\bm{y}))\leq -\log(O_L)$, then $\mathcal{M}_0$ is rejected, and $\mathcal{M}_1$ is considered.
	 \item If $\log(O_L)<\log(\pi(\mathcal{M}_0|\bm{y})/\pi(\mathcal{M}_1|\bm{y}))\leq \log(O_R$), $\mathcal{M}_0$ and $\mathcal{M}_1$ are considered.
\end{itemize} 
Here $O_R$ is a number specifying the maximum ratio for excluding models in Occam's window, and $O_L=1/O_R^{2}$ is defined by default in \cite{Raftery2012}. The search strategy can be ``up,'' adding one regressor, or ``down,'' dropping one regressor (see \cite{Madigan1994} for details about the down and up algorithms). The leaps and bounds algorithm \cite{Furnival1974} is implemented to improve the computational efficiency of this search strategy \cite{Raftery2012}. Once the set of potentially acceptable models is defined, we discard all the models that are not in $\mathcal{M}'$, and the models that are in $\mathcal{M}''$ where 1 is replaced by $\exp\left\{O_R\right\}$ due to the leaps and bounds algorithm giving an approximation to BIC, so as to ensure that no good models are discarded.

The second approach that we consider in our GUI to tackle the model space size issue is Markov chain Monte Carlo model composition (MC3) \cite{madigan1995bayesian1}.
In particular, given the space of models $\mathcal{M}_m$, we simulate a chain of $\mathcal{M}_s$ models, $s = 1, 2, ..., S<<2^K$, where the algorithm randomly extracts a candidate model $\mathcal{M}_c$ from a neighborhood of models ($nbd(\mathcal{M}_m)$) that consists of the actual model itself and the set of models with either one variable more or one variable less \cite{Raftery1997}. Therefore, there is a transition kernel in the space of models $q(\mathcal{M}_m\rightarrow \mathcal{M}_c)$, such that $q(\mathcal{M}_m\rightarrow \mathcal{M}_{c})=0 \ \forall \mathcal{M}_{c}\notin nbd(\mathcal{M}_m)$ and $q(\mathcal{M}_m\rightarrow \mathcal{M}_{c})=\frac{1}{|nbd(\mathcal{M}_m)|} \ \forall \mathcal{M}_m\in nbd(\mathcal{M}_m)$, $|nbd(\mathcal{M}_m)|$ being the number of neighbors of $\mathcal{M}_m$. This candidate model is accepted with probability

\begin{equation*}
	\alpha (\mathcal{M}_{s-1},\mathcal{M}_{c})=\min \bigg \{ \frac{|nbd(\mathcal{M}_m)|p(\bm{y} | \mathcal{M}_c)\pi(\mathcal{M}_c)}{|nbd(\mathcal{M}^{c})|p(\bm{y}| \mathcal{M}_{(s-1)})\pi(\mathcal{M}_{(s-1)})},1 \bigg \}.
\end{equation*}

Observe that by construction $|nbd(\mathcal{M}_m)|=|nbd(\mathcal{M}_c)|=k$, except in extreme cases where a model has only one regressor or has all regressors.

The Bayesian information criterion is a possible solution for the second computational issue in BMA, that is, calculating the marginal likelihood when there is no an analytic solution. Defining $h(\bm{\theta}|\mathcal{M}_j)=-\frac{\log(p(\bm{y}| \bm{\theta}_j,\mathcal{M}_j)\pi(\bm{\theta}_j | \mathcal{M}_j))}{N}$, then $p(\bm{y} | \mathcal{M}_j)=\int_{\bm{\Theta}_j} \exp\left\{-N h(\bm{\theta}|\mathcal{M}_j)\right\}  d\bm{\theta}_{j}$. If $N$ is sufficiently large (technically $N\to \infty$), we can make the following assumptions \cite{Hoeting1999}:

\begin{itemize}
	\item We can use the Laplace method for approximating integrals \cite{Tierney1986}.
	\item The posterior mode is reached at the same point as the maximum likelihood estimator (MLE), denoted by $\hat{\bm{\theta}}_{MLE}$.
\end{itemize}

We get the following results under these assumptions:
\begin{align*}
	p(\bm{y} | \mathcal{M}_j)\approx&\left( \frac{2\pi}{N}\right)^{K_j/2}|\bm{\Sigma}|^{-1/2} \exp\left\{-N h(\bm{\hat{\theta}}_j^{MLE}|\mathcal{M}_j)\right\}, \ N\rightarrow\infty,
\end{align*}
where $\bm{\Sigma}$ is the Hessian matrix of $h(\bm{\hat{\theta}}_j^{MLE}|\mathcal{M}_j)$, and $K_j=dim\left\{\bm{\theta}_j\right\}$.

This implies
\begin{align*}
	\log\left(p(\bm{y} | \mathcal{M}_j)\right)\approx& \frac{K_j}{2}\log(2\pi)- \frac{K_j}{2}\log(N) -\frac{1}{2}\log(|\bm{\Sigma}|) + \log(p(\bm{y}| \bm{\hat{\theta}}_j^{MLE},\mathcal{M}_j))\\
	&+\log(\pi(\bm{\hat{\theta}}_j^{MLE} | \mathcal{M}_j)), \ N\rightarrow\infty.
\end{align*}

Since $\frac{K_j}{2}\log(2\pi)$ and $\log(\pi(\bm{\hat{\theta}}_j^{MLE} | \mathcal{M}_j))$ are constants as functions of $\bm{y}$, and $|\bm{\Sigma}|$ is bounded by a finite constant, we have
\begin{align*}
	log\left(p(\bm{y} | \mathcal{M}_j)\right)\approx& -\frac{K_j}{2}\log(N)+\log(p(\bm{y}| \bm{\hat{\theta}}_j^{MLE},\mathcal{M}_j))= -\frac{BIC}{2}, \ N \rightarrow \infty.
\end{align*}

The marginal likelihood thus asymptotically converges to a linear transformation of the Bayesian Information Criterion (BIC), significantly simplifying its calculation. In addition, the BIC is consistent, that is, the probability of uncovering the population statistical model converges to one as the sample size converges to infinity given a $\mathcal{M}$-closed view \cite[Chap.~6]{Bernardo1994}, that is, one of the models in consideration is the population statistical model (data generating process) \cite{schwarz1978estimating, burnham2004multimodel}. In case that there is an $\mathcal{M}$-completed view of nature, that is, there is a true data generating process, but the space of models that we are comparing does not include it, the BIC asymptotically selects the model that minimizes the Kullback-Leiber (KL) divergence to the true (population) model \cite[Chap. ~4]{claeskens2008model}. 


\section{The Gaussian linear model}\label{sec10_2}

The Gaussian linear model specifies $\bm{y}=\alpha\bm{i}_N+\bm{X}_m\bm{\beta}_m+\bm{\mu}_m$ such that $\bm{\mu}_m\sim{N}(\bm{0},\sigma^2\bm{I}_n)$, and $\bm{X}_m$ does not have the column of ones. Following \cite{koop2003bayesian}, the conjugate prior for the location parameters is $\bm{\beta}_m|\sigma^2 \sim {N}(\bm{\beta}_{m0}, \sigma^2 \bm{B}_{m0})$, and the priors for $\sigma^2$ and $\alpha$ can be improper, as these parameters are common to all models $\mathcal{M}_m$. Particularly, $\pi(\sigma^2)\propto 1/\sigma^2$ (Jeffreys' prior for the linear Gaussian model, see \cite{prior1991bayesian}) and $\pi(\alpha)\propto 1$.

The selection of the hyperparameters of $\bm{\beta}_m$ is more critical, as these parameters are not common to all models. A very common prior for the location parameters in the BMA literature is the Zellner's prior \cite{zellner1986assessing}, where $\bm{\beta}_{m0}=\bm{0}_m$ and $\bm{B}_{m0}=(g_m\bm{X}_m^{\top}\bm{X}_m)^{-1}$. Observe that this covariance matrix is similar to the covariance matrix of ordinary least squares estimator of the location parameters. This suggests that there is compatibility between the prior information and the sample information, and the only parameter to elicit is $g_m\geq 0$, which facilitates the elicitation process, as eliciting covariance matrices is a very hard endeavor.

Following same steps as in Section \ref{sec43}, the posterior conditional distribution of $\bm{\beta}_m$ has covariance matrix $\sigma^2\bm{B}_{mn}$, where $\bm{B}_{mn}=((1+g_m)\bm{X}_m^{\top}\bm{X}_m)^{-1}$ (Exercise 1), which means that $g_m=0$ implies a non-informative prior, whereas $g_m=1$ implies that prior and data information have same weights. We follow \cite{fernandez2001benchmark}, who recommend
\begin{align*}
	g_m & =
	\begin{Bmatrix}
		1/K^2, & N \leq K^2\\
		1/N, & N>K^2 
	\end{Bmatrix}.
\end{align*}  
 
Given the likelihood function, 
\begin{equation*}
	p(\bm{\beta}_m, \sigma^2|\bm{y}, \bm{X}_m) = (2\pi\sigma^2)^{-\frac{N}{2}} \exp \left\{-\frac{1}{2\sigma^2} (\bm{y} - \alpha\bm{i}_N - \bm{X}_m\bm{\beta}_m)^{\top}(\bm{y} - \alpha\bm{i}_N - \bm{X}_m\bm{\beta}_m) \right\},
\end{equation*}
the marginal likelihood associated with model $\mathcal{M}_m$ is proportional to (Exercise 1) 
\begin{align*}
	p(\bm{y}|\mathcal{M}_m)&\propto \left(\frac{g_m}{1+g_m}\right)^{k_m/2} \left[(\bm{y}-\bar{y}\bm{i}_N)^{\top}(\bm{y}-\bar{y}\bm{i}_N)-\frac{1}{1+g_m}(\bm{y}^{\top}\bm{P}_{X_m}\bm{y})\right]^{-(N-1)/2},
\end{align*}
where all parameter are indexed to model $\mathcal{M}_m$, $\bm{P}_{X_m}=\bm{X}_m(\bm{X}_m^{\top}\bm{X}_m)^{-1}\bm{X}_m$ is the projection matrix on the space generated by the columns of $\bm{X}_m$, and $\bar{y}$ is the sample mean of $\bm{y}$.

We implement in our GUI three approaches to perform BMA in the Gaussian linear model: the BIC approximation using the Occam's window approach, the MC3 algorithm using the analytical expression for calculating the marginal likelihood, and an instrumental variable approach based on conditional likelihoods.\\

\textbf{Example: Simulation exercise}

Let's perform a simulation exercise to assess the performance of the BIC approximation using the Occam's window, and the Markov chain Monte Carlo model composition approaches. Let's set a model where the computational burden is low and we know the data generating process (population statistical model). In particular, we set 10 regressors such that $x_k\sim N(1, 1)$, $k =1,\dots,6$, and $x_k\sim B(0.5)$, $k=7,\dots,10$. We set $\bm{\beta}=[1 \ 0 \ 0 \ 0 \ 0.5 \ 0, 0, 0, 0, -0.7]^{\top}$ such that just $x_1$, $x_5$ and $x_{10}$ are relevant to drive $y_i=1+\bm{x}^{\top}\bm{\beta}+\mu_i$, $\mu_i\sim N(0,0.5^2)$. Observe that we just have $2^{10}=1024$ models in this setting, thus, we can calculate the posterior model probability for each model. 

Our GUI uses the commands \textit{bicreg} and \textit{MC3.REG} from the package \textit{BMA} to perform Bayesian model average in the linear regression model using the BIC approximation and MC3, respectively. These commands in turn are based on \cite{Raftery1995} and \cite{Raftery1997}. The following code shows how to perform the simulation and get the posterior mean and standard deviation using these commands with the default values of hyperparameters and tuning parameters.

\begin{tcolorbox}[enhanced,width=4.67in,center upper,
	fontupper=\large\bfseries,drop shadow southwest,sharp corners]
	\textit{R code. Simulation exercise: Bayesian model average, small setting}
	\begin{VF}
		\begin{lstlisting}[language=R]
rm(list = ls()); set.seed(010101)
N <- 1000
K1 <- 6; K2 <- 4; K <- K1 + K2
X1 <- matrix(rnorm(N*K1,1 ,1), N, K1)
X2 <- matrix(rbinom(N*K2, 1, 0.5), N, K2)
X <- cbind(X1, X2); e <- rnorm(N, 0, 0.5)
B <- c(1,0,0,0,0.5,0,0,0,0,-0.7)
y <- 1 + X%*%B + e
BMAglm <- BMA::bicreg(X, y, strict = FALSE, OR = 50) 
summary(BMAglm)
\end{lstlisting}
	\end{VF}
\end{tcolorbox} 

We can see from the results that the BIC approximation with the Occam's window, and the MC3 algorithm perform a good job finding the relevant regressors, and their posterior BMA means are very close to the population values. We also see that the BMA results are very similar in the two approaches.

\begin{tcolorbox}[enhanced,width=4.67in,center upper,
	fontupper=\large\bfseries,drop shadow southwest,sharp corners]
	\textit{R code. Simulation exercise: Bayesian model average, small setting}
	\begin{VF}
		\begin{lstlisting}[language=R]
BMAreg <- BMA::MC3.REG(y, X, num.its=500)
Models <- unique(BMAreg[["variables"]])
nModels <- dim(Models)[1]
nVistModels <- dim(BMAreg[["variables"]])[1]
PMP <- NULL
for(m in 1:nModels){
	idModm <- NULL
	for(j in 1:nVistModels){
		if(sum(Models[m,] == BMAreg[["variables"]][j,]) == K){
			idModm <- c(idModm, j)
		}else{
			idModm <- idModm
		} 
	}
	PMPm <- sum(BMAreg[["post.prob"]][idModm])
	PMP <- c(PMP, PMPm)
}
PIP <- NULL
for(k in 1:K){
	PIPk <- sum(PMP[which(Models[,k] == 1)])
	PIP <- c(PIP, PIPk)
}
plot(PIP)
Means <- matrix(0, nModels, K)
Vars <- matrix(0, nModels, K)
for(m in 1:nModels){
	idXs <- which(Models[m,] == 1)
	if(length(idXs) == 0){
		Regm <- lm(y ~ 1)
	}else{
		Xm <- X[, idXs]
		Regm <- lm(y ~ Xm)
		SumRegm <- summary(Regm)
		Means[m, idXs] <- SumRegm[["coefficients"]][-1,1]
		Vars[m, idXs] <- SumRegm[["coefficients"]][-1,2]^2 
	} 
}
BMAmeans <- colSums(Means*PMP)
BMAsd <- (colSums(PMP*Vars)  + colSums(PMP*(Means-matrix(rep(BMAmeans, each = nModels), nModels, K))^2))^0.5
BMAmeans
[1]  1.001771e+00 -5.322016e-05  6.635422e-06  3.721457e-07  4.976335e-01
[6] -1.271339e-04  1.000932e-08  2.107441e-05  6.578654e-06 -7.035557e-01 
BMAsd
[1] 1.527261e-02 1.353624e-03 5.936816e-04 1.163947e-04 1.566698e-02 1.987360e-03
[7] 2.778896e-05 1.270579e-03 6.997305e-04 3.093389e-02
BMAmeans/BMAsd
\end{lstlisting}
	\end{VF}
\end{tcolorbox}

We can perform Bayesian model averaging in our GUI for linear Gaussian models using the BIC approximation and MC3 using Algorithms \ref{alg:BMAnormalBIC} and \ref{alg:BMAnormalMC3}, respectively. We ask in Exercise 2 to perform BMA using the dataset \textit{10ExportDiversificationHHI.csv} from \cite{Jetter2015}.  

\begin{algorithm}[h!]
	\caption{Bayesian model average in linear Gaussian models using the Bayesian information criterion}\label{alg:BMAnormalBIC}
	\begin{algorithmic}[1]  		 			
		\State Select \textit{Bayesian Model Averaging} on the top panel
		\State Select \textit{Normal data} model using the left radio button
		\State Select \textit{BIC} using the right radio button under \textbf{Which type do you want to perform?}
		\State Upload the dataset selecting first if there is header in the file, and the kind of separator in the \textit{csv} file of the dataset (comma, semicolon, or tab). Then, use the \textit{Browse} button under the \textbf{Choose File} legend
		\State Type the \textit{OR} number of the Occam's window in the box under \textbf{OR: Number between 5 and 50}, this is not necessary as by default there is 50
		\State Click the \textit{Go!} button
		\State Analyze results: After a few seconds or minutes, a table appears showing, for each regressor in the dataset, the PIP (posterior inclusion probability, \textbf{p!=0}), the BMA posterior mean (\textbf{EV}), the BMA standard deviation (\textbf{SD}), and the posterior mean for models with the highest PMP. At the bottom of the table, for the models with the largest PMP, the number of variables (\textbf{nVar}), the coefficient of determination (\textbf{r2}), the BIC, and the PMP (\textbf{post prob}) are displayed
		\State Download posterior results using the \textit{Download results using BIC}. There are two files, the first has the best models by row according to the PMP (last column) indicating with a 1 inclusion of the variable (0 indicates no inclusion), and the second file has the PIP, the BMA expected value and standard deviation for each variable in the dataset
	\end{algorithmic} 
\end{algorithm}

\begin{algorithm}[h!]
	\caption{Bayesian model average in linear Gaussian models using Markov chain Monte Carlo model composition}\label{alg:BMAnormalMC3}
	\begin{algorithmic}[1]  		 			
		\State Select \textit{Bayesian Model Averaging} on the top panel
		\State Select \textit{Normal data} model using the left radio button
		\State Select \textit{MC3} using the right radio button under \textbf{Which type do you want to perform?}
		\State Upload the dataset selecting first if there is header in the file, and the kind of separator in the \textit{csv} file of the dataset (comma, semicolon, or tab). Then, use the \textit{Browse} button under the \textbf{Choose File} legend
		\State Select MC3 iterations using the \textit{Range slider} under the label \textbf{MC3 iterations:}
		\State Click the \textit{Go!} button
		\State Analyze results: After a few seconds or minutes, a table appears showing, for each regressor in the dataset, the PIP (posterior inclusion probability, \textbf{p!=0}), the BMA posterior mean (\textbf{EV}), the BMA standard deviation (\textbf{SD}), and the posterior mean for models with the highest PMP. At the bottom of the table, for the models with the largest PMP, the number of variables (\textbf{nVar}), the coefficient of determination (\textbf{r2}), the BIC, and the PMP (\textbf{post prob}) are displayed
		\State Download posterior results using the \textit{Download results using BIC}. There are two files, the first has the best models by row according to the PMP (last column) indicating with a 1 inclusion of the variable (0 indicates no inclusion), and the second file has the PIP, the BMA expected value and standard deviation for each variable in the dataset
	\end{algorithmic} 
\end{algorithm}

We show in the following code how to program a MC3 algorithm from scratch to perform BMA using the setting from Section \ref{sec10_2}. The first part of the code is the function to calculate the log marginal likelihood. This is a small simulation setting, thus we can calculate the marginal likelihood for all 1024 models, and then calculate the posterior model probability standardizing using the model with the largest log marginal likelihood. We see from the results that this model is the data generating process (population statistical model). We also find that the posterior inclusion probabilities for $x_{1}$, $x_{5}$ and $x_{10}$ are 1, whereas the PIP for the other variables are less than 0.05. Although BMA allows incorporating model uncertainty in a regression framework, sometimes it is desirable to select just one model. Two compelling alternatives are the model with the largest posterior model probability, and the median probability model. The latter is the model which includes every predictor that has posterior inclusion probability higher than 0.5. The first model is the best alternative for prediction in the case of a 0--1 loss function \cite{Clyde2004}, whereas the second is the best alternative when there is a quadratic loss function in prediction \cite{Barbieri2004}. In this simulation, the two criteria indicate selection of the data generating process.

We also show how to estimate the posterior mean and standard deviation based on BMA. We see that the posterior means are very close to the population parameters.  

\begin{tcolorbox}[enhanced,width=4.67in,center upper,
	fontupper=\large\bfseries,drop shadow southwest,sharp corners]
	\textit{R code. Simulation exercise: Bayesian model average, small setting from scratch}
	\begin{VF}
		\begin{lstlisting}[language=R]
LogMLfunt <- function(Model){
	indr <- Model == 1
	kr <- sum(indr)
	if(kr > 0){
		gr <- ifelse(N > kr^2, 1/N, kr^(-2))
		Xr <- matrix(Xnew[ , indr], ncol = kr)
		PX <- Xr%*%solve(t(Xr)%*%Xr)%*%t(Xr)
		s2pos <- c((t(y - mean(y))%*%(y - mean(y))) - t(y)%*%PX%*%y/(1 + gr))
		mllMod <- (kr/2)*log(gr/(1+gr))-(N-1)/2*log(s2pos)
	}else{
		gr <- ifelse(N > kr^2, 1/N, kr^(-2))
		s2pos <- c((t(y - mean(y))%*%(y - mean(y))))
		mllMod <- (kr/2)*log(gr/(1+gr))-(N-1)/2*log(s2pos)
	}
	return(mllMod)
}
combs <- expand.grid(c(0,1), c(0,1), c(0,1), c(0,1), c(0,1),c(0,1), c(0,1), c(0,1), c(0,1), c(0,1))
Xnew <- apply(X, 2, scale)
mll <- sapply(1:2^K, function(s){LogMLfunt(matrix(combs[s,], 1, K))})
MaxPMP <- which.max(mll); StMarLik <- exp(mll-max(mll))
PMP <- StMarLik/sum(StMarLik)
PMP[MaxPMP]
combs[MaxPMP,]
    Var1 Var2 Var3 Var4 Var5 Var6 Var7 Var8 Var9 Var10
530    1    0    0    0    1    0    0    0    0     1
PIP <- NULL
for(k in 1:K){
	PIPk <- sum(PMP[which(combs[,k] == 1)]); PIP <- c(PIP, PIPk)
}
PIP
[1] 1.00000000 0.03617574 0.03208369 0.03516743 1.00000000 0.04795509 0.03457102 0.03468819 0.03510209 1.00000000
nModels <- dim(combs)[1]; Means <- matrix(0, nModels, K)
Vars <- matrix(0, nModels, K)
for(m in 1:nModels){
	idXs <- which(combs[m,] == 1)
	if(length(idXs) == 0){
		Regm <- lm(y ~ 1)
	}else{
		Xm <- X[, idXs]; Regm <- lm(y ~ Xm)
		SumRegm <- summary(Regm)
		Means[m, idXs] <- SumRegm[["coefficients"]][-1,1]
		Vars[m, idXs] <- SumRegm[["coefficients"]][-1,2]^2 
	}
}
BMAmeans <- colSums(Means*PMP)
1.0018105888 -0.0003196423  0.0001489711  0.0002853524  0.4976225353 -0.0007229563  0.0005342718  0.0005441905  0.0005758708 -0.7035206822
BMAsd <- (colSums(PMP*Vars)  + colSums(PMP*(Means-matrix(rep(BMAmeans, each = nModels), nModels, K))^2))^0.5
0.015274980 0.003304115 0.002814491 0.003214722 0.015668278 0.004694003 0.006400541 0.006435695 0.006528471 0.030940753 
\end{lstlisting}
	\end{VF}
\end{tcolorbox} 

The following part of the code shows how to perform the MC3 algorithm. This algorithm is not necessary in this case due to being a small dimensional problem, but it helps as a pedagogical exercise. The point of departure is to set $S=100$ random models, and order their log marginal likelihoods. Thus, the logic of the algorithm is to pick the worse model among the $S$ models, and propose a candidate model to compete against it. We repeat this MC3 iterations (1000 in the code). Observe that 1000 iterations is less than the number of potential models (1024). This is the idea of the MC3 algorithm, that is, performing less iterations than the number of elements of the space of models. 

In our algorithm, we analyze all model scenarios using different conditionals and reasonably assume the same prior model probability for all models and the same cardinality for both the actual and candidate models. We can calculate the posterior model probability (PMP) in different ways. One way is to recover the unique models from the final set of $S$ models, calculate the log marginal likelihood for these models, and standardize using the best model among them. Another way is to calculate the PMP using the complete set of $S$ final models, accounting for the fact that the same model can appear multiple times in this set, thus requiring us to sum the PMPs of repeated models. An additional way is to calculate the PMP using the relative frequency with which a model appears in the final set of $S$ models. These three methods can yield different PMP, particularly when the number of MC3 iterations is small. In our setting using 1000 MC3 iterations, the data generating process got the largest PMP in the three ways to calculate the PMP. 

A remarkable point in this algorithm is that we can get just one model after substantially increasing the number of iterations (try this code using 10000 iterations). This can be a good feature if we require just one model. However, this neglects model uncertainty, which can be a desirable characteristic. We ask to program an algorithm where we end up with $S$ different models after finishing the MC3 iterations (Exercise 3).

\begin{tcolorbox}[enhanced,width=4.67in,center upper,
	fontupper=\large\bfseries,drop shadow southwest,sharp corners]
	\textit{R code. Simulation exercise: Bayesian model average, small setting from scratch}
	\begin{VF}
		\begin{lstlisting}[language=R]
M <- 100
Models <- matrix(rbinom(K*M, 1, p = 0.5), ncol=K, nrow = M)
mllnew <- sapply(1:M,function(s){LogMLfunt(matrix(Models[s,], 1, K))})
oind <- order(mllnew, decreasing = TRUE)
mllnew <- mllnew[oind]; Models <- Models[oind, ]; iter <- 1000
pb <- winProgressBar(title = "progress bar", min = 0, max = iter, width = 300); s <- 1
while(s <= iter){
	ActModel <- Models[M,]; idK <- which(ActModel == 1)
	Kact <- length(idK)
	if(Kact < K & Kact > 1){
		CardMol <- K; opt <- sample(1:3, 1)
		if(opt == 1){ # Same
			CandModel <- ActModel
		}else{
			if(opt == 2){ # Add
				All <- 1:K; NewX <- sample(All[-idK], 1)
				CandModel <- ActModel; CandModel[NewX] <- 1
			}else{ # Subtract
				LessX <- sample(idK, 1); CandModel <- ActModel
				CandModel[LessX] <- 0
			}
		}
	}else{
		CardMol <- K + 1
		if(Kact == K){
			opt <- sample(1:2, 1)
			if(opt == 1){ # Same
				CandModel <- ActModel
			}else{ # Subtract
				LessX <- sample(1:K, 1); CandModel <- ActModel
				CandModel[LessX] <- 0
			}
		}else{
			if(K == 1){
				opt <- sample(1:3, 1)
				if(opt == 1){ # Same
					CandModel <- ActModel
				}else{
					if(opt == 2){ # Add
						All <- 1:K; NewX <- sample(All[-idK], 1)
						CandModel <- ActModel; CandModel[NewX] <- 1
					}else{ # Subtract
						LessX <- sample(idK, 1); CandModel <- ActModel
						CandModel[LessX] <- 0
					}
				}
			}else{ # Add
				NewX <- sample(1:K, 1); CandModel <- ActModel
				CandModel[NewX] <- 1
			}
		}
	}
\end{lstlisting}
	\end{VF}
\end{tcolorbox} 

\begin{tcolorbox}[enhanced,width=4.67in,center upper,
	fontupper=\large\bfseries,drop shadow southwest,sharp corners]
	\textit{R code. Simulation exercise: Bayesian model average, small setting from scratch}
	\begin{VF}
		\begin{lstlisting}[language=R]
	LogMLact <- LogMLfunt(matrix(ActModel, 1, K))
	LogMLcand <- LogMLfunt(matrix(CandModel, 1, K))
	alpha <- min(1, exp(LogMLcand-LogMLact))
	u <- runif(1)
	if(u <= alpha){
		mllnew[M] <- LogMLcand; Models[M, ] <- CandModel
		oind <- order(mllnew, decreasing = TRUE)
		mllnew <- mllnew[oind]; Models <- Models[oind, ]
	}else{
		mllnew <- mllnew; Models <- Models
	}
	s <- s + 1
	setWinProgressBar(pb, s, title=paste( round(s/iter*100, 0),"% done"))
}
close(pb)
ModelsUni <- unique(Models)
mllnewUni <- sapply(1:dim(ModelsUni)[1], function(s){LogMLfunt(matrix(ModelsUni[s,], 1, K))})
StMarLik <- exp(mllnewUni-mllnewUni[1])
PMP <- StMarLik/sum(StMarLik) # PMP based on unique selected models
nModels <- dim(ModelsUni)[1]
StMarLik <- exp(mllnew-mllnew[1])
PMPold <- StMarLik/sum(StMarLik) # PMP all selected models
PMPot <- NULL
PMPap <- NULL
FreqMod <- NULL
for(m in 1:nModels){
	idModm <- NULL
	for(j in 1:M){
		if(sum(ModelsUni[m,] == Models[j,]) == K){
			idModm <- c(idModm, j)
		}else{
			idModm <- idModm
		}
	}
	PMPm <- sum(PMPold[idModm]) # PMP unique models using sum of all selected models
	PMPot <- c(PMPot, PMPm)
	PMPapm <- length(idModm)/M # PMP using relative frequency in all selected models
	PMPap <- c(PMPap, PMPapm)
	FreqMod <- c(FreqMod, length(idModm))
}
\end{lstlisting}
	\end{VF}
\end{tcolorbox} 
 
An important issue to account for regressors (model) uncertainty in the identification of causal effects, rather than finding good predictors (association relationships), is endogeneity. Thus, we also implement the instrumental variable approach of Section \ref{sec73} to tackle this issue in BMA. We assume that $\bm{\gamma}\sim {N}(\bm{0},\bm{I})$, $\bm{\beta}\sim {N}(\bm{0},\bm{I})$, and $\bm{\Sigma}^{-1} \sim {W}(3,\bm{I})$ \cite{Karl2012}.

\cite{Lenkoski2013} propose an algorithm based on conditional Bayes factors \cite{Dickey1978} that allows embedding MC3 within a Gibbs sampling algorithm. Given the candidate ($M_{c}^{2nd}$) and actual ($M_{s-1}^{2nd}$) models for the iteration $s$ in the second stage, the conditional Bayes factor is 
\begin{equation*}
	CBF^{2nd}=\frac{p(\bm{y}|M_{c}^{2nd},\bm{\gamma},\bm{\Sigma})}{p(\bm{y}|M_{s-1}^{2nd},\bm{\gamma},\bm{\Sigma})},
\end{equation*}
where 
\begin{equation*}
	p(\bm{y}|M_{c}^{2nd},\bm{\gamma},\bm{\Sigma})=\int_{\mathcal{M}^{2nd}}p(\bm{y}|\bm{\beta},\bm{\gamma},\bm{\Sigma})\pi(\bm{\beta}|M_{c}^{2nd})d\bm{\beta}\propto |\bm{B}_n|^{1/2} \exp\left\{\frac{1}{2}{\bm{\beta}_n}^{\top}\bm{B}_n^{-1}\bm{\beta}_n\right\}
	.
\end{equation*}

In the first stage,
\begin{equation*}
	CBF^{1st}=\frac{p(\bm{y}|M_{c}^{1st},\bm{\beta},\bm{\Sigma})}{p(\bm{y}|M_{s-1}^{1st},\bm{\beta},\bm{\Sigma})},
\end{equation*}
where \begin{equation*}
	p(\bm{y}|M_{c}^{1st},\bm{\beta},\bm{\Sigma})=\int_{\mathcal{M}^{1st}}p(\bm{y}|\bm{\gamma},\bm{\beta},\bm{\Sigma})\pi(\bm{\gamma}|M_{c}^{1st})d\bm{\gamma}\propto |\bm{G}_n|^{1/2} \exp\left\{\frac{1}{2}{\bm{\gamma}_n}^{\top}\bm{G}_n^{-1}\bm{\gamma}_n\right\}.
\end{equation*}
These conditional Bayes factors assume $\pi(M^{1st},M^{2sd})\propto 1$. See \cite{Lenkoski2013} for more details of the instrumental variable BMA algorithm.\footnote{\cite{Koop12} and \cite{Lenkoski2014} propose other frameworks for BMA taking into account endogeneity.}

We perform instrumental variable BMA in our GUI using the package \textit{ivbma}. The Algorithm \ref{alg:BMAIV} shows how to perform this in our GUI. 

\begin{algorithm}[h!]
	\caption{Instrumental variable Bayesian model average in linear Gaussian models}\label{alg:BMAIV}
	\begin{algorithmic}[1]  		 			
		\State Select \textit{Bayesian Model Averaging} on the top panel
		\State Select \textit{Normal data} model using the left radio button
		\State Select \textit{Instrumental variable} using the right radio button under \textbf{Which type do you want to perform?}
		\State Upload the dataset containing the dependent variable, endogenous regressors, and exogenous regressors including the constant (see Section \ref{secGUI6} for details).  User should select first if there is header in the file, and the kind of separator in the \textit{csv} file of the dataset (comma, semicolon, or tab). Then, use the \textit{Browse} button under the \textbf{Choose File} legend
		\State Upload the dataset containing the instruments (see Section \ref{secGUI6} for details).  User should select first if there is header in the file, and the kind of separator in the \textit{csv} file of the dataset (comma, semicolon, or tab). Then, use the \textit{Browse} button under the \textbf{Choose File (Instruments)} legend
		\State Write down the number of endogenous regressors in the box labeled \textbf{Number of Endogenous variables}
		\State Select MCMC iterations and burn-in using the \textit{Range slider} under the labels \textbf{MCMC iterations:} and \textbf{Burn-in Sample:}
		\State Click the \textit{Go!} button
		\State Analyze results: After a few seconds or minutes, two tables appear showing, for each regressor in the dataset, the PIP (posterior inclusion probability, \textbf{p!=0}), and the BMA posterior mean (\textbf{EV}). The top table shows the results of the second stage (main equation), and the bottom table shows the results of the first stage (auxiliary equations)
		\State Download posterior results using the \textit{Download results using IV}. There are three files, the first file has the posterior inclusion probabilities of each variable, and the BMA posterior means of the coefficients in the first stage equations, the second file shows these results for the second stage (main equation), and the third file has the posteriors chains of all parameters by iteration. 
	\end{algorithmic} 
\end{algorithm}
Let's perform a simulation exercise to assess the performance of the instrumental variable BMA to uncover the data generating process in presence of endogeneity.\\

\textbf{Example: Simulation exercise}

Let's assume that $y_i=2+0.5x_{i1}-x_{i2}+x_{i3}+\mu_i$ where $x_{i1}=4z_{i1}-z_{i2}+2z_{i3}+\epsilon_{i1}$ and $x_{i2}=-2z_{i1}+3_{i2}-z_{i3}+\epsilon_{i2}$ such that $[\epsilon_{i1} \ \epsilon_{i2} \ \mu_i]^{\top}\sim N(\bm{0}, \bm{\Sigma})$ where $\bm{\Sigma}=\begin{bmatrix} 1 & 0 & 0.8\\
0 & 1 & 0.5\\
0.8 & 0.5 & 1\end{bmatrix}$, $i=1,2,\dots,1000$. The endogeneity is due to the correlation between $\mu_i$ and $x_{i1}$ and $x_{i2}$ through the stochastic errors. In addition, there are three instruments, $z_{il}\sim U(0,1)$, $l=1,2,3$, and another 18 regressors believed to influence $y_i$, which are distributed according to a standard normal distribution.

The following code shows how to perform IV BMA using the \textit{ivbma} package. We see from the results that the PIP of $x_{i1}$, $x_{i2}$, intercept and $x_{i3}$ are equal to 1, whereas the remaining PIP are close to 0. In addition, the BMA means are also close to the population values. The PIP of the first stage equations, as well as their BMA posterior means, are very close to the populations values. The same happens with the covariance matrix. 

We ask in Exercise 4 to perform BMA based on the BIC approximation and MC3 in this simulation setting. In addition, we ask in Exercise 5 to use the datasets \textit{11ExportDiversificationHHI.csv} and \textit{12ExportDiversificationHHIInstr.csv} to perform IV BMA assuming that the log of per capita gross domestic product is endogenous (\textit{avglgdpcap}). See \cite{Jetter2015} for details.     

\begin{tcolorbox}[enhanced,width=4.67in,center upper,
	fontupper=\large\bfseries,drop shadow southwest,sharp corners]
	\textit{R code. Simulation exercise: Instrumental variable Bayesian model average}
	\begin{VF}
		\begin{lstlisting}[language=R]
rm(list = ls())
set.seed(010101)
simIV <- function(delta1,delta2,beta0,betas1,betas2,beta2,Sigma,n,z) {
	eps <- matrix(rnorm(3*n),ncol=3) %*% chol(Sigma)
	xs1 <- z%*%delta1 + eps[,1]
	xs2 <- z%*%delta2 + eps[,2]
	x2 <- rnorm(dim(z)[1])
	y <- beta0+betas1*xs1+betas2*xs2+beta2*x2 + eps[,3]
	X <- as.matrix(cbind(xs1,xs2,1,x2)) 
	colnames(X) <- c("x1en","x2en","cte","xex")
	y <- matrix(y,dim(z)[1],1)
	colnames(y) <- c("y")
	list(X=X,y=y)
}
n <- 1000 ; p <- 3 
z <- matrix(runif(n*p),ncol=p)
rho31 <- 0.8; rho32 <- 0.5;
Sigma <- matrix(c(1,0,rho31,0,1,rho32,rho31,rho32,1),ncol=3)
delta1 <- c(4,-1,2); delta2 <- c(-2,3,-1); betas1 <- .5; betas2 <- -1; beta2 <- 1; beta0 <- 2
simiv <- simIV(delta1,delta2,beta0,betas1,betas2,beta2,Sigma,n,z)
nW <- 18
W <- matrix(rnorm(nW*dim(z)[1]),dim(z)[1],nW)
YXW<-cbind(simiv$y, simiv$X, W)
y <- YXW[,1]; X <- YXW[,2:3]; W <- YXW[,-c(1:3)]
S <- 10000; burnin <- 1000
regivBMA <- ivbma::ivbma(Y = y, X = X, Z = z, W = W, s = S+burnin, b = burnin, odens = S, print.every = round(S/10), run.diagnostics = FALSE)
PIPmain <- regivBMA[["L.bar"]] # PIP outcome
PIPmain
1.0000 1.0000 1.0000 1.0000 0.0125 0.0382 0.0145 0.0148 0.0136 0.0102 0.0070 0.0527 0.0014 0.0077 0.0211 0.0081 0.0047 0.0141 0.0028 0.0063 0.0072 0.0220
EVmain <- regivBMA[["rho.bar"]] # Posterior mean outcome
EVmain
5.105361e-01 -9.828459e-01  1.996885e+00  1.005497e+00 -1.700857e-04  9.946613e-04  1.086717e-04 -1.448951e-04  1.532812e-04  1.356334e-04 -6.027285e-05  9.119699e-04 -1.581408e-05  1.050517e-04 2.488002e-04 -6.229493e-05  4.292825e-05  3.371366e-05  5.345760e-06  5.933764e-05 5.066236e-05 1.516718e-04
PIPaux <- regivBMA[["M.bar"]] # PIP auxiliary
EVaux <- regivBMA[["lambda.bar"]] # Posterior mean auxiliary
plot(EVaux[,1])
plot(EVaux[,2])
EVsigma <- regivBMA[["Sigma.bar"]] # Posterior mean variance matrix
\end{lstlisting}
\end{VF}
\end{tcolorbox} 
     


\section{Generalized linear models}\label{sec10_3}

Generalized linear models (GLM) were introduced by \cite{nelder1972generalized}, and extend the concept of linear regressions to a more general setting. These models are characterized by: i) a dependent variable ${y}_i$ whose probability distribution function belongs to the exponential family (see Section \ref{sec41}), ii) a linear predictor $\eta=\bm{x}^{\top}\bm{\beta}$, and iii) a link function such that $\mathbb{E}[Y|\bm{x}]=g^{-1}(\bm{x}^{\top}\bm{\beta})$, which implies that $g(\mathbb{E}[Y|\bm{x}])=\bm{x}^{\top}\bm{\beta}$. GLM can be extended to overdispersed exponential family \cite{McCullagh1989}.

We know from Section \ref{sec41} that the Poisson distribution belongs to the exponential family such that $p(y|\lambda)=\frac{\exp(-\lambda)\exp(y\log(\lambda))}{y!}$ or in the canonical form $p(y|\eta)=\frac{\exp(\eta y-\exp(\eta))}{y!}$, where $\eta=\log(\lambda)$, which means that $\bm{x}^{\top}\bm{\beta}=\log(\lambda)$, and consequently, $\mathbb{E}[Y|\bm{x}]=\nabla(\exp(\eta))=\exp(\eta)=\lambda=\exp(\bm{x}^{\top}\bm{\beta})$. Then, the link function in the Poisson case is the \textit{log} function. We ask in Exercise 6 to show that the link function in Bernoulli case is the \textit{logit} function. Another examples are the identity function in the case of the Gaussian distribution, and the negative inverse in the case of the gamma density.

We can use the setting of the GLM to perform BMA using the BIC approximation following \cite{Raftery1995}. In particular, $BIC=k_m\log(N)-2\log(p(\hat{\bm{\theta}_m}|\bm{y}))$, where $\hat{\bm{\theta}}_m$ is the maximum likelihood estimator. Thus, we just need to calculate the likelihood function at the maximum likelihood estimator.\\

\textbf{Example: Simulation exercises}

Let's perform some simulation exercises to assess the performance of the BIC approximation using the Occam's window in GLMs. There are 27 regressors, where $x_{i1}$ and $x_{i2}$ are just the relevant regressors in all exercises, $i=1,2,\dots,1000$.
\begin{itemize}
	\item Logit: $x_k\sim N(0, 1)$, $k =1,\dots,27$, and $P(Y_i=1|\bm{x}_i)=\exp(0.5+0.8x_{i1}-1.2x_{i2})/(1+\exp(0.5+0.8x_{i1}-1.2x_{i2}))$.
	
	\item Gamma: $x_k\sim N(0, 0.5^2)$, $k =1,\dots,27$, and $y_i\sim G(\alpha,\delta)$ where $\alpha=-(0.5+0.2x_{i1}0.1x_{i2})^{-1}$ and $\delta=1$.
	
	\item Poisson: $x_k\sim N(0, 1)$, $k =1,\dots,27$, and $\mathbb{E}[Y_i|\bm{x}_i]=\lambda_i=\exp(0.5+1.1x_{i1}+0.7x_{i2})$.   
\end{itemize}

Our GUI uses the command \textit{bic.glm} from the \textit{BMA} package to perform BMA using the BIC approximation with the Occam's window in GLMs. The Algorithm \ref{alg:BMABIC} shows how to do this in our GUI, and the following code shows how to perform BMA in logit models using the simulation setting.

\begin{algorithm}[h!]
	\caption{Bayesian model average in generalized linear models using the Bayesian information criterion}\label{alg:BMABIC}
	\begin{algorithmic}[1]  		 			
		\State Select \textit{Bayesian Model Averaging} on the top panel
		\State Select the generalized linear model using the left radio button. Options: \textit{Binomial data (Logit)}, \textit{Real positive data (Gamma)} and \textit{Count data (Poisson)}
		\State Upload the dataset selecting first if there is header in the file, and the kind of separator in the \textit{csv} file of the dataset (comma, semicolon, or tab). Then, use the \textit{Browse} button under the \textbf{Choose File} legend
		\State Type the \textit{OR} number of the Occam's window in the box under \textbf{OR: Number between 5 and 50}, this is not necessary as by default there is 50
		\State Type the \textit{OL} number of the Occam's window in the box under \textbf{OL: Number between 0.0001 and 1}, this is not necessary as by default there is 0.0025
		\State Click the \textit{Go!} button
		\State Analyze results: After a few seconds or minutes, a table appears showing, for each regressor in the dataset, the PIP (posterior inclusion probability, \textbf{p!=0}), the BMA posterior mean (\textbf{EV}), the BMA standard deviation (\textbf{SD}), and the posterior mean for models with the highest PMP. At the bottom of the table, for the models with the largest PMP, the number of variables (\textbf{nVar}), the BIC, and the PMP (\textbf{post prob}) are displayed
		\State Download posterior results using the \textit{Download results using BIC}. There are two files, the first has the best models by row according to the PMP (last column) indicating with a 1 inclusion of the variable (0 indicates no inclusion), and the second file has the PIP, the BMA expected value and standard deviation for each variable in the dataset
	\end{algorithmic} 
\end{algorithm}

The results show that the PIPs of $x_{i1}$ and $x_{i2}$ are equal 1 in all three settings, the data generating process gets the highest PMP, and the BMA posterior means are close to the population values in each simulation setting. The other variables get PIPs close to 0, except a few exceptions, and the BMA posterior means are also close to 0. This suggests that the BIC approximation does a good job finding the data generating process in generalized linear models.
 
\begin{tcolorbox}[enhanced,width=4.67in,center upper,
	fontupper=\large\bfseries,drop shadow southwest,sharp corners]
	\textit{R code. Simulation exercise: BMA for generalized linear models}
	\begin{VF}
		\begin{lstlisting}[language=R]
### Logit ###
rm(list = ls()); set.seed(010101)
n<-1000; B<-c(0.5,0.8,-1.2)
X<-matrix(cbind(rep(1,n),rnorm(n,0,1),rnorm(n,0,1)),n,length(B))
p <- exp(X%*%B)/(1+exp(X%*%B)); y <- rbinom(n, 1, p)
nXgar<-25; Xgar<-matrix(rnorm(nXgar*n),n,nXgar)
df<-as.data.frame(cbind(y,X[,-1],Xgar))
colnames(df) <- c("y", "x1", "x2", "x3", "x4", "x5", "x6", "x7", "x8", "x9", "x10", "x11", "x12", "x13", "x14", "x15", "x16", "x17", "x18", "x19", "x20", "x21", "x22", "x23", "x24", "x25", "x26", "x27")
BMAglmLogit <- BMA::bic.glm(y ~ x1+x2+x3+x4+x5+x6+x7+x8+x9+x10+x11+x12+x13+x14+x15+x16+x17+x18+x19+x20+x21+x22+x23+x24+x25+x26+x27, data = df, glm.family = binomial(link="logit"), strict = FALSE, OR = 50)
summary(BMAglmLogit)
### Gamma ###
rm(list = ls()); set.seed(010101)
n<-1000; B<- c(0.5, 0.2, 0.1)
X<-matrix(cbind(rep(1,n),rnorm(n,0,0.5),rnorm(n,0,0.5)),n,length(B))
y1 <- (X%*%B)^(-1)
y <- rgamma(n,y1,scale=1)
nXgar<-25; Xgar<-matrix(rnorm(nXgar*n),n,nXgar)
df<-as.data.frame(cbind(y,X[,-1],Xgar))
colnames(df) <- c("y", "x1", "x2", "x3", "x4", "x5", "x6", "x7", "x8", "x9", "x10", "x11", "x12", "x13", "x14", "x15", "x16", "x17", "x18", "x19", "x20", "x21", "x22", "x23", "x24", "x25", "x26", "x27")
BMAglmGamma <- BMA::bic.glm(y ~ x1+x2+x3+x4+x5+x6+x7+x8+x9+x10+x11+x12+x13+x14+x15+x16+x17+x18+x19+x20+x21+x22+x23+x24+x25+x26+x27, data = df, glm.family = Gamma(link="inverse"), strict = FALSE, OR = 50)
summary(BMAglmGamma)
### Poisson ###
rm(list = ls()); set.seed(010101)
n<-1000; B<-c(2,1.1,0.7)
X<-matrix(cbind(rep(1,n),rnorm(n,0,1),rnorm(n,0,1)),n,length(B))
y1<-exp(X%*%B); y<-rpois(n,y1)
nXgar<-25; Xgar<-matrix(rnorm(nXgar*n),n,nXgar)
df<-as.data.frame(cbind(y,X[,-1],Xgar))
colnames(df) <- c("y", "x1", "x2", "x3", "x4", "x5", "x6", "x7", "x8", "x9", "x10", "x11", "x12", "x13", "x14", "x15", "x16", "x17", "x18", "x19", "x20", "x21", "x22", "x23", "x24", "x25", "x26", "x27")
BMAglmPoisson <- BMA::bic.glm(y ~ x1+x2+x3+x4+x5+x6+x7+x8+x9+x10+x11+x12+x13+x14+x15+x16+x17+x18+x19+x20+x21+x22+x23+x24+x25+x26+x27, data = df, glm.family = poisson(link="log"), strict = FALSE, OR = 50)
summary(BMAglmPoisson)
\end{lstlisting}
	\end{VF}
\end{tcolorbox} 
  
We can take advantage of the \textit{glm} function in \textbf{R} to perform BMA programming a MC3 algorithm. The following code shows how to do this. Fist, we simulate the data, second we have a function to get the log marginal likelihood approximation using the results from the  \textit{glm} function. Then, we have the initial models to begin the MC3 algorithm. After this, we have the MC3 algorithm using small modifications of the code that we use to perform MC3 in Gaussian linear models. We can calculate the PMPs, PIPs, BMA means and standard deviations as we previously did.

The simulation setting implies $2^{27}$ models, which implies approximately 135 million models in the model space. We run our MC3 algorithm using the BIC approximation with 50000 iterations. This takes by far more time that the BIC approximation from the \textit{BMA} package, but it seems to do a good job finding the data generating process as the PMP of this model is equal 1, the posterior inclusion probabilities are equal 1 for $x_{i1}$ and $x_{i2}$, the posterior means are 1.1 and 0.7, that is, equal to the population values, and the t-ratios are by far higher than 2. However, running 50000 iterations implies mass concentration in one model, in this case the data generating process. If we run 25000 MC3 iterations, the highest PMP is 0.8, but it is not associated with the data generating process. Although, the PIP is equal 1 for $x_{i1}$ and $x_{i2}$, but there are other regressors that get high PIPs. The BMA means are equal to the population values for $x_{i1}$ and $x_{i2}$, and the PIPs for the other regressors are equal 0. The t-ratios of the regressors in the population statistical model are larger than 2 by far, whereas the t-ratios of the other regressors are equal to 0. This exercise shows that 25000 iterations were not enough to uncover the data generating process. However, this exercise also shows an important point, we need too analyze all the relevant results from the BMA analysis, no just the PMPs and/or PIPs.

We ask in Exercise 10 to use this approach to perform a BMA algorithm in the logit regression using the simulation setting of logit models of this section.    

\begin{tcolorbox}[enhanced,width=4.67in,center upper,
	fontupper=\large\bfseries,drop shadow southwest,sharp corners]
	\textit{R code. Simulation exercise: BMA for generalized linear models using MC3 from scratch}
	\begin{VF}
		\begin{lstlisting}[language=R]
rm(list = ls()); set.seed(010101)
n<-1000; B<-c(2,1.1,0.7)
X<-matrix(cbind(rep(1,n),rnorm(n,0,1),rnorm(n,0,1)),n,length(B))
y1<-exp(X%*%B); y<-rpois(n,y1)
nXgar<-25; Xgar<-matrix(rnorm(nXgar*n),n,nXgar)
df<-as.data.frame(cbind(y,X[,-1],Xgar))
colnames(df) <- c("y", "x1", "x2", "x3", "x4", "x5", "x6", "x7", "x8", "x9", "x10", "x11", "x12", "x13", "x14", "x15", "x16", "x17", "x18", "x19", "x20", "x21", "x22", "x23", "x24", "x25", "x26", "x27")
Xnew <- apply(df[,-1], 2, scale)
BICfunt <- function(Model){
	indr <- Model == 1; kr <- sum(indr)
	if(kr > 0){
		Xr <- as.matrix(Xnew[ , indr])
		model <- glm(y ~ Xr, family = poisson(link = "log"))
		model_bic <- BIC(model)
		mllMod <- -model_bic/2
	}else{
		model <- glm(y ~ 1, family = poisson(link = "log"))
		model_bic <- BIC(model); mllMod <- -model_bic/2
	}
	return(mllMod)
}
M <- 500; K <- dim(df)[2] - 1
Models <- matrix(rbinom(K*M, 1, p = 0.5), ncol = K, nrow = M)
mllnew <- sapply(1:M, function(s){BICfunt(matrix(Models[s,], 1, K))})
oind <- order(mllnew, decreasing = TRUE)
mllnew <- mllnew[oind]; Models <- Models[oind, ]
# Hyperparameters MC3
iter <- 25000
pb <- winProgressBar(title = "progress bar", min = 0, max = iter, width = 300)
s <- 1
while(s <= iter){
	ActModel <- Models[M,]
	idK <- which(ActModel == 1)
	Kact <- length(idK)
	if(Kact < K & Kact > 1){
		CardMol <- K
		opt <- sample(1:3, 1)
		if(opt == 1){ # Same
			CandModel <- ActModel
		}else{
					if(opt == 2){ # Add
			All <- 1:K
			NewX <- sample(All[-idK], 1)
			CandModel <- ActModel
			CandModel[NewX] <- 1
		}else{ # Subtract
			LessX <- sample(idK, 1)
			CandModel <- ActModel
			CandModel[LessX] <- 0
		}
	}
\end{lstlisting}
	\end{VF}
\end{tcolorbox} 

\begin{tcolorbox}[enhanced,width=4.67in,center upper,
	fontupper=\large\bfseries,drop shadow southwest,sharp corners]
	\textit{R code. Simulation exercise: BMA for generalized linear models using MC3 from scratch}
	\begin{VF}
		\begin{lstlisting}[language=R]
	}else{
		CardMol <- K + 1
		if(Kact == K){
			opt <- sample(1:2, 1)
			if(opt == 1){ # Same
				CandModel <- ActModel
			}else{ # Subtract
				LessX <- sample(1:K, 1)
				CandModel <- ActModel
				CandModel[LessX] <- 0
			}
		}else{
			if(K == 1){
				opt <- sample(1:3, 1)
				if(opt == 1){ # Same
					CandModel <- ActModel
				}else{
					if(opt == 2){ # Add
						All <- 1:K
						NewX <- sample(All[-idK], 1)
						CandModel <- ActModel
						CandModel[NewX] <- 1
					}else{ # Subtract
						LessX <- sample(idK, 1)
						CandModel <- ActModel
						CandModel[LessX] <- 0
					}
				}
			}else{ # Add
				NewX <- sample(1:K, 1)
				CandModel <- ActModel
				CandModel[NewX] <- 1
			}
		}
	}
	LogMLact <- BICfunt(matrix(ActModel, 1, K))
	LogMLcand <- BICfunt(matrix(CandModel, 1, K))
	alpha <- min(1, exp(LogMLcand-LogMLact))
	u <- runif(1)
	if(u <= alpha){
		mllnew[M] <- LogMLcand
		Models[M, ] <- CandModel
		oind <- order(mllnew, decreasing = TRUE)
		mllnew <- mllnew[oind]
		Models <- Models[oind, ]
	}else{
		mllnew <- mllnew
		Models <- Models
	}
	s <- s + 1
	setWinProgressBar(pb, s, title=paste( round(s/iter*100, 0),"% done"))
}
close(pb)
\end{lstlisting}
	\end{VF}
\end{tcolorbox}   
 
\begin{tcolorbox}[enhanced,width=4.67in,center upper,
	fontupper=\large\bfseries,drop shadow southwest,sharp corners]
	\textit{R code. Simulation exercise: BMA for generalized linear models using MC3 from scratch}
	\begin{VF}
		\begin{lstlisting}[language=R]
ModelsUni <- unique(Models)
mllnewUni <- sapply(1:dim(ModelsUni)[1], function(s){BICfunt(matrix(ModelsUni[s,], 1, K))})
StMarLik <- exp(mllnewUni-mllnewUni[1])
PMP <- StMarLik/sum(StMarLik) # PMP based on unique selected models
plot(PMP)
ModelsUni[1,]
PIP <- NULL
for(k in 1:K){
	PIPk <- sum(PMP[which(ModelsUni[,k] == 1)])
	PIP <- c(PIP, PIPk)
}
plot(PIP)
Xnew <- df[,-1]
nModels <- dim(ModelsUni)[1]
Means <- matrix(0, nModels, K)
Vars <- matrix(0, nModels, K)
for(m in 1:nModels){
	idXs <- which(ModelsUni[m,] == 1)
	if(length(idXs) == 0){
		Regm <- glm(y ~ 1, family = poisson(link = "log"))
	}else{
		Xm <- as.matrix(Xnew[, idXs])
		Regm <- glm(y ~ Xm, family = poisson(link = "log"))
		SumRegm <- summary(Regm)
		Means[m, idXs] <- SumRegm[["coefficients"]][-1,1]
		Vars[m, idXs] <- SumRegm[["coefficients"]][-1,2]^2 
	}
}
BMAmeans <- colSums(Means*PMP)
BMAsd <- (colSums(PMP*Vars)  + colSums(PMP*(Means-matrix(rep(BMAmeans, each = nModels), nModels, K))^2))^0.5 
plot(BMAmeans)
plot(BMAsd)
plot(BMAmeans/BMAsd)
\end{lstlisting}
	\end{VF}
\end{tcolorbox}   
 


%The Gaussian linear model is an example of a generalized linear model. A GLM is characterized by a distribution function that is in the exponential family, that is, $p_i(y_i|\theta_i,\phi)=h(y_i,\phi)Exp\left\{(\theta_iy_i-b(\theta_i))/a(\phi)\right\}$ (canonical representation), $y_i\stackrel{i.n.d.} {\thicksim}p_i$, $i=1,2,\dots,n$. It also has a linear predictor $\theta_i=\bm{x}_i^{\top}\bm{\beta}$, and a link function $g$ such that $E(Y_i|x_i)\equiv \mu_i=b'(\theta_i)=g^{-1}(\bm{x}_i^{\top}\bm{\beta})$ ($g$ is monotonic and differentiable), and $V(Y_i)=b''(\theta_i)a(\phi)$ \cite{McCullagh1989}. The identity function $\mu_i=\bm{x}_i^{\top}\bm{\beta}$ is the canonical link function in the case of the Gaussian model.\footnote{A canonical link functions is characterized by the existence of a sufficient statistic ($\bm{X}^{\top}\bm{y}$) equal in dimension to $\bm{\beta}$.} This statistical framework can help us to characterize:

\section{Calculating the marginal likelihood}\label{sec10_4}

The BIC is an asymptotic shortcut to approximate the marginal likelihood, and consequently, obtain the Bayes factors. However, this has limitations in moderate and small sample size applications \cite{gelfand1994bayesian}. Thus, there are other methods to calculate the Bayes factors when there is no an analytical solution of the marginal likelihood.

Observe that calculating the Bayes factor with respect to a reference model ($\mathcal{M}_0$) help to obtain the posterior model probabilities,

\begin{align*}
	\pi(\mathcal{M}_j |\bm{y})&=\frac{p(\bm{y} | \mathcal{M}_j)\pi(\mathcal{M}_j)}{\sum_{m=1}^{M}p(\bm{y} | \mathcal{M}_m)\pi(\mathcal{M}_m)}\\
	&=\frac{p(\bm{y} | \mathcal{M}_j)\pi(\mathcal{M}_j)/p(\bm{y} | \mathcal{M}_0)}{\sum_{m=1}^{M}p(\bm{y} | \mathcal{M}_m)\pi(\mathcal{M}_m)/p(\bm{y} | \mathcal{M}_0)}\\
	&=\frac{BF_{j0}\times\pi(\mathcal{M}_j)}{\sum_{m=1}^{M}BF_{l0}\times\pi(\mathcal{M}_l)}.
\end{align*}

Thus, $\pi(\mathcal{M}_j |\bm{y})=\frac{BF_{j0}}{\sum_{m=1}^{M}BF_{l0}}$ assuming equal prior model probabilities.

In addition, it has been established in many settings that the Bayes factor is consistent, that is, the probability of uncovering the true data generating process converges to 1 when the sample size converges to infinity, or, it asymptotically identifies the model that minimizes the Kullback-Leibler divergence with respect to the data generating process when this is no part of the models into consideration \cite{chib2016bayes,walker2004new,walker2004modern}.\footnote{\cite{Johnson2012} highlight the important difference between pairwise consistency, and model selection consistency. The latter requires consistency of a sequence of pairwise nested comparisons.}  

\subsection{Savage-Dickey density ratio}\label{sec10_4_1}

The Savage-Dickey density ratio is a way to calculate the Bayes factors when we compare nested models with particular priors \cite{dickey1971weighted,verdinelli1995computing}. In particular, given the parameter space $\bm{\theta}=(\bm{\omega}^{\top}, \bm{\psi}^{\top})^{\top}\in \bm{\Theta}=\bm{\Omega}\times \bm{\Psi}$, where we wish to test the null hypothesis $H_0:\bm{\omega}=\bm{\omega}_0$ (model $\mathcal{M}_1$) versus $H_1:\bm{\omega}\neq \bm{\omega}_0$ (model $\mathcal{M}_2$), if $\pi(\bm{\psi}|\bm{\omega}_0,\mathcal{M}_2)=\pi(\bm{\psi}|\mathcal{M}_1)$,\footnote{Note that a sufficient condition for this assumption is to assume the same prior for the parameters that are the same in each model. \cite{verdinelli1995computing} incorporate a correction factor when this assumption is not satisfied.} then the Bayes factor comparing $\mathcal{M}_1$ versus $\mathcal{M}_2$ is

\begin{equation}\label{eq:SD}
	BF_{12}=\frac{\pi(\bm{\omega}=\bm{\omega}_0|\bm{y},\mathcal{M}_2)}{\pi(\bm{\omega}=\bm{\omega}_0|\mathcal{M}_2)},
\end{equation}
where $\pi(\bm{\omega}=\bm{\omega}_0|\bm{y},\mathcal{M}_2)$ and $\pi(\bm{\omega}=\bm{\omega}_0|\mathcal{M}_2)$ are the posterior and prior densities of $\bm{\omega}$ under $\mathcal{M}_2$ evaluated at $\bm{\omega}_0$ (see \cite{verdinelli1995computing}). 

Equation \ref{eq:SD} is called the Savage-Dickey density ratio. A nice feature is that just requires estimation of model $\mathcal{M}_2$, and evaluation of the prior and posterior densities. This means no evaluation of the marginal likelihood \cite[Chap.~4]{koop2003bayesian}.\\

\subsection{Chib's methods}\label{sec10_4_2}

Another popular method to calculate the marginal likelihood is given by \cite{chib1995marginal} and \cite{chib2001marginal}. The former is an algorithm to calculate the marginal likelihood from the posterior draws of the Gibbs sampling algorithm, and the latter calculates the marginal likelihood from the posterior draws of the Metropolis-Hastings algorithm.

The point of departure in \cite{chib1995marginal} is the identity
\begin{align*}
	\pi(\bm{\theta}^*|\bm{y},\mathcal{M}_m)=\frac{p(\bm{y}|\bm{\theta}^*,\mathcal{M}_m)\times\pi(\bm{\theta}^*|\mathcal{M}_m)}{p(\bm{y}|\mathcal{M}_m)},
\end{align*} 
where $\bm{\theta}^*$ is a particular value of $\bm{\theta}$ of high probability, for instance, the mode. This implies that
\begin{align*}
	p(\bm{y}|\mathcal{M}_m)=\frac{p(\bm{y}|\bm{\theta}^*,\mathcal{M}_m)\times\pi(\bm{\theta}^*|\mathcal{M}_m)}{\pi(\bm{\theta}^*|\bm{y},\mathcal{M}_m)}.
\end{align*} 
We can easily calculate the numerator of this expression. However, the critical point in this expression is to calculate the denominator as we know $\pi(\bm{\theta}^*|\bm{y},\mathcal{M}_m)$ up to a normalizing constant. We can calculate this from the posterior draws. Assume that $\bm{\theta}=[\bm{\theta}^{\top}_1 \ \bm{\theta}^{\top}_2]^{\top}$, then $\pi(\bm{\theta}^*|\bm{y},\mathcal{M}_m)=\pi(\bm{\theta}^*_1|\bm{\theta}^*_2,\bm{y},\mathcal{M}_m)\times \pi(\bm{\theta}^*_2|\bm{y},\mathcal{M}_m)$. We have the first term because in the Gibbs sampling algorithm the posterior conditional distributions are available. The second is

\begin{align*}
	\pi(\bm{\theta}^*_2|\bm{y},\mathcal{M}_m)&=\int_{\bm{\Theta}_1}\pi(\bm{\theta}_1,\bm{\theta}^*_2|\bm{y},\mathcal{M}_m)d\bm{\theta}_1\\
	&=\int_{\bm{\Theta}_1}\pi(\bm{\theta}^*_2|\bm{\theta}_1,\bm{y},\mathcal{M}_m)\pi(\bm{\theta}_1|\bm{y},\mathcal{M}_m)d\bm{\theta}_1\\
	&\approx \frac{1}{S}\sum_{s=1}^S \pi(\bm{\theta}^*_2|\bm{\theta}^{(s)}_1,\bm{y},\mathcal{M}_m),
\end{align*} 

where $\bm{\theta}^{(s)}_1$ are the posterior draws of $\bm{\theta}_1$ from the Gibbs sampling algorithm. 

The generalization to more blocks can be seen in \cite{chib1995marginal} and \cite[Chap.~7]{greenberg2012introduction}. In addition, the extension to the Metropolis-Hastings algorithm can be seen in \cite{chib2001marginal}, and \cite[Chap.~7]{greenberg2012introduction}.

\subsection{Gelfand-Dey method}\label{sec10_4_3}
We can use the Gelfand-Dey method \cite{gelfand1994bayesian} when we want to calculate the Bayes factor to compare non-nested models, models where the Savage-Dickey density ratio is hard to calculate, or the Chib's methods are difficult to implement. The Gelfand-Dey method is very general, and can be used in virtually any model \cite[Chap.~5]{koop2003bayesian}.

Given a probability density function $p(\bm{\theta})$, whose support is in $\bm{\Theta}$, then
\begin{align*}
	\mathbb{E}\left[\frac{p(\bm{\theta})}{\pi(\bm{\theta}|\mathcal{M}_m)p(\bm{y}|\bm{\theta}_m,\mathcal{M}_m)}\biggr\rvert \bm{y},\mathcal{M}_m\right]&=\frac{1}{p(\bm{y}|\mathcal{M}_m)},
\end{align*} 

where the expected value is with respect to the posterior distribution given the model $\mathcal{M}_m$ (see Exercise 11).

The critical point is to select a good $p(\bm{\theta})$. \cite{geweke1999using} recommends to use $p(\bm{\theta})$ equal to a truncated multivariate normal density function with mean and variance equal to the posterior mean ($\hat{\bm{\theta}}$) and variance ($\hat{\bm{\Sigma}}$) of $\bm{\theta}$. The truncation region is $\hat{\bm{\Theta}}=\left\{\bm{\theta}:(\bm{\theta}-\hat{\bm{\theta}})^{\top}\hat{\bm{\Sigma}}^{-1}(\bm{\theta}-\hat{\bm{\theta}})\leq \chi_{1-\alpha}^2(K)\right\}$, where $\chi_{1-\alpha}^2(K)$ is the $(1-\alpha)$ percentile of the Chi-squared distribution with $K$ degrees of freedom, $K$ is the dimension of $\bm{\theta}$. We can pick small values of $\alpha$, for instance, $\alpha=0.01$.

Observe that 
\begin{align*}
	\mathbb{E}\left[\frac{p(\bm{\theta})}{\pi(\bm{\theta}|\mathcal{M}_m)p(\bm{y}|\bm{\theta}_m,\mathcal{M}_m)}\biggr\rvert \bm{y},\mathcal{M}_m\right]&\approx \frac{1}{S}\sum_{s=1}^S \left[\frac{p(\bm{\theta}^{(s)})}{\pi(\bm{\theta}^{(s)}|\mathcal{M}_m)p(\bm{y}|\bm{\theta}^{(s)}_m,\mathcal{M}_m)}\right],
\end{align*}
where $\bm{\theta}^{(s)}_m$ are draws from the posterior distribution.

Observe that we can calculate the marginal likelihoods of the models in Chapters \ref{chap6}, \ref{chap7}, \ref{chap8} and \ref{chap9} using the Chib's methods and the Gelfand-Dickey method.\\
    

\textbf{Example: Simulation exercise}

Let's check the performance of the Chib's method and Gelfand-Dey method to calculate the marginal likelihood, and consequently, the Bayes factor in a setting where we can get the analytical solution of the marginal likelihood. In particular, the Gaussian linear model with conjugate prior (see Section \ref{sec43}).

Let's assume that the data generating process is $y_{it}=0.7+0.3x_{i1}+0.7x_{i2}-0.2x_{i3}+0.2x_{i4}\mu_i$, where $\bm{x}_{i1}\sim B(0.3)$, $x_{ik}\sim N(0,1)$, $k=2,\dots,4$, and $\mu_i\sim N(0,2^2)$, $i=1,2,\dots,500$. Let's set $H_0:\beta_4={0}$ (model $\mathcal{M}_1$) versus $H_1:\ \beta_4\neq {0}$ (model $\mathcal{M}_2$).

Let's assume that $\bm{\beta}_{m0}=\bm{0}_{m0}$, $\bm{B}_{m0}=0.5\bm{I}_{m}$, $\alpha_0=\delta_0=4$. The dimensions of $\bm{0}_{m0}$ and $\bm{I}_{m}$ are 4 for model $\mathcal{M}_1$ and 5 for $\mathcal{M}_2$. In addition, let's assume equal prior probabilities.  

We know from Section \ref{sec43} that the marginal likelihood is
\begin{align*}
	p(\bf{y}|\mathcal{M}_m)&=\frac{\delta_{m0}^{\alpha_{m0}/2}}{\delta_{mn}^{\alpha_{mn}/2}}\frac{|{\bf{B}}_{mn}|^{1/2}}{|{\bf{B}}_{m0}|^{1/2}}\frac{\Gamma(\alpha_{mn}/2)}{\Gamma(\alpha_{m0}/2)},
\end{align*}
where  ${{\bm{B}}}_{mn} = ({\bm{B}}_{m0}^{-1} + {\bm{X}}_m^{\top}{\bm{X}}_m)^{-1}$, $\bm{\beta}_{mn} = {{\bf{B}}}_{mn}({\bm{B}}_{m0}^{-1}\bm{\beta}_{m0} + {\bm{X}}_m^{\top}{\bm{X}}_m\hat{\bm{\beta}}_m)$, $\alpha_{mn}=\alpha_{m0}+N$, and $\delta_{mn}=\delta_{m0}+({\bm{y}}-{\bm{X}}_m\hat{\bm{\beta}}_m)^{\top}({\bm{y}}-{\bm{X}}_m\hat{\bm{\beta}}_m)+(\hat{\bm{\beta}}_m-\bm{\beta}_{m0})^{\top}(({\bm{X}_m}^{\top}{\bm{X}_m})^{-1}+{\bm{B}}_{m0})^{-1}(\hat{\bm{\beta}}_m-\bm{\beta}_{m0})$, $m=1,2$ are the indices of the models.

The log marginal likelihoods for models $\mathcal{M}_1$ and $\mathcal{M}_2$ are -1089.82 and -1087.94, respectively. This implies a $2\times\log(BF_{21})=3.75$ which means positive evidence against model $\mathcal{M}_1$ (see Table \ref{tab:guide}).

We calculate the log marginal likelihood using the Chib's method taking into account that 
\begin{align*}
	\log(p(\bm{y}|\mathcal{M}_m))&=\log(p(\bm{y}|\bm{\theta}^*,\mathcal{M}_m))+\log(\pi(\bm{\theta}^*|\mathcal{M}_m))-\log(\pi(\bm{\theta}^*|\bm{y},\mathcal{M}_m)),\\
\end{align*}
where $p(\bm{y}|\bm{\theta}^*,\mathcal{M}_m)$ is the value of a normal density with mean $\bm{X}_m\bm{\beta}_{m}^*$ and variance $\sigma^{2*}_m\bm{I}_N$ evaluated at $\bm{y}$. In addition, $\log(\pi(\bm{\theta}^*|\mathcal{M}_m))=\log(\pi(\bm{\beta}_m^*|\sigma^{2*}_m))+\log(\pi(\sigma^{2*}_m))$, where the first term is the density of a normal with mean $\bm{\beta}_{m0}$ and variance matrix $\sigma^{2*}\bm{B}_{m0}$ evaluated at $\bm{\beta}_m^*$, and the second term is the density of an inverse-gamma with parameters $\alpha_{m0}/2$ and $\delta_{m0}/2$ evaluated at $\sigma^{2*}_m$. Finally, the third term in the right hand of the previous expression is $\log(\pi(\bm{\theta}^*|\bm{y},\mathcal{M}_m))=\log(\pi(\bm{\beta}_m^*|\sigma^{2*}_m,\bm{y}))+\log(\pi(\sigma^{2*}_m|\bm{y}))$, where the first term is the density of a normal with mean $\bm{\beta}_{mn}$ and variance matrix $\sigma^{2*}_m\bm{B}_{mn}$ evaluated at $\bm{\beta}_m^*$, and the second term is the density of an inverse-gamma with parameters $\alpha_{mn}/2$ and $\delta_{mn}/2$ evaluated at $\sigma^{2*}_m$. We use the modes of the posterior draws of $\bm{\beta}_m$ and $\sigma^2_m$ as reference values. 

We get the same value, up to two decimals, for the log marginal likelihood of the restricted and unrestricted models using the Chib's method and the analytical expression. Thus, $2\times\log(BF_{21})=3.75$, that is, positive evidence against model $\mathcal{M}_1$ (see Table \ref{tab:guide}).

We calculate the log marginal likelihood using the Gelfand-Dey method taking into account that
\begin{align*}
	\log\left[\frac{p(\bm{\theta}^{(s)})}{\pi(\bm{\theta}^{(s)}|\mathcal{M}_m)p(\bm{y}|\bm{\theta}^{(s)}_m,\mathcal{M}_m)}\right]&=\log(p(\bm{\theta}^{(s)}))-\log(\pi(\bm{\theta}^{(s)}|\mathcal{M}_m))-\log(p(\bm{y}|\bm{\theta}^{(s)}_m,\mathcal{M}_m)),
\end{align*}
where $p(\bm{\theta}^{(s)})$ is the truncated multivariate normal density of Subsection \ref{sec10_4_3} evaluated at $\bm{\theta}^{(s)}=[\bm{\beta}^{(s)\top} \ \sigma^{2(s)}]^{\top}$, which is the $s$-th posterior draw of the Gibbs sampling algorithm, such that $\bm{\theta}^{(s)}$ satisfies the truncation restriction. $\log(\pi(\bm{\theta}^{(s)}|\mathcal{M}_m))=\log(\pi(\bm{\beta}_m^{(s)}|\sigma^{2(s)}_m))+\log(\pi(\sigma^{2(s)}_m))$, where the first term is the density of a normal with mean $\bm{\beta}_{m0}$ and variance matrix $\sigma^{2(s)}\bm{B}_{m0}$ evaluated at $\bm{\beta}_m^{(s)}$, and the second term is the density of an inverse-gamma with parameters $\alpha_{m0}/2$ and $\delta_{m0}/2$ evaluated at $\sigma^{2(s)}_m$. The third term $p(\bm{y}|\bm{\theta}^{(s)},\mathcal{M}_m)$ is the value of a normal density with mean $\bm{X}_m\bm{\beta}_{m}^{(s)}$ and variance $\sigma^{2(s)}_m\bm{I}_N$ evaluated at $\bm{y}$.

The log marginal likelihoods of the restricted and unrestricted models using the Gelfand-Dey method are -1087.43 and -1084.53, respectively. This implies $2\times \log(BF_{21})=5.80$, which is positive evidence in favor of the unrestricted model.

We see in this example that these methods give good approximations to the true marginal likelihoods. However, the Chib's method did a better job than the Gelfand-Dey method. In addition, the computational demand in the Gelfand-Dey method is by far larger than the Chib's method. We can see this because the Chib's method requires just an evaluation, whereas the Gelfand-Dey method requires many evaluations based on the posterior draws. However, the Gelfand-Dey method is more general, and give uncertainty measures regarding the log marginal likelihood calculations.

The following code shows how to do this calculations.


\begin{tcolorbox}[enhanced,width=4.67in,center upper,
	fontupper=\large\bfseries,drop shadow southwest,sharp corners]
	\textit{R code. Simulation exercise: Bayes factors}
	\begin{VF}
		\begin{lstlisting}[language=R]
set.seed(010101)
N <- 500; K <- 5; K2 <- 3 
B <- c(0.7, 0.3, 0.7, -0.2, 0.2) 
X1 <- rbinom(N, 1, 0.3)
X2 <- matrix(rnorm(K2*N), N, K2)
X <- cbind(1, X1, X2)
Y <- X%*%B + rnorm(N, 0, sd = 2)
# Hyperparameters
d0 <- 4
a0 <- 4
b0 <- rep(0, K)
cOpt <- 0.5
LogMarLikLM <- function(X, c0){
	K <- dim(X)[2]; N <- dim(X)[1]	
	# Hyperparameters
	B0 <- c0*diag(K); b0 <- rep(0, K)
	# Posterior parameters
	bhat <- solve(t(X)%*%X)%*%t(X)%*%Y
	Bn <- as.matrix(Matrix::forceSymmetric(solve(solve(B0) + t(X)%*%X))) 
	bn <- Bn%*%(solve(B0)%*%b0 + t(X)%*%X%*%bhat)
	dn <- as.numeric(d0 + t(Y)%*%Y+t(b0)%*%solve(B0)%*%b0-t(bn)%*%solve(Bn)%*%bn)
	an <- a0 + N
	# Log marginal likelihood
	logpy <- (N/2)*log(1/pi)+(a0/2)*log(d0)-(an/2)*log(dn) + 0.5*log(det(Bn)/det(B0)) + lgamma(an/2)-lgamma(a0/2)
	return(-logpy)
}
LogMarM2 <- -LogMarLikLM(X = X, c0 = cOpt)
LogMarM1 <- -LogMarLikLM(X = X[,1:4], c0 = cOpt)
BF12 <- exp(LogMarM1-LogMarM2) 
BF12; 1/BF12
2*log(1/BF12}
\end{lstlisting}
	\end{VF}
\end{tcolorbox}            


\begin{tcolorbox}[enhanced,width=4.67in,center upper,
	fontupper=\large\bfseries,drop shadow southwest,sharp corners]
	\textit{R code. Simulation exercise: Bayes factors}
	\begin{VF}
		\begin{lstlisting}[language=R]
# Chib's method
sig2Post <- MCMCpack::rinvgamma(S,an/2,dn/2)
BetasGibbs <- sapply(1:S, function(s){MASS::mvrnorm(n = 1, mu = bn, Sigma = sig2Post[s]*Bn)})
# Mode function for continuous data
mode_continuous <- function(x){
	density_est <- density(x)       
	mode_value <- density_est$x[which.max(density_est$y)]  
	return(mode_value)
}
# Unrestricted model
BetasMode <- apply(BetasGibbs, 1, mode_continuous)
Sigma2Mode <- mode_continuous(sig2Post)
VarModel <- Sigma2Mode*diag(N)
MeanModel <- X%*%BetasMode
LogLik <- mvtnorm::dmvnorm(c(Y), mean = MeanModel, sigma = VarModel, log = TRUE, checkSymmetry = TRUE)
LogPrior <- mvtnorm::dmvnorm(BetasMode, mean = rep(0, K), sigma = Sigma2Mode*cOpt*diag(K), log = TRUE, checkSymmetry = TRUE)+log(MCMCpack::dinvgamma(Sigma2Mode, a0/2, d0/2))
LogPost1 <- mvtnorm::dmvnorm(BetasMode, mean = bn, sigma = Sigma2Mode*Bn, log = TRUE, checkSymmetry = TRUE)
LogPost2 <- log(MCMCpack::dinvgamma(Sigma2Mode, an/2, dn/2))
LogMarLikChib <- LogLik + LogPrior -(LogPost1 + LogPost2)
# Restricted model
anRest <- N + a0; XRest <- X[,-5]
KRest <- dim(XRest)[2]; B0Rest <- cOpt*diag(KRest) 
BnRest <- solve(solve(B0Rest)+t(XRest)%*%XRest)
bhatRest <- solve(t(XRest)%*%XRest)%*%t(XRest)%*%Y
b0Rest <- rep(0, KRest)
bnRest <- BnRest%*%(solve(B0Rest)%*%b0Rest+t(XRest)%*%XRest%*%bhatRest)
dnRest <- as.numeric(d0 + t(Y-XRest%*%bhatRest)%*%(Y-XRest%*%bhatRest)+t(bhatRest - b0Rest)%*%solve(solve(t(XRest)%*%XRest)+B0Rest)%*%(bhatRest - b0Rest))
sig2PostRest <- MCMCpack::rinvgamma(S,anRest/2,dnRest/2)
BetasGibbsRest <- sapply(1:S, function(s){MASS::mvrnorm(n = 1, mu = bnRest, Sigma = sig2PostRest[s]*BnRest)})
BetasModeRest <- apply(BetasGibbsRest, 1, mode_continuous)
Sigma2ModeRest <- mode_continuous(sig2PostRest)
VarModelRest <- Sigma2ModeRest*diag(N)
MeanModelRest <- XRest%*%BetasModeRest
LogLikRest <- mvtnorm::dmvnorm(c(Y), mean = MeanModelRest, sigma = VarModelRest, log = TRUE, checkSymmetry = TRUE)
LogPriorRest <- mvtnorm::dmvnorm(BetasModeRest, mean = rep(0, KRest), sigma = Sigma2ModeRest*cOpt*diag(KRest), log = TRUE, checkSymmetry = TRUE)+log(MCMCpack::dinvgamma(Sigma2ModeRest, a0/2, d0/2))
LogPost1Rest <- mvtnorm::dmvnorm(BetasModeRest, mean = bnRest, sigma = Sigma2ModeRest*BnRest, log = TRUE, checkSymmetry = TRUE)
LogPost2Rest <- log(MCMCpack::dinvgamma(Sigma2ModeRest, anRest/2, dnRest/2))
LogMarLikChibRest <- LogLikRest + LogPriorRest -(LogPost1Rest + LogPost2Rest)
BFChibs <- exp(LogMarLikChibRest-LogMarLikChib)
BFChibs; 1/BFChibs; 2*log(1/BFChibs)
\end{lstlisting}
\end{VF}
\end{tcolorbox}

\begin{tcolorbox}[enhanced,width=4.67in,center upper,
	fontupper=\large\bfseries,drop shadow southwest,sharp corners]
	\textit{R code. Simulation exercise: Bayes factors}
	\begin{VF}
		\begin{lstlisting}[language=R]
# Gelfand-Dey method
GDmarglik <- function(ids, X, Betas, MeanThetas, VarThetas, sig2Post){
	K <- dim(X)[2]; Thetas <- c(Betas[ids,], sig2Post[ids])
	Lognom <- (1/(1-alpha))*mvtnorm::dmvnorm(Thetas, mean = MeanThetas, sigma = VarThetas, log = TRUE, checkSymmetry = TRUE)
	Logden1 <- mvtnorm::dmvnorm(Betas[ids,], mean = rep(0, K), sigma = sig2Post[ids]*cOpt*diag(K), log = TRUE, checkSymmetry = TRUE) + log(MCMCpack::dinvgamma(sig2Post[ids], a0/2, d0/2))
	VarModel <- sig2Post[ids]*diag(N)
	MeanModel <- X%*%Betas[ids,]
	Logden2 <- mvtnorm::dmvnorm(c(Y), mean = MeanModel, sigma = VarModel, log = TRUE, checkSymmetry = TRUE)
	LogGDid <- Lognom - Logden1 - Logden2
	return(LogGDid)
}
sig2Post <- MCMCpack::rinvgamma(S,an/2,dn/2)
Betas <- LaplacesDemon::rmvt(S, bn, Hn, an)
Thetas <- cbind(Betas, sig2Post)
MeanThetas <- colMeans(Thetas); VarThetas <- var(Thetas)
iVarThetas <- solve(VarThetas)
ChiSQ <- sapply(1:S, function(s){(Thetas[s,]-MeanThetas)%*%iVarThetas%*%(Thetas[s,]-MeanThetas)})
alpha <- 0.01; criticalval <- qchisq(1-alpha, K + 1)
idGoodThetas <- which(ChiSQ <= criticalval)
pb <- winProgressBar(title = "progress bar", min = 0, max = S, width = 300)
InvMargLik2 <- NULL
for(s in idGoodThetas){
	LogInvs <- GDmarglik(ids = s, X = X, Betas = Betas, MeanThetas = MeanThetas, VarThetas = VarThetas, sig2Post = sig2Post)
	InvMargLik2 <- c(InvMargLik2, LogInvs)
	setWinProgressBar(pb, s, title=paste( round(s/S*100, 0),"% done"))
}
close(pb); mean(InvMargLik2)
# Restricted model
anRest <- N + a0; XRest <- X[,-5]
KRest <- dim(XRest)[2]; B0Rest <- cOpt*diag(KRest) 
BnRest <- solve(solve(B0Rest)+t(XRest)%*%XRest)
bhatRest <- solve(t(XRest)%*%XRest)%*%t(XRest)%*%Y
b0Rest <- rep(0, KRest)
bnRest <- BnRest%*%(solve(B0Rest)%*%b0Rest+t(XRest)%*%XRest%*%bhatRest)
dnRest <- as.numeric(d0 + t(Y-XRest%*%bhatRest)%*%(Y-XRest%*%bhatRest)+t(bhatRest - b0Rest)%*%solve(solve(t(XRest)%*%XRest)+B0Rest)%*%(bhatRest - b0Rest))
HnRest <- as.matrix(Matrix::forceSymmetric(dnRest*BnRest/anRest))
sig2PostRest <- MCMCpack::rinvgamma(S,anRest/2,dnRest/2)
BetasRest <- LaplacesDemon::rmvt(S, bnRest, HnRest, anRest)
ThetasRest <- cbind(BetasRest, sig2PostRest)
MeanThetasRest <- colMeans(ThetasRest)
VarThetasRest <- var(ThetasRest)
iVarThetasRest <- solve(VarThetasRest)
\end{lstlisting}
	\end{VF}
\end{tcolorbox}  

\begin{tcolorbox}[enhanced,width=4.67in,center upper,
	fontupper=\large\bfseries,drop shadow southwest,sharp corners]
	\textit{R code. Simulation exercise: Bayes factors}
	\begin{VF}
		\begin{lstlisting}[language=R]
ChiSQRest <- sapply(1:S, function(s){(ThetasRest[s,]-MeanThetasRest)%*%iVarThetasRest%*%(ThetasRest[s,]-MeanThetasRest)})
idGoodThetasRest <- which(ChiSQRest <= criticalval)
pb <- winProgressBar(title = "progress bar", min = 0, max = S, width = 300)
InvMargLik1 <- NULL
for(s in idGoodThetasRest){
	LogInvs <- GDmarglik(ids = s, X = XRest, Betas = BetasRest, MeanThetas = MeanThetasRest, VarThetas = VarThetasRest, sig2Post = sig2PostRest)
	InvMargLik1 <- c(InvMargLik1, LogInvs)
	setWinProgressBar(pb, s, title=paste( round(s/S*100, 0),"% done"))
}
close(pb); summary(coda::mcmc(InvMargLik1))
mean(InvMargLik1)
BFFD <- exp(mean(InvMargLik2)-mean(InvMargLik1))
BFFD; mean(1/BFFD); 2*log(1/BFFD)
\end{lstlisting}
	\end{VF}
\end{tcolorbox}  

\begin{comment}
The Bayes factor using the Savage-Dickey density ratio is

\begin{align*}
	BF_{12}&=\frac{\pi(\bm{\beta}_{5}=\bm{0}_2|\bm{y},\mathcal{M}_2)}{\pi(\bm{\beta}_{5}=\bm{0}_2|\mathcal{M}_2)},
\end{align*}
where $\pi(\bm{\beta}_{5}=\bm{0}_2|\bm{y},\mathcal{M}_2)$ is the density of a  student's t-distribution with mean equal to the last element of $\bm{\beta}_{2n}$, scale parameter $\bm{H}_{2n,5,5}$, that is, the element 5:5 of the matrix $\bm{H}_{2n}$, where $\bm{H}_{2n}=\delta_{2n}\bm{B}_{2n}/\alpha_{2n}$, and degrees of freedom $\alpha_{2n}$.

Given that the prior distribution of $\bm{\beta}|\sigma^2$ is $N(\bm{\beta}_0,\sigma^2\bm{B}_0)$, this implies that $\bm{\beta}_{5}|\sigma^2\sim N(\bm{\beta}_{0,5},\sigma^2\bm{B}_{0,5,5})$. Then,
\begin{align*}
	\pi(\bm{\beta}_{5}|\mathcal{M}_2)&=\int_{0}^{\infty}(2\pi\sigma^2)^{-1/2}|\bm{B}_0|^{-1/2}\exp\left\{-\frac{1}{2\sigma^2}(\bm{\beta}_{4:5}-\bm{\beta}_{0,4:5})^{\top}\bm{B}_{0,4:5}^{-1}(\bm{\beta}_{4:5}-\bm{\beta}_{0,4:5})\right\}\\
	&\times \frac{(\delta_0/2)^{\alpha_0/2}}{\Gamma(\alpha_0/2)}\left(\frac{1}{\sigma^2}\right)^{\alpha_0/2+1}\exp \left\{-\frac{\delta_0}{2\sigma^2} \right\}d\sigma^2
\end{align*}
\end{comment}

\begin{comment}
	Given the prior specification of Section \ref{sec10_2}, the Bayes factor is
	\begin{align*}
		BF_{12}&=\frac{\left(\frac{g_1}{1+g_1}\right)^{k_1/2} \left[(\bm{y}-\bar{y}\bm{i}_N)^{\top}(\bm{y}-\bar{y}\bm{i}_N)-\frac{1}{1+g_1}(\bm{y}^{\top}\bm{P}_{X_1}\bm{y})\right]^{-(N-1)/2}}{\left(\frac{g_2}{1+g_2}\right)^{k_2/2} \left[(\bm{y}-\bar{y}\bm{i}_N)^{\top}(\bm{y}-\bar{y}\bm{i}_N)-\frac{1}{1+g_2}(\bm{y}^{\top}\bm{P}_{X_2}\bm{y})\right]^{-(N-1)/2}},
	\end{align*}
	where $\bm{P}_{X_m}=\bm{X}_m(\bm{X}_m^{\top}\bm{X}_m)^{-1}\bm{X}_m$ is the projection matrix on the space generated by the columns of $\bm{X}_m$, $m=1,2$. In particular, $k_1=2$, $k_2=4$, $\bm{X}_1$ does not take into account $x_{i3}$ neither $x_{i4}$, whereas $\bm{X}_2$ takes into account all the regressors.
	
	The Bayes factor of the restricted model versus the unrestricted model is equal to 0.0129. This is strong evidence against model $\mathcal{M}_1$ according to the Kass and Raftery guidelines (see Table \ref{tab:guide}). 
	
	Let's calculate the Bayes factor using the Savage-Dickey density ration. We have in this setting
	\begin{align*}
		\pi(\bm{\beta}=\bm{0}_{0m}|\mathcal{M}_2)&=(2\pi\sigma^2)^{-2/7}|g_m\bm{X}_m^{\top}\bm{X}_m|^{-1/2}\\
		&\times\exp\left\{-\frac{1}{2\sigma^2}(\bm{0}_{0m}-\bm{\beta}_{0})^{\top}(g_m\bm{X}_m^{\top}\bm{X}_m)^{-1}(\bm{0}_{0m}-\bm{\beta}_{0})\right\}
	\end{align*}  
	
	We know that the marginal likelihood function of the linear Gaussian model with independent priors does not have an analytical solution. Then, let's perform a simulation exercise assuming that there are five fixed regressors, for instance, factors in treatment effects, and other five additional regressors that we are not sure about their relevance in the model. We can use BMA in this setting to check robustness of the treatment factors regarding the model specification associated with the other five regressors.
	
	Let's assume that the data generating process is $y_{it}=0.8+0.3x_{i3}+0.7x_{i7}-0.5x_{10i}+\mu_i$, where $\bm{x}_{i}\sim B(5, \bm{p})$, $\bm{p}=[0.1 \ 0.3 \ 0.3 \ 0.2 \ 0.1]^{\top}$, $x_{ik}\sim N(0,1)$, $k=6,\dots,10$, and $y_i=0.7+0.3x_{i3}+0.7x_{i7}-0.5x_{i10}+\mu_i$, $\mu_i\sim N(0,1)$.
	
	
	
	
	where all parameter are indexed to model $\mathcal{M}_m$, $\bm{P}_{X_m}=\bm{X}_m(\bm{X}_m^{\top}\bm{X}_m)^{-1}\bm{X}_m$ is the projection matrix on the space generated by the columns of $\bm{X}_m$, and $\bar{y}$ is the sample mean of $\bm{y}$. 
	
	In the Gaussian linear model with independent priors (see Section \ref{sec62}), we have ${\bf{y}}={\bf{X}}\bm{\bm{\beta}}+\bm{\mu}$ such that $\bm{\mu}\sim N(\bf{0},\sigma^2\bf{I}_N)$, $\bm{\beta} \sim N(\bm{\beta}_0, {\bf{B}}_0)$ and $\sigma^2 \sim IG(\alpha_0/2, \delta_0/2)$. Then, $\bm{\beta}|\sigma^2, {\bf{y}}, {\bf{X}} \sim N(\bm{\beta}_n, \sigma^2{\bf{B}}_n)$ and $\sigma^2|\bm{\beta}, {\bf{y}}, {\bf{X}} \sim IG(\alpha_n/2, \delta_n/2)$, where  ${\bf{B}}_n = ({\bf{B}}_0^{-1} + \sigma^{-2} {\bf{X}}^{\top}{\bf{X}})^{-1}$, $\bm{\beta}_n= {\bf{B}}_n({\bf{B}}_0^{-1}\bm{\beta}_0 + \sigma^{-2} {\bf{X}}^{\top}{\bf{y}})$, $\alpha_n = \alpha_0 + N$ and $\delta_n = \delta_0 + ({\bf{y}}-{\bf{X}}\bm{\beta})^{\top}({\bf{y}}-{\bf{X}}\bm{\beta})$.
	
	We can use as reference the model with all the regressors. Then,
	\begin{align*}
		\pi(\bm{\beta}=\bm{0}_{0k}|\mathcal{M}_2)=(2\pi)^{-2/10}|\bm{B}_{0}|^{-1/2}\exp\left\{-\frac{1}{2}(\bm{0}_{0k}-\bm{\beta}_{0})^{\top}\bm{B}_{0}^{-1}(\bm{0}_{0k}-\bm{\beta}_{0})\right\}
	\end{align*}  
	
	\begin{align*}
		\pi(\bm{\beta}=\bm{0}_{0m}|\mathcal{M}_2)&=(2\pi\sigma^2)^{-2/7}|g_m\bm{X}_m^{\top}\bm{X}_m|^{-1/2}\\
		&\times\exp\left\{-\frac{1}{2\sigma^2}(\bm{0}_{0m}-\bm{\beta}_{0})^{\top}(g\bm{X}_m^{\top}\bm{X}_m)^{-1}(\bm{0}_{0m}-\bm{\beta}_{0})\right\}
	\end{align*}  
\end{comment}
    

\section{Summary}\label{sec10_5}
In this chapter we introduced Bayesian model average in generalized linear models. In the case of linear Gaussian models, we perform BMA using three approaches: the Bayesian information criterion approximation with the Occam's window, the Markov chain Monte Carlo model composition algorithm, and the conditional Bayes factors when taking into account endogeneity. In the case of other generalized linear models, logit, gamma and Poisson, we show how to use the BIC approximation to perform BMA. In the case that the BIC approximation does not perform a good job due to small or moderate sample sizes, we present alternative ways to calculate the marginal likelihood: the Savage-Dickey density ration, Chib's method and Gelfand-Dey method.  

\section{Exercises}\label{sec10_6}

\begin{enumerate}
	\item The Gaussian linear model specifies $\bf{y}=\alpha\bm{i}_N+\bm{X}_m\bm{\beta}_m+\bm{\mu}_m$ such that $\bm{\mu}_m\sim{N}(\bm{0},\sigma^2\bm{I}_n)$, and $\bm{X}_m$ does not have the column of ones. Assuming that $\pi(\sigma^2)\propto 1/{\sigma^2}$, $\pi(\alpha)\propto 1$, and $\bm{\beta}_m|\sigma^2 \sim {N}(\bm{0}_{k_m}, \sigma^2 (g_m\bm{X}_m^{\top}\bm{X}_m)^{-1})$.
	\begin{itemize}
		\item Show that the posterior conditional distribution of $\bm{\beta}_m$ is $N(\bm{\beta}_{mn},\sigma^2\bm{B}_{mn})$, where $\bm{\beta}_{mn}=\bm{B}_{mn}\bm{X}_m^{\top}\bm{y}$ and $\bm{B}_{mn}=((1+g_m)\bm{X}_m^{\top}\bm{X}_m)^{-1}$.
		\item Show that the marginal the marginal likelihood associated with model $\mathcal{M}_m$ is proportional to
		\begin{align*}
			p(\bm{y}|\mathcal{M}_m)&\propto \left(\frac{g_m}{1+g_m}\right)^{k_m/2} \left[(\bm{y}-\bar{y}\bm{i}_N)^{\top}(\bm{y}-\bar{y}\bm{i}_N)-\frac{1}{1+g_m}(\bm{y}^{\top}\bm{P}_{X_m}\bm{y})\right]^{-(N-1)/2},
		\end{align*}
		where all parameter are indexed to model $\mathcal{M}_m$, $\bm{P}_{X_m}=\bm{X}_m(\bm{X}_m^{\top}\bm{X}_m)^{-1}\bm{X}_m$ is the projection matrix on the space generated by the columns of $\bm{X}_m$, and $\bar{y}$ is the sample mean of $\bm{y}$.
		
		Hint: Take into account that $\bm{i}_N^{\top}\bm{X}_m=\bm{0}_{k_m}$ due to all columns being centered with respect to their means.
	\end{itemize}

\item \textbf{Determinants of export diversification I}

\cite{Jetter2015} use BMA to study the determinants of export diversification. Use the dataset \textit{10ExportDiversificationHHI.csv} to perform BMA using the BIC approximation and MC3 to check if these two approaches agree. 

\item \textbf{Simulation exercise of the Markov chain Monte Carlo model composition continues}

Program an algorithm to perform MC3 where the final $S$ models are unique. Use the simulation setting of Section \ref{sec10_2} increasing the number of regressors to 40, this implies approximately 1.1e+12 models.

\item \textbf{Simulation exercise of IV BMA continues}

Use the simulation setting with endogeneity in Section \ref{sec10_2} to perform BMA based on the BIC approximation and MC3.

\item \textbf{Determinants of export diversification II}

Use the datasets \textit{11ExportDiversificationHHI.csv} and \textit{12ExportDiversificationHHIInstr.csv} to perform IV BMA assuming that the log of per capita gross domestic product is endogenous (\textit{avglgdpcap}). See \cite{Jetter2015} for details.

\item Show that the link function in the case of the Bernoulli distribution is $\log\left(\frac{\theta}{1-\theta}\right)$.

\item \cite{ramirez2020dynamic,ramirez2021specification} perform variable selection using the file \textit{13InternetMed.csv}. In this data set, the dependent variable is an indicator of Internet adoption (internet) for 5000 households in Medell\'in (Colombia) during the period 2006--2014. This dataset contains information about 18 potential determinants, which means 262144 ($2^{18}$) potential models just taking into account variable uncertainty (see these papers for details about the data set). Perform BMA using the logit link function using this data set.  

\item \cite{Serna2018} use the file \textit{14ValueFootballPlayers.csv} to analyze the market value of soccer players in the most important leagues in Europe. In particular, there are 26 potential determinants of the market value (dependent variable) of a stratified sample of 335 soccer players in the five most important leagues in Europe (see \cite{Serna2018} for details). Use this data set to perform BMA using the gamma distribution setting default values for Occam's window.  

\item Use the dataset \textit{15Fertile2.csv} from \cite[p.~547]{Wooldridge2012} to perform BMA using the Poisson model with the log link. This data set has information about 1,781 women from Botswana in 1988 (for details, see \textbf{https://rdrr.io/cran/wooldridge/man/fertil2.html}, and take into account that we deleted some variables and omitted observations with NA values). The dependent variable is the number of children ever born (ceb), which is a count variable, as a function of 19 potential determinants.

\item Perform BMA in the logit model using MC3 and the BIC approximation using the simulation setting of Section \ref{sec10_3}.

\item Show that 
\begin{align*}
	\mathbb{E}\left[\frac{p(\bm{\theta})}{\pi(\bm{\theta}|\mathcal{M}_m)p(\bm{y}|\bm{\theta}_m,\mathcal{M}_m)}\biggr\rvert \bm{y},\mathcal{M}_m\right]&=\frac{1}{p(\bm{y}|\mathcal{M}_m)},
\end{align*}
where the expected value is with respect to the posterior distribution given the model $\mathcal{M}_m$.        
	
\end{enumerate}