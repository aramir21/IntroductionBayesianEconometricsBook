\chapter{Approximate Bayesian methods}\label{chap15}

\textit{Approximate Bayesian methods} are a family of techniques designed to handle situations where the likelihood function lacks an analytical expression, is highly complex, or the problem has high dimensionality, whether due to a large parameter space or a massive dataset \cite{martin2024approximating}. In the former case, traditional Markov Chain Monte Carlo (MCMC) and importance sampling algorithms fail to provide a solution, while in the latter, these algorithms struggle to produce accurate estimates within a reasonable time.  

However, there is no free lunch, \textit{Approximate Bayesian methods} address these challenges at the cost of providing an approximation to the posterior distribution rather than the \textit{exact} posterior. Nonetheless, asymptotic results show that the approximation improves as the sample size increases.

In this chapter, I first present \textit{simulation-based approaches}, which are designed to address situations where the likelihood is highly complex and may lack an analytical solution. In the second part, I introduce \textit{optimization approaches}, which are intended to handle high-dimensional problems. Specifically, I discuss approximate Bayesian computation (ABC) and Bayesian synthetic likelihood (BSL), the two most common \textit{simulation-based approaches}. Then, I present integrated nested Laplace approximations (INLA) and \textit{variational Bayes} (VB), the two most common \textit{optimization approaches} for high-dimensional problems.

\section{Simulation-based approaches}\label{sec15_1}
Taking into account the fundamental equation for performing parameter inference in the Bayesian framework,  
\begin{align*}
	\pi(\boldsymbol{\theta} \mid \mathbf{y}) & \propto p(\mathbf{y} \mid \boldsymbol{\theta}) \times \pi(\boldsymbol{\theta}),
\end{align*}  
we see in Section \ref{sec51} that MCMC algorithms, such as the Gibbs sampler (Section \ref{sec511}) and Metropolis-Hastings (Section \ref{sec512}), require evaluation of the likelihood function \( p(\boldsymbol{y} \mid \boldsymbol{\theta}) \) in the posterior conditional distribution or the acceptance probability, respectively. This is also the case for importance sampling when calculating the importance weights (Section \ref{sec52}).  

Thus, what happens when the likelihood function does not have an analytical expression? This situation arises in many models involving unobserved heterogeneity (i.e., unobserved taste preferences), models defined by quantile functions (e.g., the g-and-k distribution), or dynamic equilibrium model (e.g., repeated game models).

\textit{Simulation-based algorithms} provide a Bayesian solution when we face this situation, namely, when the likelihood function lacks an analytical expression or is highly complex. The only requirement is that we must be able to simulate synthetic data from the model conditional on the parameters. Therefore, these algorithms obtain an approximation to the posterior draws by simulating from the prior distribution $\pi(\boldsymbol{\theta})$ and then using these draws to simulate from the likelihood $p(\mathbf{y} \mid \boldsymbol{\theta})$.

\subsection{Approximate Bayesian computation}\label{sec15_12}

\textit{Approximate Bayesian Computation} (ABC) is designed to handle inferential situations where the likelihood function \( p(\boldsymbol{y} \mid \boldsymbol{\theta}) \) is intractable or highly complex. It was introduced in population genetics by \cite{tavare1997inferring, pritchard1999population} and later generalized by \cite{beaumont2002approximate}. The basic intuitive origin of ABC appears to have been introduced by \cite{rubin1984bayesianly}. A growing body of literature explores its applications in biology, cosmology, finance, economics, and other fields.

The requirement in ABC is the ability to simulate from the parametric model. The process begins by drawing samples from the prior distribution \( \pi({\boldsymbol{\theta}}) \) multiple times, $\boldsymbol{\theta}\in\mathbb{R}^K$, and then simulating data from the model given each \( {\boldsymbol{\theta}^{(s)}}, s=1,2,\dots,S \). The resulting synthetic data, \( \boldsymbol{z}^{(s)} \in \mathbb{R}^n \) is used to compute summary statistics \( \boldsymbol{\eta}(\boldsymbol{z}^{(s)}) \in \mathbb{R}^L, L\geq K \).  

Next, we compare the synthetic summary statistics with the observed summary statistics \( \boldsymbol{\eta}(\boldsymbol{y}) \) using a distance metric \( d\left\{ \boldsymbol\eta ({\boldsymbol y}),{\boldsymbol \eta }({\boldsymbol z}^{(s)})\right\} \), typically the Euclidean distance. We retain the prior draws that generate synthetic summary statistics closest to the observed ones, that is, \( d\left\{ \boldsymbol\eta ({\boldsymbol y}),{\boldsymbol \eta }({\boldsymbol z}^{(s)})\right\}\leq \epsilon \), forming an approximation of the posterior distribution \( \pi_{\epsilon}(\boldsymbol{\theta},\boldsymbol{z} \mid \boldsymbol{\eta}(\boldsymbol{y})) \).

The simplest algorithm is the accept/reject approximate Bayesian computation (ABC-AR) (see Algorithm \ref{ABC0}).

\begin{algorithm}
	\caption{Accept/reject ABC}\label{ABC0}
	\begin{algorithmic}[1]
		\For{\texttt{$s=1,\dots,S$}}
		\State Draw ${\boldsymbol {\theta} }^{s}$ from $\pi({ \boldsymbol{\theta} }),$
		\State Simulate ${\boldsymbol z}^{s}=(z_{1}^{s},z_{2}^{s},...,z_{n}^{s})^{\top}$from the model, $p(\cdot|{\boldsymbol{\theta} }^{s})$
		\State Calculate
		$d_{(s)}=d\{{\boldsymbol\eta }({\boldsymbol y}),{\boldsymbol \eta }({\boldsymbol z}^{s})\}$
		\EndFor
		\State Order the distances $d_{(1)}\leq\cdots\leq d_{(S)}$
		\State Select all $\boldsymbol{\theta}^s$ such that $d_{(i)}\leq \epsilon$, where $\epsilon>0$ is the tolerance level. 
	\end{algorithmic}
\end{algorithm}

Note that the posterior distribution is conditional on the summary statistics \( \boldsymbol{\eta}(\boldsymbol{y}) \) and the tolerance parameter \( \epsilon \). This implies that we obtain an approximation to the target distribution \( \pi(\boldsymbol{\theta} \mid \boldsymbol{y}) \) because \( \boldsymbol{\eta}(\boldsymbol{y}) \) is not a sufficient statistic in most cases, and \( \epsilon > 0 \), these conditions introduce bias \cite{blum2010approximate}. However, ABC performs well compared to full-likelihood approaches in low-dimensional parameter spaces \cite{beaumont2002approximate}.

Furthermore, \cite{frazier2018asymptotic} show in Theorems 1 and 2 that Bayesian consistency and asymptotic normality hold, provided that \( \epsilon \to 0 \) fast enough as \( n \to +\infty \). In particular, the requirement is that the proportion of accepted draws converges to 0 at a rate of \( n^{-K / 2} \). Additionally, Theorem 2 in \cite{frazier2018asymptotic} shows that \( 100(1 -\alpha)\% \) Bayesian credible regions using ABC have frequentist coverage of \( 100(1 -\alpha)\% \). 

We should note from these asymptotic results that ABC suffers from the \textit{curse of dimensionality}. Specifically, given a sample size of 1,000 and two parameters, the proportion of accepted draws should be 0.1\%, meaning we would require one million prior draws to obtain 1,000 posterior draws. On the other hand, if the number of parameters is three, we would require 31.62 million prior draws. This limitation of ABC has attracted attention; see Chapter 8 of \cite{sisson2018handbook} for some potential solutions.

It is a common practice in ABC to perform a regression adjustment after retaining the draws \cite{beaumont2002approximate, leuenberger2010bayesian, sisson2018handbook}. This adjustment reduces bias in posterior draws by performing a simple linear regression between the selected draws and the discrepancy between the observed and simulated summary statistics, \( \boldsymbol{\theta}^{(s)} = \alpha + \beta \left(\boldsymbol{\eta}(\boldsymbol{y}) - \boldsymbol{\eta}(\boldsymbol{z}^{(s)})\right) + \mu^{(s)} \). Then, the posterior draws are adjusted using the slope estimate \( \boldsymbol{\theta}^{\text{adj},(s)} = \boldsymbol{\theta}^{(s)} - \hat{\beta} \left(\boldsymbol{\eta}(\boldsymbol{y}) - \boldsymbol{\eta}(\boldsymbol{z}^{(s)})\right) \).

The accept/reject ABC algorithm is inefficient, as all draws are independent; thus, there is no learning from previous draws. This intensifies the computational burden. Therefore, \cite{marjoram2003markov, wegmann2009efficient} introduced Markov Chain Monte Carlo ABC (ABC-MCMC) algorithms, and \cite{sisson2007sequential, drovandi2011estimation, del2012adaptive, lenormand2013adaptive} proposed sequential Monte Carlo approaches (ABC-SMC). However, results comparing ABC-MCMC and ABC-SMC with ABC-AR are controversial regarding computational efficiency \cite{bertorelle2010abc}. In addition, ABC-AR is very simple and easily allows parallel computing \cite{frazier2019approximate}. Nevertheless, ABC-SMC is now the recommended approach, as it does not require tuning the algorithm's tolerance \cite{martin2024approximating}, and there are open-source implementations that facilitate its use.

New developments in ABC have focused on using empirical measures calculated from the observed ($\hat{\mu}_n$) and synthetic ($\hat{\mu}_{\boldsymbol{\theta}}^{(s)}$) data to replace summary statistics. Thus, $d\left\{ \boldsymbol\eta ({\boldsymbol y}),{\boldsymbol \eta }({\boldsymbol z}^{(s)})\right\}$ is replaced by $\mathcal{D}\left\{ \hat{\mu}_n,\hat{\mu}_{\boldsymbol{\theta}}^{(s)}\right\}$, where the latter is a discrepancy measure, such as the Kullback-Leibler divergence \cite{jiang2018approximate}. However, \cite{drovandi2022comparison} found in their simulation exercises that the best-performing summary statistics approach performs at least as well as the best discrepancy-measure approaches. The key point is to select informative summary statistics.\\

\textbf{Example: g-and-k distribution for financial returns} 

\textit{EasyABC} package in \textbf{R}.

\subsection{Bayesian synthetic likelihood}\label{sec15_13}

\section{Optimization approaches}\label{sec15_2}

Even in situations where the likelihood function has an analytical expression, but there are huge datasets, MCMC and IS algorithms require pointwise evaluation of the likelihood function. This implies a huge number of operations, and these methods are not designed to be \textit{scalable}. Further, if there is a large parameter space, these methods are neither \textit{scalable} in the number of parameters.

\textit{Optimization approaches} are designed to scale to high parameter spaces and large datasets. The trick is to change simulation with optimization. 

\subsection{Integrated nested Laplace approximations}\label{sec15_21}
\textit{Integrated nested Laplace approximations} approximates $\pi(\boldsymbol{\theta} \mid \mathbf{y})$ by a combination of low-dimensional deterministic integration and optimization steps. 

\subsection{Variational Bayes}\label{sec15_22}
\textit{Variational Bayes} replaces $\pi(\boldsymbol{\theta} \mid \mathbf{y})$ by an approximation produced by calculus of variations. 

\section{Summary}\label{sec15_3}
\textit{Simulation-based algorithms} suffer from the \textit{curse of dimensionality} in parameter space, whereas \textit{Optimization approaches} require evaluation of the likelihood functions. Thus, recent approaches mix these two approaches to overcome situations where both phenomena are present.

\section{Exercises}\label{sec15_4}

\begin{enumerate}
	\item Simulation exercise in ABC: A model with endogeneity.
\end{enumerate}

