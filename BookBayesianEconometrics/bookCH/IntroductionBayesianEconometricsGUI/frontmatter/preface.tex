\chapter*{Preface}
The main goal of this book is to make more approachable the Bayesian econometric framework to students, researchers and practitioners who want to understand and apply this statistical/econometric approach, but who do not have time to develop programming skills. I tried to have a balance between applicability and theory. Then, this book comes with a very friendly graphical user interface (GUI) to implement the most common econometric models, but also contains the basic mathematical developments, as well as their code implementation, for those who are interested in advancing in more complex models.

\textbf{To instructors and students}

This book is divided in three parts, foundations (chapters 1 to 5), regression analysis (chapters 6 to 10), and \textit{Advanced} methods (chapters 11 to 15). Our graphical user interface (GUI) targets the second part. This can be download at \textbf{https://github.com/besmarter/BSTApp}. Instructors and students can have all codes, simulated and real data sets are there. To install our GUI just type \textbf{shiny::runGitHub(``besmarter/BSTApp" , launch.browser=T)} in the \textbf{R} package console or any \textbf{R} code editor, and execute it.

Students should have some basic knowledge in probability theory and statistics, particularly, regression analysis. It is strongly recommended to have some familiarity with standard univariate and multivariate probability distributions.

I included some formal and computational exercises at the end of each chapter. This would help students to have a better understanding of the material shown in each chapter. A manual with the solutions of exercises accompanies this book.

Instructors can use this book as a text in a course of introduction to Bayesian Econometrics with a high emphasis on implementation and applications. This book is complentary, rather than substitute, of excellent books in the topic such as \cite{rossi2012bayesian,greenberg2012introduction, geweke2005contemporary, lancaster2004introduction} and \cite{koop2003bayesian}.

\textbf{Acknowledgments}

I started our GUI in the 2016 after being diagnosed with cervical dystonia. I used to work in this side project on weekends, I named this time ``nerd weekends", and it was a kind of release from my health condition. Once I got better, I invited Mateo Graciano, my former student, business partner and friend, to be part of the project, he helped me a lot developing our GUI, and I am enormously thankful to Mateo. I would also like to thank members of the BEsmarter research group from Universidad EAFIT, and NUMBATs members from Monash University for your comments and recommendations to improve our GUI.

This book is an extension of the paper \textit{A GUIded tour of Bayesian regression} \cite{Ramirez2020}, which is a brief user guide of our GUI. So, I decided to write this book to show the underlying theory and codes in our GUI, and use it as a text book in my course in Bayesian econometrics. I acknowledge and offer my gratitude to my students in this subject, their insight and thoughtful questions have helped me to get a better understanding of this material.   

I also thank Chris Parmeter for your suggestions about how to present our user guide, Professor Raul Pericchi and Juan Carlos Correa who introduced me to Bayesian statistics, Liana Jacobi and Chun Fung Kwok (Jackson) from the University of Melbourne and David Frazier from Monash University for nice talks and amazing collaborations in Bayesian Econometrics, Professor Peter Diggle to support my career, and particularly, Professor Gael Martin, who gave me a chance to work with her, she is an inspiring intellectual figure. Finally, my colleagues and staff from Universidad EAFIT have always given me their support.

To my parents, Orlando and Nancy, who have given me their unconditional support. They have taught me that the primary aspect of the human being's spiritual evolution is humility. I am in my way to learn this.


